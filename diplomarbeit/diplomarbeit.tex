\documentclass[a4paper,ngerman,naustrian,DIV=12,BCOR=1cm]{scrbook}
\usepackage[T1]{fontenc}
\usepackage[utf8]{inputenc}
\usepackage{fancyhdr}
\pagestyle{fancy}
\setcounter{secnumdepth}{3}
\usepackage{babel}
\usepackage{textcomp}
\usepackage{url}
\usepackage{makeidx}
\makeindex
\usepackage{graphicx}
\PassOptionsToPackage{normalem}{ulem}
\usepackage{ulem}
\usepackage[unicode=true,
 bookmarks=true,bookmarksnumbered=false,bookmarksopen=false,
 breaklinks=true,pdfborder={0 0 0},backref=false,colorlinks=false]
 {hyperref}
\hypersetup{pdftitle={sblit},
 pdfauthor={Martin Exner, Andreas Novak, Nikola Szucsich},
 pdfsubject={Diplomarbeit},
 pdfkeywords={Diplomarbeit}}

\makeatletter

%%%%%%%%%%%%%%%%%%%%%%%%%%%%%% LyX specific LaTeX commands.
\pdfpageheight\paperheight
\pdfpagewidth\paperwidth

%% Because html converters don't know tabularnewline
\providecommand{\tabularnewline}{\\}

%%%%%%%%%%%%%%%%%%%%%%%%%%%%%% Textclass specific LaTeX commands.
\newcommand{\strong}[1]{\textbf{#1}}
\newcommand{\code}[1]{\texttt{#1}}

%%%%%%%%%%%%%%%%%%%%%%%%%%%%%% User specified LaTeX commands.
%%%%%%%%%%%%
% Latex-Vorspann
\usepackage{lastpage}
\usepackage{listings}
\usepackage{blindtext}

%% geht nicht mit jeder Latex Variante, gibt aber ein schöneres Layout
\usepackage{microtype} 

%% Aufzählungen nicht so weit einrücken
\usepackage{enumitem}
%\setitemize{leftmargin=*} 

% Serifenschrift für Überschriften
\addtokomafont{disposition}{\rmfamily}

% Serifenschrift für Description-Lists
\setlist[description]{font=\rmfamily}

%\usepackage{caladea}
%\usepackage[T1]{fontenc}
\usepackage{lmodern}

%% für pandoc
%% maximale Breite der Bilder
\setkeys{Gin}{width=0.90\linewidth,keepaspectratio}

\makeatother

\usepackage{listings}
\addto\captionsnaustrian{\renewcommand{\lstlistingname}{Listing}}
\addto\captionsngerman{\renewcommand{\lstlistingname}{Listing}}
\renewcommand{\lstlistingname}{Listing}


% glossaries
\usepackage[toc,acronym]{glossaries}
\makeglossaries

% DO NOT ADD ENTRIES HERE
% Add them in your *_glossary.tex file instead

\newglossaryentry{partnership}{
	name=Partnerschaft,
	plural=Partnerschaften,
	description={Um Dateien/Dateiblöcke extern speichern zu können, muss eine
	sogenannte Partnerschaft mit anderen Geräten eingangen werden, bei der man
	Dateiblöcke auf den jeweilig anderen Geräten speichert, während man selber
	Speicherplatz für diese Partner freigibt, in dem deren verschlüsselte
	Dateiblöcke	gespeichert werden}
}

\newglossaryentry{syncpartner}{
	name=Synchronisationspartner,
	plural=Synchronisationspartner,
	description={Gerät mit dem der \sblit-Ordner synchronisiert wird}
}

\newglossaryentry{p2plink}{
	name=Peer-to-Peer-Link,
	plural=Peer-to-Peer-Links,
	description={Ein direkte Verbindung zwischen zwei Endgeräten}
}

\newglossaryentry{syncconflict}{
	name=Synchronisationskonflikt,
	plural=Synchronisationskonflikte,
	description={Wenn eine synchronisierte Datei von zwei Rechnern bearbeitet wird,
	ohne, dass sie in der Zwischenzeit synchronisert werden konnte. Es existieren
	dann zwei Versionen der gleichen Datei. Man spricht dann von einem Konflikt}
}

\newglossaryentry{filecloud}{
	name=Filecloud,
	description={Externer Speicher im Internet, auf dem Dateien gespeichert werden
	können. Meistens bestehend aus einem oder mehreren Servern}
}


\newglossaryentry{cl}{
	name=Kommunikationsschicht,
	description={Eine Schicht im netzwerktechnischen Sinne, die zur Kommunikation verwendet wird},
	see={[Siehe:]{gls_dcl}}
}

\newglossaryentry{p2pnet}{
	name=Peer-to-Peer-Netzwerk,
	description={Ein Kommunikationsnetzwerk, in dem die einzelnen Kommunikationspartner im Gegensatz zu einer klassischen Client-Server-Infrastruktur direkt miteinander kommunizieren}
}

\newglossaryentry{gls_dcl}{
	name=Decentralized Communication Layer,
	description={Die auf einem dezentralen \gls{p2pnet} basierende \gls{cl}, die \sblit zur Kommunikation benutzt}
}

\newacronym[see={[Glossar:]{gls_dcl}}]{dcl}{DCL}{Decentralized Communication Layer\glsadd{gls_dcl}}

\newglossaryentry{aenc}{
	name=Asymmetrisches Verschlüsselungsverfahren,
	description={Verschlüsselungsverfahren, bei dem zur Entschlüsselung einer Nachricht ein anderer Schlüssel verwendet wird, als zur Verschlüsselung verwendet wurde},
	see={[Siehe:]{rsa}}
}

\newglossaryentry{rsa}{
	name=RSA,
	description={\gls{aenc}, benannt nach und erstmals veröffentlich von Ron Rivest, Adi Shamir and Leonard Adleman \cite{wikipedia:rsa}},
	see={[Siehe:]{aenc}}
}

\newglossaryentry{hash}{
	name=Hashalgorithmus,
	plural=Hashalgorithmen,
	description={Algorithmus, der eine große Eingabemenge auf eine kleinere Zielmenge (\gls{hashval}) abbildet \cite{wikipedia:hash}}
}

\newglossaryentry{Hash}{
	name=Hash,
	description={Im Quellcode von \gls{dcl} definierte Klasse zur Anwendung von \glspl{hash} auf \gls{Data}-Objekte}
}

\newglossaryentry{hashval}{
	name=Hashwert,
	plural=Hashwerte,
	description={Ergebnis bei Anwendung eines \gls{hash}}
}

\newglossaryentry{digestlength}{
	name=Digest Length,
	description={Die Länge der \glspl{hashval} eines \gls{hash}}
}

\newglossaryentry{cnt}{
	name=CircleNetworkType,
	description={Standard \gls{nt} im \gls{dcl}},
	see={[Siehe:]{nt}}
}

\newglossaryentry{nt}{
	name=Network Type,
	description={Beschreibt ein Netzwerk innerhalb des \gls{dcl} samt Adresskonzept und Routingverfahren},
	see={[Siehe:]{cnt}}
}

\newglossaryentry{Data}{
	name=Data,
	description={Im Quellcode von \gls{dcl} definierte Klasse zur Speicherung von Binärdaten}
}

\newglossaryentry{partnerdevice}{
	name=Partnergerät,
	description={Ein Gerät eines anderen \sblit-Nutzers, mit dem eine Partnerschaft besteht}
}
\newglossaryentry{authreq}{
	name=Authenticity Request,
	description={Eine Nachricht zur Anforderung einer Challenge, um die Authentizität des Gegenübers sicherzustellen}
}
\newglossaryentry{authres}{
	name=Authenticity Response,
	description={Die Antwort auf den \gls{authreq}, um die eigene Authentizität zu beweisen}
}
\newglossaryentry{filereq}{
	name=File Request,
	description={Die Anfrage an das Gegenüber, ob eine Datei benötigt wird}
}
\newglossaryentry{fileres}{
	name=File Response,
	description={Die Antwort auf einen File Request, der eine Datei akzeptiert oder verweigert}
}
\newglossaryentry{filemsg}{
	name=File Message,
	description={Die eigentliche Übertragung einer Datei}
}
\newglossaryentry{filedel}{
	name=File Delete Message,
	description={Eine Nachricht, die dazu dient, Geräte über eine Löschung einer Datei zu informieren}
}
\newglossaryentry{logfile}{
	name=Logfile,
	description={Die Datei, in der der Versionsverlauf einer Datei steht. Außerdem werden hier die Geräte, auf denen die Datei schon vorhanden ist, gelistet}
}
\newglossaryentry{refdev}{
	name=Device Refresh Message,
	description={Eine Nachricht, die eine aktualisierte Version von Partnergeräten oder eigenen Geräten enthält}
}
\newglossaryentry{partfilereq}{
	name=Partner File Request,
	description={Ein \gls{filereq} an ein Partnergerät \referenz{Partnergerät} oder von einem Partnergerät}
}
\newglossaryentry{partfileres}{
	name=Partner File Response,
	description={Ein \gls{fileres} an ein Partnergerät \referenz{Partnergerät} oder von einem Partnergerät}
}
\newglossaryentry{partfilemsg}{
	name=Partner File Message,
	description={Eine \gls{filemsg} an ein Partnergerät \referenz{Partnergerät} oder von einem Partnergerät}
}
\newglossaryentry{partfiledel}{
	name=Partner File Delete Message,
	description={Eine \gls{filedel} an ein Partnergerät \referenz{Partnergerät} oder von einem Partnergerät}
}
\newglossaryentry{watchservice}{
	name=\code{WatchService},
	description={Eine Schnittstelle zum Filesystem, welche benachrichtigt wird, wenn sich eine Datei in einem bestimmten Ordner ändert}
}
\newglossaryentry{watchkey}{
	name=\code{WatchKey},
	description={Ein Objekt, das vom \gls{watchservice} erstellt wird, wenn sich etwas in dem zu überwachenden Ordner ändert}
}
%\newacronym[see={[Glossar:]{authreq}}]{authreq}{AuthReq}{Authenticity Request\glsadd{authreq}}



\newglossaryentry{gls_gui}{
	name=Graphical User Interface,
	description={Eine grafische Oberfläche für Benutzer, um eine Anwendung komfortabel bedienen zu können}
}

\newglossaryentry{versionconflict}{
	name=Versionskonflikt,
	plural=Versionskonflikte,
	description={Wenn eine synchronisierte Datei auf zwei Rechnern gleichzeitig
	bearbeitet wird, tritt ein Konflikt auf, da zwei Versionen der gleichen Datei
	existieren und somit keine der Versionen synchronisiert und von allen
	übernommen werden kann.}
}

\newacronym[see={[Glossar:]{gls_gui}}]{gui}{GUI}{Graphical User Interface\glsadd{gls_gui}}

\newglossaryentry{Prix Ars Electronica}{
	name=Prix Ars Electronica,
	description={Ist der traditionsreichste Medienkunstwettbewerb wird gemeinsam
  von der Ars Electronica Linz GmbH und dem ORF Oberösterreich in Zusammenarbeit mit dem OK Offenes
  Kulturhaus Oberösterreich und dem Brucknerhaus Linz veranstaltet.
}
}



\begin{document}
%%%%%%
% Weitere Einstellungen siehe Latex-Vorspann%%%%%%%%%%%%%%%%%%%%%%%%%%%%%%%%%%%%%%%%%%%%%%%%%%%%%%%%%%%%%%%%%%%%%%%%%%%%%%%%%%
% falls man die erste Zeile der Absätze nicht einrücken will
% dann sollte man aber etwas mehr Abstand zwischen den Absätzen erlauben
%%\setlength{\parindent}{0pt}
%%\setlength{\parskip}{1.5ex plus0.5ex minus0.5ex}
% Auch Fußnoten bündig ausrichten
\deffootnote[]{1em}{1em}{\textsuperscript{\thefootnotemark\ }}
% Listen etwas wenige einrücken, erfordert enumitem
\setitemize{leftmargin=*}

%%%%%%%%%%%%%%%%%%%%%%%%%%%%%%%%%%%%%%%%%%%%%%%%%%%%%%%%%%%%%%%%%%%%%%%%%%%%%%%%%%
%  Kopf und Fußzeilen -- links und rechts verschieden 
\newcommand{\kopfseitenummer}{{\bfseries \thepage}}
\newcommand{\kopfkapl}{{\bfseries\leftmark}}
\newcommand{\kopfkapr}{{\bfseries\rightmark}}
\newcommand{\kopfbild}{\includegraphics[width=25mm]{HTL3RLogoRGB}}
\newcommand{\kopfHTL}{Höhere Technische Bundeslehranstalt Wien 3, \\Rennweg 	Abteilung für Informationstechnologie}
\renewcommand{\chaptermark}[1]%
  {\thispagestyle{fancy}\markboth{\thechapter.\ #1}{}}%\thispagestyle{fancy}
\renewcommand{\headrulewidth}{0pt}
%\lhead[\fancyplain{\kopfbild}{\kopfbild}]% li aussen
%      {\fancyplain{\kopfHTL}{\kopfHTL}}% re innen
%\rhead[\kopfHTL]% li innen
%      {\kopfbild}% re aussen

%% mit kapitelautor kann man den Autor festlegen oder auf leer setzen - steht dann in der Fußzeile.
\newcommand{\kapitelautor}{}

%%%
% Alternative: am Rand (Marginale)
%\setlength{\marginparsep}{-5mm}
%\mbox{}\marginpar{\raggedleft\hspace{0pt}Autor: Hans Huber}

%% kopf links: [linke] und {rechte} Seite

\lhead[\kopfbild]{\fancyplain{}{\kopfkapl}}
\rhead[\fancyplain{}{\kopfkapr}]{\kopfbild}
\chead{}

\lfoot[\kopfseitenummer]{\kapitelautor}
\cfoot[]{}
\rfoot[\kapitelautor]{\kopfseitenummer}

%%
% einfaches "siehe ..." - das Ziel muss man markieren
\newcommand{\kap}[1]{Kapitel~\ref{#1}, Seite~\pageref{#1}}
\newcommand{\siehe}[1]{siehe \kap{#1}}


%### TAGS COMMAND ####
\newcommand{\tags}[2]{\hspace{0in}#2}
%\newcommand{\tags}[1]{\textit{Tags: #1}}

%%%%%Anfang Titelseite
\pagenumbering{roman}
\title{Diplomarbeit}
\begin{titlepage}
\begin{minipage}[b]{1\columnwidth}
\parbox[b]{50mm}{\includegraphics[width=45mm]{HTL3RLogoRGB}}
\hfill
\parbox[b]{130mm}{\footnotesize \textsc{Höhere Technische Bundeslehranstalt} Wien 3, Rennweg\\
IT \& Mechatronik\\
\\
HTL Rennweg :: Rennweg 89b\\
A-1030 Wien :: Tel +43 1 24215-10 :: Fax DW 18
}\\
\mbox{}
\end{minipage}

\vspace{1cm}


\begin{center}
\textbf{\LARGE{}Diplomarbeit}{\large{}}\\
{\large{}\vspace{15mm}
 }\textbf{\large{}sblit}\\
 \vspace{15mm}
 ausgeführt an der\\
 Höheren Abteilung für Informationstechnologie/Netzwerktechnik\\
 der Höheren Technischen Lehranstalt Wien 3 Rennweg\\
 \vspace{1cm}
 im Schuljahr 2014/2015\\
 \vspace{1cm}
 durch\\
 \vspace{0.5cm}
\textbf{\large{}Martin Exner}\\
\textbf{\large{}Andreas Novak}\\
\textbf{\large{}Nikola Szucsich}\\

\par\end{center}{\large \par}

\begin{center}
\vspace{20mm}
 \normalsize unter der Anleitung von\\
 \vspace{0.5cm}
 August Hörandl
\par\end{center}

\begin{center}
\vspace{5mm}
Wien, \today 
\par\end{center}

\end{titlepage}%%%%%%%%%%%%%%%%%%%%% Ende Titelseite %%%%%%%%%%%%%%%%%%%%%%


\addchap*{Kurzfassung}

% Auf Seiten mit einem neuen Kapitel ist keine Kopfzeile -- kann man sich aber wünschen
\thispagestyle{fancy}

Diese Diplomarbeit zielt darauf ab, einen Dienst zur Synchronisation von Dateien
über das Internet in der Form freier Software zu schaffen.
Zentrale Stellen wie Server sollen dabei im Sinne der Sicherheit vor Eingriffen
durch Unbefugte nicht eingesetzt werden.
Statt der Zwischenspeicherung von Daten auf Cloudspeicher eines herkömmlichen
Anbieters werden Daten in verschlüsselten Blöcken auf den Geräten anderer Nutzer
zwischengespeichert.

An die Stelle einer klassischen Client-Server-Infrastruktur tritt ein
dezentrales Peer-to-Peer-Netzwerk.
Dadurch und durch die Verteilung von Daten auf Geräte anderer Teilnehmer
auftretende Problemstellungen sollen im Rahmen dieser Diplomarbeit gelöst
werden.
Dazu zählen dezentrale Adressierung, dezentrales Routing, Fairness von
gegenseitiger Speicherfreigabe, Sicherheit und Effizienz von Kommunikation und
Synchronisation sowie Behandlung von Synchronisationskonflikten.



\addchap*{Abstract}

% mit Kopfzeile
\thispagestyle{fancy}

That's why.

\blindtext[1]



\addchap*{Ehrenwörtliche Erklärung}

% mit Kopfzeile
\thispagestyle{fancy}

Ich versichere, 
\begin{itemize}
\item dass ich meinen Anteil an dieser Diplomarbeit selbstständig verfasst
habe, 
\item dass ich keine anderen als die angegebenen Quellen und Hilfsmittel
benutzt habe 
\item und mich auch sonst keiner unerlaubten Hilfe bzw. Hilfsmittel bedient
habe.
\end{itemize}
\bigskip{}
Wien, am \today

<eigenhändige Unterschriften aller Teammitglieder>


\addchap*{Präambel}

\thispagestyle{fancy}

Die Inhalte dieser Diplomarbeit entsprechen den Qualitätsnormen für
,,Ingenieurprojekte`` gemäß §\,29 der Verordnung des Bundesministers
für Unterricht und kulturelle Angelegenheiten über die Reife- und
Diplomprüfung in den berufsbildenden höheren Schulen, BGBl. Nr. 847/1992,
in der Fassung der Verordnungen BGBl. Nr. 269/1993, Nr. 467/1996 und
BGBl. II Nr. 123/97.

\vspace{10mm}


Liste der betreuenden Lehrer: 

Prof. DI August Hörandl, \textit{Hauptbetreuer}

Prof. DI Franz Breunig, \textit{Betreuer}

Prof. DI Herbert Sasshofer, \textit{Betreuer}

\vspace{10mm}

%%%%%%%%%%%%%%%%%%%%%%%%%%%%%%%%%%%%%%%%%%%%%%%%%%%%%%%%%%%%%%%%%%%%%%%%%%%%%%%%%%%%%%%%
%Verzeichnisse -- machen wir mit fancy headern
\renewcommand*{\chapterpagestyle}{fancy}
\cleardoublepage{}
\tableofcontents{}
\cleardoublepage{}
\listoftables
\cleardoublepage{}
\listoffigures

\cleardoublepage{}

%hier geht es los mit dem Text - auf einer rechten Seite
\pagenumbering{arabic}
\pagestyle{fancy}
\thispagestyle{fancy} 



\chapter{Decentralized Communication Layer}
\renewcommand{\kapitelautor}{Autor: Martin Exner}

\section{Einleitung}
% TODO niemand hat besondere berechtigungen, niemand kann die funktion des netzwerks beeinträchtigen
Um die Hauptanforderung an die Umsetzung, den Verzicht auf zentrale Server im System, realisieren zu können,
wird ein \gls{p2pnet} benötigt. Über dieses läuft die Kommunikation der Synchronisationsanwendung.
Dabei muss das Netzwerk vollständig dezentral aufgebaut sein, um ohne Server funktionieren zu können.

Dieses \gls{p2pnet} ist in einer separaten \gls{cl}, \gls{dcl}, umgesetzt.
Das ermöglicht es einerseits, \gls{dcl} für Anwendungen von fremden Entwicklern zu öffnen, sodass diese auf das schon bestehende Netzwerk
zurückgreifen können und nicht erst ein eigenes umsetzen müssen, und andererseits erleichtert die klare Abgrenzung der Funktionen
zwischen \gls{cl} und eigentlicher Synchronisationsanwendung die Umsetzung beider erheblich.

\section{Anforderungen an die \gls{cl}}

Das \gls{p2pnet} des \gls{dcl} muss eine Reihe von Eigenschaften aufweisen,
um für die Anwendung eingesetzt werden zu können. Diese Eigenschaften decken sich im wesentlichen mit
üblichen Anforderungen an herkömmliche, nicht dezentrale Netzwerke und lauten wie folgt:
\begin{description}
	\item [{Adressierung:}]
		
		Den Teilnehmern müssen eindeutige Adressen zugewiesen werden können, die auf ihre Echtheit überprüfbar sind.
	
	
	\item [{Routing:}]
		
		Zwischen Teilnehmern müssen anhand ihrer Adressen Kommunikationskanäle aufgebaut werden können.

\end{description}

Durch den dezentralen Ansatz von \gls{dcl} gestaltet sich die Realisierung dieser Eigenschaften jedoch
anders als das beispielsweise für ein übliches Computernetzwerk oder das Internet der Fall wäre.
So können Adressen nicht von zentralen, dazu bemächtigten Behörden vergeben werden und das Routing
kann nicht von speziellen Teilnehmern des Netzwerks in einer hierarchischen Organisation erfolgen.



\section{Adressierung}

\subsection{Notwendigkeit}
Um in einer auf dem \gls{dcl} basierenden Anwendung Synchronisationgruppen bilden zu können, ist es
nötig, die Teilnehmer dieser Gruppen als solche erkennen zu können.
Das erfordert wiederum die permanente und eindeutige Adressierung dieser Teilnehmer.
Da sich die öffentlichen IP-Adressen der meisten privaten Internetanschlüsse und somit des Großteils der
Zielgruppe von \sblit periodisch ändern, eignen sich diese jedoch nicht als Adressen im \gls{dcl}. Es
wird also ein anderes Adresskonzept benötigt, mit dem ohne höhere Behörde allen Teilnehmern eindeutige
Adressen zugewiesen werden können, die auf ihre Echtheit überprüfbar sind.

Es ist nicht ausreichend, die Teilnehmer ihre Adressen willkürlich selbst bestimmen zu lassen und eine
Funktion zu implementieren, die überprüft, ob eine neu generierte Adresse im Netzwerk schon existiert,
da dieser Ansatz keinerlei Sicherheit vor einer absichtlichen Übernahme der Adresse eines anderen
Teilnehmers durch einen Angreifer bietet.

Dieses Unterkapitel beschreibt die Adressierung im \gls{cnet}.

% TODO schwierigkeit: keine zentrale Stelle, Eindeutigkeit

\subsection{Adressierung mit \gls*{rsa}}
\glslink{aenc}{Asymmetrische Verschlüsselungsverfahren} wie \gls{rsa} eignen sich durch ihre Eigenschaften
ausgezeichnet für die Adressierung innerhalb eines Netzwerks, in dem alle Teilnehmer die gleichen
Berechtigungen haben und in dem keine höhere Instanz existiert, die Adressen vergeben und diese
verifizieren kann.
\tags{key-info, schlüsselpaar, asymmetrisch}

Bei asymmetrischen Verschlüsselungsverfahren kommen sogenannte Schlüsselpaare, bestehend aus zwei
Schlüsseln, zum Einsatz. Die Besonderheit liegt darin, dass eine Nachricht, die mit einem Schlüssel
aus dem Schlüsselpaar verschlüsselt wurde, bei asymmetrischen Verfahren im Gegensatz zu symmetrischen
Verfahren nicht mit dem selben Schlüssel auch wieder entschlüsselt werden kann, sondern ausschließlich
mit dem anderen Schlüssel des Schlüsselpaars.
Gleichzeitig kann aus einem Schlüssel eines Schlüsselpaars der dazugehörige andere Schlüssel des
Schlüsselpaars nicht in absehbarer Zeit berechnet werden.

Dadurch wird es möglich, ein Adressierungssystem umzusetzen, das die Anforderungen im Bezug auf
Überprüfbarkeit der Adressen erfüllt. Dazu wird eine Adresse angenommen, die sich aus einem der
beiden Schlüssel aus dem Schlüsselpaar ableitet. Dieser Schlüssel wird bewusst veröffentlicht,
während der andere Schlüssel aus dem Schlüsselpaar geheim gehalten wird.
Der Schlüssel aus dem Schlüsselpaar, der veröffentlicht wird, wird auch \emph{Öffentlicher Schlüssel}
oder \emph{Public Key} genannt, der weiterhin geheim gehaltene Schlüssel \emph{Privater Schlüssel} oder
\emph{Private Key}.

Öffentliche Schlüssel als Grundlage für Adressen haben den Vorteil, dass die Adressen ohne höhere
Behörde oder zentrale Stelle auf ihre Echtheit überprüft werden können und somit fälschungssicher sind.
Ein Mechanismus zur Überprüfung solch einer Adresse wird im nächsten Abschnitt beschrieben.

\subsection{Überprüfung von Adressen}
Dadurch, dass eine mit einem öffentlichen Schlüssel verschlüsselte Nachricht nicht mit wieder mit dem
öffentlichen Schlüssel entschlüsselt werden kann, sondern ausschließlich mit dem dazugehörigen privaten
Schlüssel, kann der Besitz des gesamten Schlüsselpaars bewiesen werden, ohne mehr als den öffentlichen
Schlüssel preisgeben zu müssen: Eine beliebige Folge von Daten wird vom überprüfenden Teilnehmer
generiert, mit der Grundlage der Adresse des zu überprüfenden Teilnehmers, also seinem öffentlichen
Schlüssel verschlüsselt und anschließend an diesen übermittelt. Dort wird die empfangene Nachricht vom zu
überprüfenden Teilnehmer mit seinem privaten Schlüssel, den nur dieser Teilnehmer besitzt, wieder
entschlüsselt und zurück an den überprüfenden Teilnehmer gesendet.

Decken sich die ursprünglich vom überprüfenden Teilnehmer generierten Daten mit denen, die vom zu
überprüfenden Teilnehmer entschlüsselt wurden, ist der Besitz des gesamten Schlüsselpaars, und nicht
lediglich des öffentlichen Schlüssels, bewiesen. In anderen Worten, die vom überprüften Teilnehmer
bekanntgegebene Adresse ist echt.

%TODO konkretes Beispiel

%TODO Erklärung der Challenge mit Nikolas mergen

\subsection{Eindeutigkeit von Adressen}
\label{dcl-addr-uniqueness}
Die Eindeutigkeit der generierten \gls{rsa}-Schlüsselpaare und somit der Adressen kann zwar nicht
garantiert werden, eine Kollision ist jedoch aufgrund der Länge der verwendeten Schlüssel und der
Anzahl der dadurch möglichen Schlüsselpaare dermaßen unwahrscheinlich, dass davon ausgegangen
werden kann, dass eine Kollision praktisch nicht auftritt. \cite{crypto.stackexchange.com/a/2559:rsa-key-collision}

\subsection{Adressformat}

\subsubsection{\gls*{rsa}-Schlüssellänge}
Die für die Adressen im \gls{dcl} verwendeten öffentlichen \gls{rsa}-Schlüssel haben eine Länge von \addrkeybits
Bits, was \addrkeybytes Bytes entspricht.
IPv4 und IPv6-Adressen haben mit jeweils 32 Bits (4 Bytes) bzw. 128 Bits (16 Bytes) vergleichsweise kurze
Adressen. Trotzdem ist eine Adressierung mit 128 Bits mehr als ausreichend und wesentlich längere
Adressen, wie sie beispielsweise bei direkter Verwendung von RSA-Schlüsseln entstehen, bringen
lediglich Nachteile und keinerlei Vorteile mit sich, da der dabei entstehende Overhead bei Nachrichten
eine wesentliche Reduktion der Übertragungseffizienz mit sich bringt.

\subsubsection{Adressverkürzung}
Die im \gls{dcl} verwendeten Adressen decken sich nicht mit den öffentlichen \gls{rsa}-Schlüsseln,
sondern basieren lediglich darauf. Da ohne Weiterverarbeitung der Schlüssel zu lange Adressen entstünden,
wird auf die Daten der öffentlichen Schlüssel ein \gls{hash} angewandt, welcher die Daten in einem ersten
Schritt wesentlich verkürzt. In einem optionalen zweiten Schritt können die bereits verkürzten
Daten weiter verkürzt werden, um unnötig lange Adressen zu eliminieren.

Nachfolgend ist jener Teil des Quellcodes von \gls{dcl} angeführt, welcher im \code{\gls{cnt}}
zur Adressverkürzung angewandt wird.

\javalisting
\begin{minipage}{\linewidth}
\begin{lstlisting}[caption={Adressverkürzung im \gls*{cnt} (Java)},captionpos=b]
Data scaleAddress(Address address) {
	
	Data fullData = address.toData();
	Data hashedData = hash.update(fullData).finish();
	
	if(byteLength == hashedData.length()) return hashedData;
	
	Data scaledData = new Data(byteLength);
	
	for(int i = 0; i < hashedData.length(); i++) {
		int scaledIndex = i % scaledData.length();
		scaledData.setByte(
			scaledIndex,
			(byte)( scaledData.getByte(scaledIndex)
					^ hashedData.getByte(i))
		);
	}
	
	return scaledData;
	
}
\end{lstlisting}
\end{minipage}

\begin{description}
	
	\descriptionitem{\code{address}}
		Das \javaarg \code{address} referenziert ein \code{\gls{Address}}-Objekt, welches den
		als Grundlage für eine Adresse verwendeten öffentlichen Schlüssel enthält.
	
	\descriptionitem{\code{\gls{Data}}}
		\glsdesc{Data}.
	
	\descriptionitem{\code{hash}}
		Die \javainstvar \code{hash} referenziert ein \code{\gls{Hash}}-Objekt, welches den
		im \gls{nt} angegebenen \gls{hash} implementiert.
	
	\descriptionitem{\code{byteLength}}
		Die \javainstvar \code{byteLength} enthält die Ziellänge der Adressen, in bytes.
	
\end{description}

Mit \code{address.toData()} wird der öffentliche Schlüssel der übergebenen Adresse zuerst in Binärdaten
umgewandelt, auf welche dann ein \gls{hash} angewandt wird. Entspricht die Ziellänge von Adressen im
\gls{nt} der \gls{digestlength} des verwendeten \gls{hash}, so wird der \gls{hashval} \code{hashedData}
direkt zurückgegeben.

Anderenfalls wird der \gls{hashval} aus \code{hashedData} auf Binärdaten der Länge \code{byteLength},
gespeichert in \code{scaledData}, abgebildet und diese zurückgegeben.




\chapter{Sblit}
\renewcommand{\kapitelautor}{Autor: Nikola Szucsich}

\section{Kommunikation}\label{kommunikation}
Um zu verhindern, dass Daten mitgelesen werden, verwendet \sblit den sicheren \nameref{Applicationchannel} (siehe Seite \pageref{Applicationchannel}) der Kommunikationsschicht.\\
\sblitg verwendet zur Kommunikation fünf verschiedene Nachrichten:
\begin{itemize}
	\item Authentifizierungsanfragen
	\item Antworten auf Authentifizierungsanfragen
	\item Dateianfragen
	\item Antworten auf Dateianfragen
	\item Die eigentliche Übertragung der Datei
	\item Löschanfragen
\end{itemize}
\subsection{\gls{authreq}}
Angenommen zwei Geräte eines Besitzers wollen Daten austauschen. Wie können sich diese sicher sein, dass es sich auch um das richtige Gerät handelt? Hier kommt die Authentifizierung ins Spiel. \gls{authreq}s dienen zur Sicherstellung der Authentizität des Gerätes (im Folgenden Gerät A), mit dem ein anderes Gerät (im Folgenden Gerät B) eine Verbindung aufbaut. Dazu schickt das Gerät B zufällige Daten an Gerät A mit der Aufforderung, diese zu verschlüsseln. Die Nachricht ist dabei folgendermaßen aufgebaut:
\begin{figure}[H]
\begin{centering}

\begin{bytefield}[bitwidth=3em]{8}
	\\
	\bitheader{0-7} \\
	
	\begin{rightwordgroup}{\isprotomsgtype}
		\wordbox[tlr]{1}{0}
	\end{rightwordgroup} \\
	
	\begin{rightwordgroup}{\isprotomsgdata}
		\wordbox[tblr]{4}{Zufallsdaten, 64 Bytes} 
	\end{rightwordgroup}
	
\end{bytefield}

\par\end{centering}
\protect\caption{\gls{authreq}}
\end{figure}
%TODO Umformulieren
Die Gesamtlänge der Daten, die mit RSA-2048 verschlüsselt werden, darf maximal 128 Byte lang sein. Um Gerät A 64 Byte zur Verfügung zu stellen, beträgt die Länge der von Gerät B gesendeten Daten 64 Byte. Diese 64 Byte werden von Gerät A benötigt, um zu verhindern, dass Gerät B sich gewünschte Werte von Gerät A verschlüsseln lässt.

\subsection{\gls{authres}}
Bevor die Daten wieder zurückgeschickt werden, müssen diese von Gerät B verschlüsselt werden. Dies geschieht mit dem Private-Key des Gerätes B. Dabei wird vorher noch ein zufälliger Wert an die empfangenen Daten angefügt, um das unter Authentifizierungsanfragen beschriebene Problem zu lösen.\\
Der Aufbau der Nachricht lautet wie folgt:
\begin{figure}[H]
\begin{centering}

\begin{bytefield}[bitwidth=3em]{8}
	\\
	\bitheader{0-7} \\
	
	\begin{rightwordgroup}{\isprotomsgtype}
		\wordbox[tlr]{1}{1}
	\end{rightwordgroup} \\
	
	\begin{rightwordgroup}{\isprotomsgdata}
		\wordbox[tblr]{4}{Verschlüsselte Daten, 128 Bytes} 
	\end{rightwordgroup}
	
\end{bytefield}

\par\end{centering}
\protect\caption{\gls{authres}}
\end{figure}
Wird dieses Paket nun von Gerät B empfangen, kann der Inhalt mit dem Public-Key des Gerätes A, also dessen Adresse, entschlüsselt werden. Von den erhaltenen Daten werden die ersten 64 Byte mit den ursprünglich gesendeten 64 Byte verglichen. Stimmen die beiden Werte überein, kann das Gerät seine Authentizität beweisen. Stimmen diese jedoch nicht überein, handelt es sich um einen Betrüger, der offensichtlich den richtigen Private-Key zur von ihm angegebenen Adresse nicht kennt.
		
\subsection{\gls{filereq}} \label{Dateianfrage}
Bevor eine Datei an ein anderes Gerät (im Folgenden Gerät B) verschickt wird, schickt das Gerät, das die Datei besitzt (im Folgenden Gerät A), eine Dateianfrage an das Gerät B. Dies hat 2 Gründe: Erstens muss eruiert werden, ob die Datei überhaupt von Gerät B benötigt wird, oder ob besagtes Gerät schon diese Datei besitzt. Zweitens besteht die Möglichkeit eines Konfliktes \referenz{Konflikt}. Daher ist die Dateianfrage folgendermaßen aufgebaut:
\begin{figure}[H]
\begin{centering}

\begin{bytefield}[bitwidth=3em]{8}
	\\
	\bitheader{0-7} \\
	
	\begin{rightwordgroup}{\isprotomsgtype}
		\wordbox[tlr]{1}{2}
	\end{rightwordgroup} \\
	
	\begin{rightwordgroup}{\isprotomsgdata}
		\wordbox[tlr]{2}{Dateipfad, variable Länge} \\
		\skippedwords \\
		\wordbox[lr]{1}{} \\
		\wordbox[tlr]{2}{Versionsverlauf, variable Länge} \\
		\skippedwords \\
		\wordbox[blr]{1}{}
	\end{rightwordgroup}
	
\end{bytefield}
\par\end{centering}
\protect\caption{\gls{filereq}}
\end{figure}
\begin{description} 
	\item[{Dateipfad}] \hfill \\
		Hierbei wird der zu \sblit's Hauptordner relative Dateipfad mitgeschickt. Der absolute Dateipfad wird aus dem Grund nicht mitgeschickt, da der Ort des \sblit-Ordners nicht auf allen Geräten gleich sein muss. Befindet sich der Ordner auf Gerät A beispielsweise unter C:/Users/Susanne/ kann sich der Ordner auf Gerät B auch unter /home/susanne/dateien/ befinden.
	\item[{Versionsverlauf}] \hfill \\
		Der Versionsverlauf beinhaltet alle Hashes einer Datei seit der letzten komplett synchronisierten Version. Das heißt, dass auf jedem Gerät aktuell entweder diese oder eine neuere Version gespeichert ist.
\end{description}

		
\subsection{\gls{fileres}}
Die Antwort auf eine Dateianfrage beinhaltet folgende Parameter: 

\begin{figure}[H]
\begin{centering}

\begin{bytefield}[bitwidth=3em]{8}
	\\
	\bitheader{0-7} \\
	
	\begin{rightwordgroup}{\isprotomsgtype}
		\wordbox[tlr]{1}{3}
	\end{rightwordgroup} \\
	
	\begin{rightwordgroup}{\isprotomsgdata}
		\wordbox[tlr]{1}{Need-Flag, 1 Byte} \\
		\wordbox[tlr]{2}{Dateipfad, variable Länge} \\
		\skippedwords \\
		\wordbox[lr]{1}{} \\
		\wordbox[tlr]{2}{Hash, variable Länge} \\
		\skippedwords \\
		\wordbox[blr]{1}{}
	\end{rightwordgroup}
	
\end{bytefield}
\par\end{centering}
\protect\caption{\gls{fileres}}
\end{figure}

\begin{description}
	\item[{Need-Flag}] \hfill \\
		Dieses Feld enthält einen Hexadezimalwert, der darüber Auskunft gibt, ob die Datei benötigt wird oder nicht. Steht in diesem Feld der Hexadezimalwert 0x00, wird die Datei nicht benötigt. Steht hier hingegen der Hexadezimalwert 0x01, wird die Datei benötigt. Dieses Byte hilft den Datenverkehr zu reduzieren. So muss nicht eine ganze Datei verschickt werden muss, obwohl diese gar nicht gebraucht wird.
	\item[{Dateipfad}] \hfill \\
		Wie auch bei der Dateianfrage wird in der Antwort auf die Dateianfrage der zu \sblit's Hauptordner relative Pfad mitgeschickt.
	\item[{Hash}] \hfill \\
		Hier wird noch einmal der letzte Hash des Versionsverlaufs der Dateianfrage verschickt, um sicherzustellen, dass die Datei in der Zwischenzeit nicht geändert wurde. 
\end{description}
Nach Empfang der Dateianfrage wird zunächst geprüft, ob die Datei vorhanden und in der aktuellst ist.  Ist die lokale Datei nicht aktuell, wird das Need-Flag, auf den Hexadezimalwert 0x01 gesetzt. Ist die Datei auf dem aktuellsten Stand, wird besagtes Flag auf den Hexadezimalwert 0x00 gesetzt. Außerdem wird überprüft, ob ein Konflikt aufgetreten ist \referenz{Konflikterkennung}.\\
Nach Empfang der Antwort wird zunächst überprüft, ob der Hashwert der Datei mit dem erhaltenen Pfad übereinstimmt. Stimmt der Hashwert in der Anfrage nicht mit dem aktuellen Hashwert überein, wird die Antwort verworfen. Dies kann beispielsweise passieren, wenn in der Zwischenzeit eine neuere Version der Datei erzeugt wurde. Stimmt dieser jedoch überein, kann die Datei nun im nächsten Schritt verschickt werden.
		
\subsection{Die eigentliche Übertragung der Datei}
Bei der eigentlichen Übertragung der Datei werden folgenden Daten mit der Datei mitgesendet: 
\begin{figure}[H]
\begin{centering}

\begin{bytefield}[bitwidth=3em]{8}
	\\
	\bitheader{0-7} \\
	
	\begin{rightwordgroup}{\isprotomsgtype}
		\wordbox[tlr]{1}{5}
	\end{rightwordgroup} \\
	
	\begin{rightwordgroup}{\isprotomsgdata}
		\wordbox[tlr]{2}{Dateiinhalt, variable Länge} \\
		\skippedwords \\
		\wordbox[lr]{1}{} \\
		\wordbox[tlr]{2}{Dateipfad, variable Länge} \\
		\skippedwords \\
		\wordbox[lr]{1}{} \\
		\wordbox[tlr]{2}{Hashes, variable Länge} \\
		\skippedwords \\
		\wordbox[lr]{1}{} \\
		\wordbox[tlr]{2}{Geräte mit der aktuellen Verision, variable Länge} \\
		\skippedwords \\
		\wordbox[blr]{1}{}
	\end{rightwordgroup}
	
\end{bytefield}
\par\end{centering}
\protect\caption{\gls{filemsg}}
\end{figure}

\begin{description}
	\item[{Dateiinhalt}] \hfill \\
		Der Dateiinhalt wird als binäres \code{Data}-Objekt verschickt.
	\item[{Geräte mit der aktuellen Version}] \hfill \\
		Hier stehen die Adressen aller Geräte, die die neuste Version schon haben. Ist die Version auf allen Geräten aktuell, kann diese von den Partnergeräten gelöscht werden. Daher wird diese Liste an Geräten immer mit der Datei mitgeschickt. Außerdem können somit unnötige Anfragen an Geräte, die die Datei schon besitzen verhindert werden.
	\item[{Dateipfad}] \hfill \\
		Der Dateipfad wird benötigt, damit die Datei auf dem zu synchronisierenden Gerät weiß, an welchen Ort die Datei gespeichert werden soll. Dies ist der gleiche relative Ort, wie auf dem Gerät, das die Anfrage geschickt hat.
	\item[{Hashes}] \hfill \\
		Die Hashes werden mitgeschickt, um einen einheitlichen Versionsverlauf sicherzustellen. Das verhindert das auftreten von Konflikten, welche gar keine sind. Weiters dient der aktuellste Hash dazu,  sicherzustellen, dass die Datei, die versendet wurde, auch so ankommt, wie sie versendet wurde. Verhasht man die neue Datei, muss das Ergebnis mit dem aktuellsten mitgeschickten Hash übereinstimmen. Andernfalls muss die Datei neu gesendet werden.
\end{description}
		
\subsection{Löschanfrage}
\begin{figure}[H]
\begin{centering}

\begin{bytefield}[bitwidth=3em]{8}
	\\
	\bitheader{0-7} \\
	
	\begin{rightwordgroup}{\isprotomsgtype}
		\wordbox[tlr]{1}{5}
	\end{rightwordgroup} \\
	
	\begin{rightwordgroup}{\isprotomsgdata}
		\wordbox[tlr]{2}{Dateipfad, variable Länge} \\
		\skippedwords \\
		\wordbox[blr]{1}{} \\
	\end{rightwordgroup}
	
\end{bytefield}
\par\end{centering}
\protect\caption{\gls{filemsg}}
\end{figure}		
Eine Löschanfrage beinhaltet den Pfad der zu löschenden Datei. Nach Empfang der Löschanfrage wird die Datei gelöscht.



\section{Datei-Verarbeitung}\label{dateiverarbeitung}
\subsection{Allgemein}
Ein wesentlicher Bestandteil von \sblit ist die Verarbeitung der Dateien. Dazu zählen das Erkennen von Änderungen sowie das Erkennen und Lösen von Konflikten.
\subsection{Versionierung}\label{Logfile}
Ein wichtiges Mittel zur Verwaltung der Dateien ist das \gls{logfile}. Dieses enthält Informationen zu allen Dateien im \sblit-Ordner. Dazu zählen der relative Dateipfad, eine Liste mit Hashes, die für die Versionierung zuständig sind, und eine Liste an Geräten, auf denen die Datei bereits auf dem aktuellen Stand ist. Weiters wird in dieser Datei festgehalten, ob Dateien im Konflikt zu anderen Dateien stehen. Die Versionierung ist vor allem für die Konflikterkennung \referenz{Konflikterkennung} notwendig. Hierbei werden Hashes für alle Versionen der Dateien gespeichert. Die Version wird beim Speichern der Datei um den aktuellen Hashwert erweitert und bei Konvergenz auf allen Geräten auf die aktuelle gemeinsame Version reduziert. 

Damit \sblit weiß, wann die Dateien von den Partnergeräten \referenz{Partnergerät} wieder gelöscht werden können, wird eine Liste von Geräten, mit Dateien auf dem aktuellen Stand, gespeichert. Diese Liste wird bei jeder Änderung mit der Liste aller eigenen Geräte verglichen. Sobald alle eigenen Geräte eingetragen sind, werden die Partnergeräte aufgefordert, die Datei zu löschen. 

\subsection{Reaktionen auf Dateiänderungen}
Wenn eine Datei neu erstellt, bearbeitet oder gelöscht wird, erkennt dies \sblit und leitet diese Informationen an den Synchronisationsprozess weiter. Außerdem ist das Erkennen einer Änderung im \sblit-Ordner wichtig, damit sie im \gls{logfile} protokolliert werden kann. Das \gls{logfile} wird sowohl für die Konflikterkennung \referenz{Konflikterkennung} als auch für die Reduktion der benötigten Bandbreite \referenz{Dateianfrage} genutzt.

Um Änderungen zu erkennen, verwendet \sblit das sogenannte \gls{watchservice}. Das \gls{watchservice} wird bei Änderungen in dem Ordner, der vom Benutzer für die Synchronisation festgelegt wurde, benachrichtigt. Je nach dem, ob eine Datei angelegt, verändert oder gelöscht wurde, erfolgen unterschiedliche Arbeitsschritte. Beim Anlegen einer neuen Datei, wird ihr Pfad samt Hash in das \gls{logfile} geschrieben und eine Dateianfrage an die anderen eigenen Geräte geschickt. Wird eine Datei verändert, wird ein neuer Hash zu den bereits vorhandenen Hashes hinzugefügt. Außerdem wird eine Dateianfrage an die anderen eigenen Geräte gesendet. Wird eine Datei gelöscht, wird sie samt Hashes aus dem \gls{logfile} entfernt. Anschließend wird eine Löschanfrage an die eigenen Geräte versandt.\\ \\
\listingstart{Initialisierung des \gls{watchservice}}
WatchService watcher = filesDirectory.getFileSystem()
		.newWatchService();
filesDirectory.register(watcher,
		StandardWatchEventKinds.ENTRY_CREATE,
		StandardWatchEventKinds.ENTRY_DELETE,
		StandardWatchEventKinds.ENTRY_MODIFY);
WatchKey watchKey = watcher.take();
List<WatchEvent<?>> events = watchKey.pollEvents();
Thread.sleep(TIME_TO_SLEEP);
Map<String, LinkedList<Data>> logs = getLogs();
Map<String, LinkedList<Data>> synchronizedDevices = 
		getSynchronizedDevices();
\end{lstlisting}
\begin{description}
	\descriptionitem{filesDirectory}
	\code{filesDirectory} enthält den vom Benutzer festgelegten \sblit-Ordner.
	\descriptionitem{watcher}
	\code{watcher} wird benötigt, um das \gls{watchservice} einen bestimmten Ordner überwachen zu lassen.
	\descriptionitem{watchKey}
	\gls{watchkey} enthält eine Liste an Dateien, die neu erstellt, geändert oder gelöscht wurden.
	\descriptionitem{events}
	Jedes \code{WatchEvent} aus der \code{List events} enthält eine Datei, die sich geändert hat und die Information, ob sie neu erstellt, verändert oder gelöscht wurde. 
	\descriptionitem{logs}
	\code{logs} enthält den Versionsverlauf aller Dateien, die sich im \sblit-Ordner befinden.
	\descriptionitem{synchronizedDevices}
	\code{synchronizedDevices} enthält zu jeder Datei eine Liste an Geräten, auf denen die akuelle Version gespeichert ist.
\end{description}
Zunächst wird ein neues Objekt vom Typ \code{\gls{watchservice}} erstellt. Dieses überwacht den festgelegten \sblit-Ordner hinsichtlich neu erstellter, geänderter und gelöschter Dateien. Zur Registrierung des \gls{watchservice} im Dateisystem, werden in der \code{register}-Methode das \gls{watchservice}-Objekt und die Fälle, in denen das \gls{watchservice} benachrichtigt werden soll, festgelegt. In diesem Fall ist sind das neue Dateien (\code{ENTRY\_CREATE}), Dateiänderungen (\code{ENTRY\_MODIFY}) und gelöschte Dateien (\code{ENTRY\_DELETE}). Die \code{take}-Methode wartet auf Änderungen im festgelegten Ordner und übergibt sie dann an das \code{\gls{watchkey}}-Objekt. Die Anweisung \code{watchKey.pollEvents()} unterteilt den \gls{watchkey} in einzelne Dateiänderungen und formatiert diese für den weiteren Gebrauch. 

Der Thread wird angehalten, da das Programm beim Empfangen einer Datei das Logfile bearbeitet. Dies passiert, damit \sblit weiß, dass die Datei nicht vom Nutzer verändert wurde. \\ \\
\listingstart{Unterteilen in einzelne Dateien}
for (WatchEvent event : events) {
	String path = event.context().toString();
	File changedFile = new File(
			Configuration.getSblitDirectory()
					+ Configuration.slash + path);
\end{lstlisting}
\begin{description}
	\descriptionitem{event}
	\code{event} enthält Informationen zur Erstellung, Änderung oder Bearbeitung einer Datei aus der Liste \code{events}. 
	\descriptionitem{path}
	\code{path} enthält den relativen Pfad zur geänderten Datei.
	\descriptionitem{changedFile}
	\code{changedFile} enthält die Datei, die gerade geändert wurde.
\end{description}
\listingstart{Erstellen einer Datei}
if (event.kind() == StandardWatchEventKinds.ENTRY_CREATE 
		&& logs.get(path) == null) {
	byte[] fileContent = 
			Files.readAllBytes(changedFile.toPath());
	Data hash = Crypto.sha1(new Data(fileContent));
	LinkedList<Data> hashes = new LinkedList<>();
	hashes.add(hash);
	logs.put(path, hashes);
	LinkedList<Data> devices = new LinkedList<>();
	devices.add(Configuration
			.getPublicAddressKey().toData());
	synchronizedDevices.put(path, devices);
	filesToPush = refreshFilesArray(filesToPush, changedFile);
}
\end{lstlisting}
\begin{description}
	\descriptionitem{fileContent}
	\code{fileContent} enthält den Inhalt der Datei, die neu erstellt wurde.
	\descriptionitem{hash}
	In \code{hash} wird der Hashwert des Inhalts der Datei gespeichert.
	\descriptionitem{hashes}
	\code{hashes} enthält den Versionsverlauf einer Datei. 
	\descriptionitem{devices}
	\code{devices} enthält alle Geräte, auf denen die Datei aktuell ist. In diesem Fall ist das genau das Gerät, auf dem der Code ausgeführt wird, da diese Datei gerade erst erstellt wurde.
\end{description}
Wenn eine Datei erstellt wurde, wird sie zu allererst ausgelesen. Anschließend wird der Inhalt verhasht und dem Versionverlauf hinzugefügt. Außerdem wird das Gerät, auf dem der Code ausgeführt wird, zur Geräteliste \code{devices} hinzugefügt. Schließlich wird die Datei zur Liste der zu synchronisierenden Dateien hinzugefügt.\\ \\
\listingstart{Löschen einer Datei}
else if (event.kind() == StandardWatchEventKinds
		.ENTRY_DELETE) {
	logs.remove(path);
	synchronizedDevices.remove(path);
	filesToDelete = refreshFilesArray(filesToDelete,
			changedFile);
}
\end{lstlisting}
Beim Löschen einer Datei wird der zugehörige Versionsverlauf ebenfalls entfernt. Außerdem wird der Variable \code{filesToDelete} der Pfad der gelöschten Datei hinzugefügt, um die anderen eigenen Geräte auch darüber zu informieren, dass eine Datei gelöscht wurde. \\ \\
\listingstart{Bearbeiten einer Datei}
else if (event.kind() == StandardWatchEventKinds
		.ENTRY_MODIFY) {
	byte[] fileContent = readFile(event);
	Data hash = Crypto.sha1(new Data(fileContent));
	LinkedList<Data> hashes = logs.get(path);
	if (!hashes.contains(hash)) {
		hashes.add(hash);
		filesToPush = refreshFilesArray(filesToPush,
				changedFile);
		LinkedList<Data> devices = new LinkedList<>();
		devices.add(Configuration
				.getPublicAddressKey().toData());
		synchronizedDevices.put(path, devices);
	}
}
\end{lstlisting}
\begin{description}
\descriptionitem{fileContent}
	\code{fileContent} enthält den Inhalt der Datei, die bearbeitet wurde.
	\descriptionitem{hash}
	In \code{hash} wird der Hashwert des Inhalts der Datei gespeichert.
	\descriptionitem{hashes}
	\code{hashes} enthält den Versionsverlauf der Datei. 
	\descriptionitem{devices}
	\code{devices} enthält alle Geräte, auf denen die Datei aktuell ist. In diesem Fall ist das genau jenes Gerät, auf dem der Code ausgeführt wird, da diese Datei gerade bearbeitet wurde.
\end{description}
Wurde die Datei bearbeitet, wird sie ausgelesen und anschließend verhasht. Nachdem der Versionsverlauf für die Datei in der Variable \code{hashes} gespeichert wurde, wird überprüft, ob der Versionsverlauf die Datei schon enthält. Ist dies der Fall, ist ein Fehler aufgetreten und die neue Version wird nicht synchronisiert. Falls der Versionsverlauf die aktuelle Version noch nicht enthält, wird der Hash der aktuellen Version im Versionsverlauf gespeichert. Anschließend wird eine neue Liste an Geräten erstellt, auf denen schon die neuste Version synchronisiert ist. In diesem Fall wird nur das eigene Gerät hinzugefügt, da die neuste Version bis jetzt nur auf diesem Gerät vorhanden ist. \\ \\
\listingstart{Schreiben des \gls{logfile}s}
}
logFile.createNewFile();
write(files, synchronizedDevices);
\end{lstlisting}
Das \gls{logfile} wird neu erstellt und die neuen Daten werden anschließend darin gespeichert.
\subsection{Konflikte}\label{Konflikt}
\subsubsection{Allgemein}
Ein Konflikt ist ein Problem, das bei Synchronisationsdiensten vorkommt. Dieser tritt auf, wenn eine Datei bearbeitet wird, bevor sie synchronisiert werden kann. 

Dazu ein Beispiel: Susanne hat einen Laptop und einen Stand-Rechner, auf denen sie mithilfe von \sblit einen Ordner synchronisiert. Im Normalfall, also wenn kein Konflikt auftritt, bearbeitet Susanne eine Datei auf dem Laptop. Nach dem Speichern wird die Datei auf den Stand-Rechner übertragen und dort gespeichert. Die Datei ist nun auf beiden Geräten synchron.

Angenommen, Susanne schaltet nun den Laptop aus und bearbeitet anschließend die Datei noch einmal auf dem Stand-Rechner. Da der Laptop ausgeschaltet ist, kann die Datei nicht synchronisiert werden. Auf dem Weg zur Arbeit fällt Susanne noch eine Verbesserungsmöglichkeit der Datei ein und sie bearbeitet die Datei auf dem Laptop ohne einer Verbindung zum Internet. In der Arbeit angekommen, packt Susanne wieder ihren Laptop aus und verbindet sich zum Internet. Die Datei kann jetzt nicht auf den neusten Stand gebracht werden, da ja zwei unterschiedliche Versionen existieren. Würde die Datei einfach vom Stand-Rechner auf den Laptop kopiert werden, gingen die Neuerungen am Laptop verloren, umgekehrt würde die Datei vom Laptop die Änderungen am Stand-Rechner überschreiben. Diesen Zustand zweier verschiedener Versionen der gleichen Datei, ohne die Änderungen des jeweils anderen Gerätes schon berücksichtigt zu haben, nennt man einen Konflikt.
%TODO eventuell noch ergänzen

\subsubsection{Konflikterkennung}\label{Konflikterkennung}
Um Konflikte zu erkennen, verwendet \sblit eine interne Versionierung der Dateien. Diese wird im Kapitel \linkt{Logfile} näher erklärt. Bei einer Dateianfrage \referenz{Dateianfrage} werden alle Hashes einer Datei mitgesendet. Diese werden nach dem Empfang der Anfrage mit den Hashes der Datei mit dem gleichen Namen und aller Konfliktdateien dieser Datei im lokalen \gls{logfile} verglichen. Stimmen die beiden aktullen Hashes aus Dateianfrage und der Datei mit dem gleichen Namen im \gls{logfile} überein, muss die Datei gar nicht übertragen werden, da sie schon auf dem neusten Stand ist. Das heißt natürlich auch, dass kein Konflikt auftritt. Stimmt der aktuelle Hash im lokalen \gls{logfile} mit einem in der Dateianfrage überein, tritt ebenfalls kein Konflikt auf, da der aktuelle lokale Hash dem Gerät, das die Anfrage verschickt hat, schon bekannt ist. 

Stimmt der Versionsverlauf aus dem \gls{filereq} mit dem einer Konfliktdatei überein, wurde eine Version synchronisiert, bevor ein drittes Gerät einen Konflikt bemerkt hat. In diesem Falle könnten mehrere Konfliktdateien mit nur zwei verschiedenen Inhalten erstellt werden. Passiert dies, werden die Namen der beiden Dateien vertauscht, um das eben genannte Szenario zu verhindern.

Ist der aktuelle Hash der lokalen Datei jedoch nicht in der Dateianfrage enthalten, wurde die lokale Datei bearbeitet, bevor diese auf den aktuellsten Stand gebracht werden konnte. Anders gesagt: Ein Konflikt ist aufgetreten.

\subsubsection{Konfliktlösung}
Damit die Änderungen von einem Gerät nicht überschrieben werden, speichert \sblit beide Versionen. Da jedoch beide Dateien nicht den gleichen Namen haben können, benennt das Gerät, das den Konflikt erkennt, (im Folgenden Gerät A) die lokale Datei um. Anschließend schickt Gerät A eine Antwort an das Gerät, das die Dateianfrage geschickt hat,  (im Folgenden Gerät B) in der es die neue Version anfordert, als ob kein Konflikt aufgetreten wäre. Außerdem schickt Gerät A eine Dateianfrage mit der neuen Datei an Gerät B. Gerät B sendet nun die angeforderte Datei an Gerät A und akzeptiert die Konfliktdatei, da es eine Datei mit dem gleichen Namen noch nicht besitzt. 

Nach dem Empfang der Datei speichtert Gerät A diese unter dem ursprünglichen Namen. Des Weiteren sendet Gerät A die Konfliktdatei mit dem geänderten Namen an Gerät B. Die neue Datei wird auf Gerät B unter dem neuen Namen gespeichert.

\subsubsection{Umsetzung}
Um Konflikte zu erkennen und zu lösen wird der folgende Code verwendet: \\ \\
\listingstart{Erkennen eines Konflikts}
private void handleConflict(LinkedList<Data> requestedHashes, LinkedList<Data> ownHashes, String path) throws IOException {
	String sblitDirectory = Configuration.getSblitDirectory() + Configuration.slash;
	if (!requestedHashes.contains(ownHashes.get(ownHashes.size() - 1))) {
		int dotIndex = path.lastIndexOf(".");
		File conflictFile;
		if (dotIndex > path.lastIndexOf(Configuration.slash) + 1) {
			for (int i = 1;; i++) {
				conflictFile = new File(sblitDirectory + path.substring(0, dotIndex) + "(Conflict " + i + ")" + path.substring(dotIndex));
				if (!conflictFile.exists()) {
					break;
				}
			}
		} else {
			for (int i = 1;; i++) {
				conflictFile = new File(sblitDirectory + path
						+ "(Conflict " + i + ")");
				if (!conflictFile.exists()) {
					break;
				}
			}
		}
		File file = new File(sblitDirectory + path);
		Files.copy(file.toPath(), conflictFile.toPath());
	}
}
\end{lstlisting}

\begin{description}
	\descriptionitem{requestedHashes} Der Parameter gibt an, welche Hashes in der Dateianfrage geschickt wurden. Die Hashes haben den Datentyp Data, welcher in \gls{dcl} definiert wurden. Um eine unbestimmte Menge davon speichern zu können, werden diese in eine LinkedList geschrieben. Diese hat den Vorteil, dass die Elemente die Reihenfolge behalten, in der sie in die Liste geschrieben wurden.
	
	\descriptionitem{ownHashes} Dieser Parameter enthält eine \code{LinkedList} an Hashes vom Datentyp Data, welche gemeinsam den Versionsverlauf der lokalen Datei ergeben. 
	
	\descriptionitem{path} Der Parameter path enthält den zum \sblit-Ordner relativen Pfad.
	
	\descriptionitem{sblitDirectory} Diese Variable vom Datentyp String enthält den absoluten Pfad des \sblit-Ordners. Dieser setzt sich zusammen aus \code{Configuration.getSblitDirectory()} und \code{Configuration.slash}. \code{Configuration.getSblitDirectory()} liefert den \sblit-Ordner zurück, der in der \code{Configuration}-Klasse definiert ist. Außerdem wird, je nach dem, ob das Betriebssystem Windows oder Unix-basierend ist, ein Backslash oder ein Slash an den Pfad gehängt. Die Zuweisung des richtigen Slashes wird am Programmstart bei der Initialisierung der \code{Configuration}-Klasse vorgenommen.

	\descriptionitem{dotIndex} In dieser Variable befindet sich der Index des letzten Punktes im Dateipfad. Dieser dient dazu, um herauszufinden, ob die Datei eine Endung hat.
	
	\descriptionitem{conflictFile} Das conflictFile ist die neue Datei, die erzeugt wird, wenn ein Konflikt auftritt.
	
	\descriptionitem{file} Die Variable file enthält die ursprüngliche Datei.
\end{description}
Enthält der empfangene Versionsverlauf den aktuellsten Hash im \gls{logfile}, 
Mithilfe der \code{if}-Anweisung in Zeile sechs wird überprüft, ob der empfangene Versionsverlauf den letzten Hash, der im lokalen \gls{logfile} gespeichert ist, nicht enthält. Somit wird überprüft ob ein Konflikt aufgetreten ist. 

In der \code{if}-Anweisung in Zeile 10 wird herausgefunden, ob die Datei eine Dateiendung besitzt. Dafür wird überprüft, ob der letzte Slash mehr als einen Platz vor dem Punkt ist, damit versteckte Dateien und Ordner unter Linux ausgeschlossen werden können. Diese werden nämlich mit einem Punkt vor dem Datei- oder Ordnernamen gekennzeichnet.

Die \code{for}-Schleife in Zeile 15 stellt zusammen mit der \code{if}-Anweisung in Zeile 21 sicher, dass die Konfliktdatei nicht schon vorhanden ist. Existiert diese noch nicht, wird die Schleife abgebrochen und der Name der Konfliktdatei bleibt erhalten.



\chapter{Beschreibung der Formatierung}

% wer hat diese Kapitel geschrieben oder leer
\renewcommand{\kapitelautor}{Autor: Hans Huber}

\thispagestyle{fancy}


\section{Vorlagen}

In diesen Kapitel gibt es einige Muster für Dinge die oft vorkommen.
Und etwas Blindtext damit man auch volle Seiten sieht.


\subsection{Formatvorlagen}

Alle Formatierungen sollten mit Formatvorlagen vorgenommen werden.
Spätestens bei der Konvertierung in ein Ebook rächen sich diese ,,Sünden``:
Ebooks sind HTML Dokumente mit einer Formatierung mittels CSS.

Auch bei der Umwandlung in interaktive PDFs ist eine konsequente Formatierung
wichtig.


\subsection{Schriften und Absätze}

Hier findet man eine Beschreibung des Layouts -- Details folgen weiter
unten.
\begin{description}
\item [{Schrift:}] dieses \LaTeX{}-Dokument verwendet die Standardschriften.
Die Schriftgröße soll 12\,pt betragen.
\item [{Absatz:}] entweder verwendet man wie in \LaTeX{} einen etwas größeren
Seitenrand oder einen größeren Zeilenabstand. Beides sorgt für bessere
Lesbarkeit. Die erste Zeile eines Absatzes wird etwas eingerückt (nicht
die erste Zeile nach einer Überschrift, nach einem Bild etc.) und
bzw. oder es gibt einen Abstand zwischen den Absätzen. Am Ende und
Anfang einer Seite sollten mindestens zwei Zeilen eines Absatzes sein
(keine Schusterjungen%
\footnote{siehe \url{http://www.typolexikon.de/s/schusterjunge.html}%
} und Hurenkinder%
\footnote{siehe \url{http://www.typolexikon.de/h/hurenkind.html}%
}).
\item [{Blocksatz:}] Alle Texte werden im Blocksatz gesetzt. Die Silbentrennung
ist dann obligatorisch.
\item [{Kapitelüberschriften:}] Überschriften erster Ordnung sollten auf
rechten Seiten beginnen. Über jeder Überschrift sollte ein Abstand
sein. Alle Überschriften müssen mit de nächsten Absatz ,,zusammengehalten``
werden -- keine einsamen Überschriften am Ende einer Seite.
\item [{Inhaltsverzeichnis:}] das Inhaltsverzeichnis sollte möglichst kompakt
sein. Als Gliederung dienen fette Hauptüberschriften und etwas Abstand
über den Zeilen.
\item [{Seitenformat:}] der Ausdruck erfolgt zweiseitig, ein entsprechender
Bundsteg ist zu berücksichtigen%
\footnote{Die Einstellung der Seitenränder ist keinesfalls beliebig. Sie sollte
bewährten Regeln folgen, {[}\ldots{}{]}. Die häufige Zielvorgabe
,,Den Platz auf dem Papier möglichst gut ausnutzen`` ist keine typografische
sondern eine extrem laienhafte Regel. aus \cite{layout}%
}. Nach Rücksprache mit dem Betreuer kann auch eine einseitige Variante
gewählt werden. Bei Bedarf könne auch einzelne Seiten im Querformat
gesetzt werden.
\item [{Kopfzeile:}] die Kopfzeile sollte dieser Vorlage entsprechen. Falls,
nach Rücksprache mit dem Betreuer, der Ausdruck nur in Schwarz-weiß
erfolgt, kann das Logo entfallen.
\item [{Fußzeile:}] hier ist Platz für den Autor des Kapitels und die Seitennummer.
Wie bei technischen Publikationen üblich ist die Einleitung und die
Verzeichnisse mit römischen Seitennummern versehen. Das eigentliche
Dokument wird mit arabischen Ziffern nummeriert. Beide Nummerierungen
sind unabhängig voneinander und beginnen jeweils bei 1.
\item [{Autor:}] Jedes Kapitel muss auch einem Autor haben. Das sieht man
in der Fußzeile oder als Textbox in der Nähe der Überschrift. Alternativ
kann es im Anhang eine Liste geben. Das ist besonders wichtig wenn
es viele Beilagen, z.B. Handbücher ohne direkte Angabe des Autors,
gibt.
\item [{PDF:}] Die PDF Metainformation sollten richtig sein (Autor etc.)
-- siehe Datei/Eigenschaften. Links auf Webseiten, Verweise innerhalb
des Dokuments, das Inhaltsverzeichnis, die Fußnoten usw. sollten ,,klickbar``
sein.
\end{description}

\subsection{Bilder\label{sub:Bilder}}

Das Bild als Gleitobjekt ist genau hier, oder oben auf der Seite,
oder unten, aber immer zentriert mit Nummer und Beschreibung -- wenn
es sinnvoll ist auch mit Querverweis (siehe Abbildung \ref{Bild11}).
Durch Gleitobjekte, d.~h. Bilder oben oder unten auf der Seite statt
,,genau hier``, werden halbleere Seiten durch besonders große Bilder
vermieden.

Wichtig: alle Bilder oder andere Medien z.~B. Screenshots, Audio
oder Video für EBooks und interaktive PDFs sollten mit einen entsprechenden
Quellennachweis versehen sein.

\begin{figure}[tbh]
\begin{centering}
\includegraphics[scale=0.6]{HTL3RLogoRGB}
\par\end{centering}

\protect\caption{Ein Bild}
\label{Bild11}
\end{figure}



\subsection{Tabellen}

In der folgenden Tabelle sieht man: es gibt immer eine Nummer und
eine Beschreibung. Besonders längere Tabellen sollten eventuell als
Gleitobjekt am Ende oder Anfang einer Seite positioniert werden. Geht
die Tabelle über mehrere Seiten so ist die Überschrift zu wiederholen.

\begin{table}[h]
\begin{centering}
\begin{tabular}{|c|c|c|}
\hline 
Überschrift & Wert & noch einer\tabularnewline
\hline 
\hline 
1 & abc & Hallo\tabularnewline
\hline 
2 & def & Latex\tabularnewline
\hline 
\end{tabular}
\par\end{centering}

\protect\caption{So eine tolle Tabelle}
\end{table}



\subsection{Formel}

Etwas Text als Überleitung zu einer Formel:

\selectlanguage{ngerman}%
\[
f(x)=\left\{ \begin{array}{cc}
\log_{8}x & x>0\\
0 & x=0\\
\sum_{i=1}^{5}\alpha_{i}+\sqrt{-\frac{1}{x}} & x<0
\end{array}\right.
\]


\selectlanguage{naustrian}%
Wenn man sehr viele Formeln hat sollte man diese auch nummerieren.
Besonders bei Verweisen ist das sehr sinnvoll.


\subsection{Sourcecode}

Sourcecode sollte in einer Schrift mit fixer Breite sein. Falls man
Verweise braucht sollte man die Listings auch nummerieren. 

% das kann auch ganz oben stehen
% das braucht man nur einmal
\lstset{numbers=left, numberstyle=\tiny, stepnumber=2, numbersep=5pt, showspaces=true, frame=single}
% einmal oder immer was anderes
\lstset{language=C}

% hier könnte man auch aus Dateien lesen
\begin{lstlisting} 
#include <stdio.h>

int main() 
{ 
  printf("Hello world\n"); 
} 
\end{lstlisting} 

Die genaue Formatierung ist freigestellt: Einstellungen wie bunt bzw.
fett, Markierung von Leerzeichen und Zeilennummerierung kann an den
Bedarf der Diplomarbeit angepasst werden. 

Beispiel Java mit anderen Einstellungen -- nur als Beispiel, in der
Diplomarbeit sollte man sich an eine einheitliches Format halten.
Bei längeren Listings muss man eventuell mit Umbrüchen rechnen, oder
man verwendet einen Rahmen der frei angeordnet werden kann (\siehe{sub:Bilder}).

% Einstellungen für die fogenden Listings
% entweder mit \begin{listing} oder in Lyx als Programmlisting
\lstset{numbers=right, numberstyle=\tiny, stepnumber=2, numbersep=5pt, showspaces=false, frame=single}
\lstset{language=Java}

Achtung \LaTeX{}-User: Listing kann keine Umlaute, aber unter \cite{listingtipp}
gibt es eine Lösung.

\inputencoding{latin9}\begin{lstlisting}[caption={Java Beispiel},captionpos=b]
import java.awt.*;  
import java.awt.event.*;
public class AL extends Frame
                 implements WindowListener, ActionListener {
  TextField text = new TextField(20);
  Button b;    
  private int numClicks = 0;
 
  public static void main(String[] args) {
    AL myWindow = new AL("My first window");
    myWindow.setSize(350,100);
    myWindow.setVisible(true);    
  } 
}
\end{lstlisting}
\inputencoding{utf8}


\subsection{Fachbegriffe}

Fachbegriffe in einer Fremdsprache oder Kommandos sollten einheitlich
gekennzeichnet werden. Bei Latex verwendet man dazu ,,logisches Markup``,
bei Word oder Open/Libre-office wird all diesen Wörtern wird eine
Vorlage zugewiesen, das Aussehen wird dann an einer Stelle zentral
festgelegt. 

Als Beispiel soll \emph{Text to Speech}\index{Text to Speech: Umwandlung von Texten in Sprache}
dienen. Solche Wörter sollte natürlich in ein Glossar aufgenommen
werden. 

Oder der Befehl \strong{dir} für die Kommandozeile. Die Angabe von
Dateinamen sollte auch einheitlich sein: entweder \emph{/etc/passwd}
oder \strong{C:\textbackslash{}system32}.


\subsection{Zitieren}

Die Quellenangabe kann in Form eines Vollbelegs in der Fußnote%
\footnote{aus Zitat --- Wikipedia, Die freie Enzyklopädie, \url{http://de.wikipedia.org/w/index.php?title=Zitat},
Abgerufen 2014-09-14%
}(bei technischen Dokumenten eher unüblich) oder als Kurzbeleg am Schluss
der gesamten Arbeit aufgeführt werden. Beim Kurzbeleg sind dabei verschiedene
Formen üblich. Der platzsparendste, aber am wenigsten aussagekräftige
Zitierstil ist die fortlaufende Nummerierung aller zitierten Quellen
{[}123{]}.

Insbesondere in der Informatik üblich ist eine Kombination der ersten
drei Buchstaben des Autorennamens und der letzten beiden Ziffern des
Erscheinungsjahres (z. B. „The04“ für Theisen 2004). Wohl am weitesten
verbreitet ist der vollständige Verfassernamen mit Erscheinungsjahr,
wobei mehrere Quellen desselben Autors innerhalb eines Jahres durch
fortlaufende Buchstaben kenntlich gemacht werden (z. B. „Theisen 2004c“).
Weniger üblich, aber am aussagekräftigsten ist die Quellenangabe unter
Hinzufügung eines Schlagwortes, das den mit der Materie vertrauten
Leser zumeist bereits die zitierte Quelle erkennen lässt, z. B. in
der Form „Theisen (Wissenschaftliches Arbeiten, 2004)“.

Obwohl mehrere Zitierstile bzw. Zitiertechniken zur Verfügung stehen,
werden in einem Dokument üblicherweise nicht mehrere verwendet; ein
ausgewählter Zitierstil wird im gesamten Dokument konsequent beibehalten.
Ein gute Übersicht bietet \cite{wiki:zitat}.


\subsubsection{Quellenverzeichnis}

Unter \LaTeX{} kann das Programm Bib\TeX{} zur Erstellung von Literaturangaben
verwendet werden. 
\begin{itemize}
\item Links auf Wikipedia sollten vermieden werden.
\item Jeder Link sollte mit einem Abfragedatum versehen sein.
\item Das Literaturverzeichnis kommt an das Ende des Dokuments.
\end{itemize}
Viele Details dazu findet man bei \cite{wiki:zitat}.


\subsubsection{Rechtliches zum Zitieren}

Achtung: nicht gekennzeichnete Zitate (Plagiate) führen zu einer negativen
Beurteilung der Diplomarbeit.

Nach \cite{wiki:quelle}:

§ 57 des österreichischen Urheberrechtsgesetzes\cite{ris57} enthält
detaillierte Vorschriften über die Quellenangabe, unter anderem: Werden
Stellen oder Teile von Sprachwerken nach §\,46 vervielfältigt, so
sind sie in der Quellenangabe so genau zu bezeichnen, dass sie in
dem benutzten Werke leicht aufgefunden werden können. In den Erläuterungen
(ErlRV) heißt es: Bei Entlehnungen aus umfangreichen Werken muss also
in der Quellenangabe auch die Seite, der Abschnitt, das Kapitel oder
der Akt, wo sich die entlehnte Stelle befindet, angeführt werden (Dillenz,
Materialien zum österreichischen Urheberrecht, 134, zitiert nach \cite{dittrich},
S. 621)

2002 nahm der österreichische OGH zur Frage der Quellenangabe in der
Entscheidung Riven Rock Stellung: Nach § 57 Abs 4 UrhG bedarf die
Unterlassung einer Quellenangabe der Rechtfertigung durch die im redlichen
Verkehr geltenden Gewohnheiten und Gebräuche. Bei Auslegung dieser
Bestimmung ist eine Abwägung der Interessen des Urhebers mit jenen
des zur freien Werknutzung Berechtigten nach dem Verständnis loyaler,
den Belangen des Urhebers mit Verständnis gegenübertretenden, billig
und gerecht denkenden Benutzern (Vinck aaO § 63 Rz 2) geboten und
danach zu beurteilen, ob dem freien Werknutzer neben der Nennung des
Autors/Verlags auch die Nennung des Namens des Übersetzers von in
einer Rundfunksendung verlesenen Roman-Zitaten zumutbar ist.


\section{Bad Practice}

Was man vermeiden sollte -- diese Dinge führen zur Mehrarbeit und
verursachen zusätzlichen Stress in der hektischen Zeit knapp vor dem
Abgabetermin.


\paragraph{Stil}
\begin{itemize}
\item Extrem lange, geschachtelte Sätze und/oder endlose Textpassagen ohne
Gliederung durch Absätze.\nopagebreak

\begin{itemize}
\item Vielleicht bzw. sinnvollerweise lassen Sie den Text auch von einer
,,außenstehenden Person`` lesen. 
\end{itemize}
\item Aufzählungen im Text statt Listen. Wie man hier sieht dürfen bei Listen
auch mehrere Sätze stehen.
\item \uline{Unterstreichen} ist ein Relikt aus ,,Schreibmaschinen-Zeiten``.
\item Eine Diplomarbeit ist keine Erzählung. Natürlich kann man als ,,ich``
oder ,,wir`` auf ,,unsere Probleme`` eingehen, aber im Allgemeinen
ist ein formaler, beschreibender und technischer Stil einzuhalten.
\item Eine Diplomarbeit ist auch keine Email oder SMS: Schreiben Sie ganze
Sätze ohne kryptische Abkürzungen und Smileys.
\item Ein Mindestmaß an Interpunktion wird vorausgesetzt. Eventuell lassen
Sie den Text durch eine kundige Person Ihres Vertrauens korrigieren.
\item Es gibt viele verschiedene Striche, und alle sehen verschieden aus:
Gedankenstriche, Bindestriche und Minus kommen in einer Diplomarbeit
häufig vor.
\item Weitere Wörter die Ihren Betreuer verzweifeln lassen -- natürlich
nur bei übermäßiger Verwendung

\begin{itemize}
\item Welcher/Welches, Hierbei
\end{itemize}
\end{itemize}

\paragraph{Technik}
\begin{itemize}
\item (viele) händische Formatierungen statt Formatvorlagen.
\item zusätzliche manuelle Seitenumbrüche oder Leerzeilen für ein ,,schöneres``
Layout. Es gibt bei den Absatzformatierungen tolle Möglichkeiten für
Abstände vor und nach einem Absatz bzw. zum Beeinflussen des Textflusses.
\item Arbeiten Sie mit dem Programm statt gegen das Programm:

\begin{itemize}
\item Verweise als fixer Text. Nutzen Sie die Möglichkeiten der Textverarbeitung.
\item Dinge die ,,kompliziert`` einzugeben sind, sind meist falsch --
richtige Lösungen sind in allen Programmen auch ,,leicht`` zu erreichen%
\footnote{Oder Sie verwenden ein für Ihre Zwecke schlecht geeignetes Programm.%
}.
\end{itemize}
\item Kontrollieren Sie beim fertigen PDF die Angaben unter Datei / Eigenschaften
-- dort sollten sinnvolle Dinge stehen. Bei Latex wird dazu das Paket
\texttt{\code{\texttt{hyperref}}} verwendet.
\end{itemize}

\section{Details zu Formatierung}


\subsection{Schriftarten}

Die Word- und Libreoffice-vorlage verwenden etwas andere Schriften
als das Latex Dokument.


\section{Beispiele}


\subsection{Zitieren mit Latex}

Am Beispiel der URL \url{http://de.wikibooks.org/wiki/LaTeX-Kompendium}
und des Buches ,,\LaTeX{}: Einführung``.

Man braucht eine \texttt{.bib} Datei mit den notwendigen Informationen:

\begin{lstlisting}[language={[LaTeX]TeX},inputencoding={utf8},extendedchars=false]
@book{kopka1991latex,
  title={LaTeX: Einführung},
  author={Kopka, Helmut and Rahtz, Sebastian},
  volume={2},
  year={1991},
  publisher={Addison-Wesley} 
}

@online{latexKomp,
  author = {},
  title ={LaTeX-Kompendium - Wikibooks, Sammlung freier Lehr-, 
Sach- und Fachbücher},
  url = {http://de.wikibooks.org/wiki/LaTeX-Kompendium},
  lastchecked = {2014.09.14},
}
\end{lstlisting}

Diese Einträge werden dann im Text verwendet: \texttt{\textbackslash{}cite\{kopka1991latex\}
}und das Ergebnis sieht dann so \cite{kopka1991latex} und so \cite{latexKomp}
aus. Gleichzeitig erscheinen diese Einträge auch im Literaturverzeichnis
am Ende des Dokuments (Format: \texttt{plaindin}).


\subsection{Formatierungen }

Der Autor kann natürlich auch eingreifen und zum Beispiel\\
Zeilenumbrüche erzwingen und \\
\\
Leerzeilen -- aber das sollte man nicht machen. \newpage{}

Ein Seitenumbruch kostet mich auch nicht mehr als einen müden Lacher,
ist aber noch seltener wirklich sinnvoll.

Schriftgrößen:
\begin{itemize}
\item {\tiny{}tiny} 
\item {\scriptsize{}scriptsize} 
\item {\footnotesize{}footnotesize} 
\item {\small{}small} 
\item normalsize 
\item {\large{}large} 
\item {\Large{}Large} 
\item {\LARGE{}LARGE} 
\item {\huge{}huge} 
\item {\Huge{}Huge} \end{itemize}
\begin{enumerate}
\item Nummerierungen
\item sind 
\item auch 
\item easy
\end{enumerate}
\textbf{bissl was fettes} \textit{italienisches} \uline{für unten
drunter}.


\subsection{Fülltext}

\Blindtext


\chapter{Planung}

\section{Kapitel aus der anderen
Datei}\label{kapitel-aus-der-anderen-datei}

Dieses Kapitel wurde als \emph{diplomarbeit2.md} geschrieben und dann
mit \emph{pandoc} in \TeX~umgewandelt.

\begin{lstlisting}[language=bash]
pandoc --listings -s diplomarbeit2.md -o diplomarbeit2.tex 
\end{lstlisting}

Wie man sieht ist das ganz einfach, sogar Listings sind möglich. Und nun
zu einem Bild.

\begin{figure}[htbp]
\centering
\includegraphics{HTL3RLogo.png}
\caption{Der Text steht unterhalb}
\end{figure}

Achtung: Pandoc skaliert die Bilder nicht! Hier hilft nur eine
vorhergehende Skalierung des Bildes. Oder nachträgliches Editieren --
ganz einfach die passende Breite in der \emph{.tex} Datei ausbessern.

Man kann auch die Breite aber auch durch La\TeX~Befehle angeben -- das
ändert aber die Standardbreite aller folgenden Bilder!

\setkeys{Gin}{width=0.6\textwidth,}

\begin{figure}[htbp]
\centering
\includegraphics{HTL3RLogo.png}
\caption{Das kleinere Bild}
\end{figure}

\setkeys{Gin}{width=2cm}

\begin{figure}[htbp]
\centering
\includegraphics{HTL3RLogo.png}
\caption{Das ganz kleine Bild}
\end{figure}

Auch Listen sind kein Problem, wichtig sind nur Leerzeilen zwischen den
Listenpunkten. Hier sieht man eine einfache Aufzählung.

\begin{itemize}
\item
  wichtig
\item
  auch ganz lange Texte können bei Listen geschrieben werden.

  Sogar mehrere Absätze sind möglich.
\item
  Ende der Liste.
\end{itemize}

Welches Zeichen am Anfang der Liste steht ist dabei leicht einzustellen,
im \emph{pandoc} Manual gibt es nähere Infos:

\begin{enumerate}
\def\labelenumi{\arabic{enumi}.}
\item
  eins
\item
  zwei

  \begin{enumerate}
  \def\labelenumii{\roman{enumii}.}
  \itemsep1pt\parskip0pt\parsep0pt
  \item
    zwei eins -- Mindestens 4 Zeichen eingerückt
  \item
    zwei zwei
  \end{enumerate}
\item
  drei. \emph{Pandoc} zählt richtig, das Zeichen am Anfang der Zeile ist
  nur ein Muster!
\end{enumerate}


\chapter{Umsetzung}

\Blindtext


\chapter{Ergebnisse}

% wer hat diese Kapitel geschrieben oder leer
\renewcommand{\kapitelautor}{Autor: Susi Sorglos}

\blindmathpaper\Blindtext


\chapter{Evaluation}

% wer hat diese Kapitel geschrieben oder leer 
\renewcommand{\kapitelautor}{Autor: Blindtext}

\Blindtext

\Blinddocument\Blindtext\Blinddocument[2]\Blindtext\Blinddocument[5]\Blindtext\Blinddocument[10]\Blindtext\Blindtext

%%%%%%%%%%%%%%%%%%%%%%%%%%%%%%%%%%%%%%%%%%%%%%%%%%%%%%%%%%%%%%%%%%%%%%%%%%%%%%%%%%%%%%%%%%
% wer hat diese Kapitel geschrieben oder leer
\renewcommand{\kapitelautor}{}

\appendix

\chapter{Anhang 1\label{chap:Anhang-1}}

was auch immer

\printindex{}

\bibliographystyle{plaindin}
\bibliography{diplom}

%\renewcommand*{\acronymtype}{acronym}
\printglossary[type=\acronymtype]
\printglossary

\end{document}
