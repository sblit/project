\subsection{Ausgangssituation}
In der heutigen Zeit nimmt die Technik einen großen Einfluss auf unser tägliches Leben. Jeder durchschnittliche Haushalt besitzt mindestens
einen Computer mit Internetanschluss und die meisten Leute haben heutzutage auch ein Smartphone. Selten allerdings bleibt es bei den zwei
Gerätschaften und so kommt zum Beispiel ein Notebook für die Schule beziehungsweise Arbeit zum Einsatz. Oftmals müssen dabei die selben
Dateien auf verschiedenen Geräten bearbeitet werden und das möglichst ohne große Versionskonflikte (Glossareintrag). Daten müssen von
jedem Gerät und zu jederzeit erreichbar sein, weshalb Cloud- und Dateisynchronisationsdienste einen immer größer werdenden Stellenwert
bekommen. Das outsourcen von Speicherplatz beziehungsweise Rechenleistung auf eine zentrale Stelle, meistens bestehend aus einem Server
und die dahinter liegende Datenbank des jeweiligen Anbieters, soll dabei mit hoher Uptime den Zugriff auf die privaten Dateien sicherstellen. 

\subsection{Funktionsweise üblicher Dateisynchronisationsdienste}
Die Idee hinter \sblit basiert auf den sicherheitsbezogenen Problemen, die mit der Speicherung der Daten auf einer zentralen Stelle
einhergehen. Die Erleuterung der Funktionsweise solch üblicher Dateisynchronsationsdienste soll beim Verstehen der grundsätzlichen Idee und des Motivs hinter \sblit helfen.

Beim Synchronisieren einer Datei, wird eine Kopie der Datei auf den vom Betreiber zur Verfügung gestellten Server beziehungsweise auf die Datenbank dahinter hochgeladen und gespeichert. Wenn diese Datei auf weitere Geräte synchronisiert werden soll, laden sich diese 

Praktisch --> Funktion.
Was stört uns daran?
Wie machen wir es?
Warum ist unseres so cool?
Warum sind wir so geil?
