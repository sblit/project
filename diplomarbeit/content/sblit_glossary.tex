\newglossaryentry{partnerdevice}{
	name=Partnergerät,
	description={Ein Gerät eines anderen \sblit-Nutzers, mit dem eine Partnerschaft besteht}
}
\newglossaryentry{authreq}{
	name=Authenticity Request,
	description={Eine Nachricht zur Anforderung einer Challenge, um die Authentizität des Gegenübers sicherzustellen}
}
\newglossaryentry{authres}{
	name=Authenticity Response,
	description={Die Antwort auf den \gls{authreq}, um die eigene Authentizität zu beweisen}
}
\newglossaryentry{filereq}{
	name=File Request,
	description={Die Anfrage an das Gegenüber, ob eine Datei benötigt wird}
}
\newglossaryentry{fileres}{
	name=File Response,
	description={Die Antwort auf einen File Request, der eine Datei akzeptiert oder verweigert}
}
\newglossaryentry{filemsg}{
	name=File Message,
	description={Die eigentliche Übertragung einer Datei}
}
\newglossaryentry{filedel}{
	name=File Delete Message,
	description={Eine Nachricht, die dazu dient, Geräte über eine Löschung einer Datei zu informieren}
}
\newglossaryentry{filedelpart}{
	name=File Delete Message,
	description={Eine Nachricht, die dazu dient, Dateien von Partnergeräten zu löschen.}
}
\newglossaryentry{logfile}{
	name=Logfile,
	description={Die Datei, in der der Versionsverlauf einer Datei steht. Außerdem werden hier die Geräte, auf denen die Datei schon vorhanden ist, gelistet}
}
\newglossaryentry{refdev}{
	name=Device Refresh Message,
	description={Eine Nachricht, die eine aktualisierte Version von Partnergeräten oder eigenen Geräten enthält}
}
\newglossaryentry{partfilereq}{
	name=Partner File Request,
	description={Ein \gls{filereq} an ein Partnergerät \referenz{Partnergerät} oder von einem Partnergerät}
}
\newglossaryentry{partfileres}{
	name=Partner File Response,
	description={Ein \gls{fileres} an ein Partnergerät \referenz{Partnergerät} oder von einem Partnergerät}
}
\newglossaryentry{partfilemsg}{
	name=Partner File Message,
	description={Eine \gls{filemsg} an ein Partnergerät \referenz{Partnergerät} oder von einem Partnergerät}
}
\newglossaryentry{partfiledel}{
	name=Partner File Delete Message,
	description={Eine \gls{filedel} an ein Partnergerät \referenz{Partnergerät} oder von einem Partnergerät}
}
\newglossaryentry{watchservice}{
	name=WatchService,
	description={Eine Schnittstelle zum Filesystem, welche benachrichtigt wird, wenn sich eine Datei in einem bestimmten Ordner ändert}
}
\newglossaryentry{watchkey}{
	name=WatchKey,
	description={Ein Objekt, das vom \gls{watchservice} erstellt wird, wenn sich etwas in dem zu überwachenden Ordner ändert}
}
%\newacronym[see={[Glossar:]{authreq}}]{authreq}{AuthReq}{Authenticity Request\glsadd{authreq}}

