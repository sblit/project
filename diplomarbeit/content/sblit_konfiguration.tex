\subsection{Allgemein}
Mit dem Konfigurieren von \sblit werden benutzerspezifische Anforderungen berücksichtigt und ein reibungsloser Betrieb sichergestellt. Die Konfiguration wird unter Windows im Appdata-Verzeichnis im Ordner \datei{SBLIT} und auf Unix-basierten Systemen im Home-Verzeichnis im Ordner \datei{.SBLIT} gespeichert. Dieses Verzeichnis sollte jedoch nicht direkt editiert werden. Sämtliche Einstellungen erfolgen über das Einstellungsmenü, das über den System-Tray erreicht werden kann.\\
Die Konfiguration enthält folgende Dateien:
\subsection{Gerätedatei freceivers.txt}
\datei{freceivers.txt} enthält eine Liste der Partnergeräte. Darin wird festgehalten, ob die eigenen Dateien auf Partnergeräten \referenz{Partnergerät} gespeichert werden. Eine detaillierte Beschreibung findet sich im Kapitel \linkt{Partnerschaften}. \\
Die Adress-Bytes der Partnergeräte werden durch Beistriche getrennt in hexadezimaler Form gespeichert. \\ %TODO Formattierung
Zum Schreiben der Datei wird folgender Code verwendet: \\  \\
\javalisting
\begin{minipage}{\linewidth}
\begin{lstlisting}[caption={Schreiben der Gerätedatei freceivers.txt},captionpos=b]
private synchronized void updateDevices(File file,
		Map<String, String> devices) throws IOException {
	String temp = devices.toString();
	temp = temp.substring(1, temp.length() - 1);
	Files.write(file.toPath(), temp.getBytes(),
			StandardOpenOption.TRUNCATE_EXISTING);
}
\end{lstlisting}
\end{minipage}
\begin{description}
	\descriptionitem{file}
	Der Parameter \code{file} legt die Datei fest, die die Adressen der Partnergeräte enthält. Diese Parametrisierung ist deswegen notwendig, weil die Methode \code{updateDevices(file, devices)} auch für das Schrieben von receivers.txt \referenz{receivers} verwendet wird.
	\descriptionitem{devices}
	Der Parameter \code{devices} enthält die Liste aller Geräte, auf denen die Daten gespeichert werden.
	\descriptionitem{IOException}
	\code{IOException} weist auf einen Fehler beim Schreiben der Datei hin.
	\descriptionitem{temp}%TODO Variablennamen ändern
	\code{temp} enthält den Inhalt, der in die Datei geschrieben werden soll. 
\end{description}
In den Zeilen 3 und 4 werden die Adressen der Partnergeräte in das Speicherformat übertragen. In Zeile 4 werden die durch die \code{.toString()}-Methode  eingefügten geschwungenen Klammern wieder entfernt. In Zeile 5 wird die im \code{file}-Parameter angegebene Datei mit den neuen Adressen aktualisiert. \code{StandardOpenOption.TRUNCATE\_EXISTING} bedeutet, dass die Datei \datei{freceivers.txt} überschrieben wird, falls diese schon vorhanden ist.

\subsection{\gls{logfile} logs.txt}
Das \gls{logfile} \referenz{Logfile} \datei{logs.txt} enthält die Versionsverläufe aller Dateien im \sblit-Ordner und dessen Unterordnern im Format: \\ \code{Pfad1=Version1,Version2;Adresse1,Adresse2;Pfad2} \\
\code{Pfad1} bezeichnet den relativen Pfad vom \sblit-Ordner zur betrachteten \datei{Datei 1}. Die einzelnen Versionen (\code{Version1}, \code{Version2}) sind Hashwerte der \datei{Datei 1} zu verschiedenen Zeitpunkten, deren Bytes in hexadezimaler Form im \gls{logfile} gespeichert werden. Analog dazu werden auch die Adressen der Geräte, die bereits die aktuellste Version der \datei{Datei 1} enthalten, in hexadezimaler Form gespeichert. \code{Pfad2} ist der Pfad zu jener Datei, mit der \datei{Datei 1} in Konflikt steht. Dieser ist optional.\\
Zum Schreiben des \gls{logfile}s, wird folgende Methode verwendet: \\ \\
\javalisting
\begin{minipage}{\linewidth}
\begin{lstlisting}[caption={Schreiben des Logfiles \datei{logs.txt}},captionpos=b]
private synchronized void write(
		Map<String, LinkedList<Data>> versions,
		Map<String, LinkedList<Data>> synchronizedDevices
		Map<String, String> conflictOf)
		throws IOException {
	String logs = "";
	if (versions.size() > 0) {
		for (String path : versions.keySet()) {
			logs += "\n" + path + "=";
			String temp = "";
			for (Data data : versions.get(path)) {
				temp += "," + data.toString();
			}
			logs += temp.substring(1) + ";";
			temp = "";
			for (Data data : synchronizedDevices
					.get(path)) {
				temp += "," + data.toString();
			}
			logs += temp.substring(1);
			if(conflictOf.get(path) != null) {
				temp += ";" + conflictOf.get(path);
			}
		}
		logs = logs.substring(1);
	}
	Files.write(logFile.toPath(), logs.getBytes());
}
\end{lstlisting}
\end{minipage}
\begin{description}
	\descriptionitem{versions}
	\code{versions} enthält alle im \sblit-Ordner vorhandenen Dateien und deren Versionsverlauf.
	\descriptionitem{synchronizedDevices}
	\code{synchronizedDevices} enthält alle im \sblit-Ordner vorhandenen Dateien und die Geräte, auf denen die aktuellste Version bereits gespeichert ist.
	\descriptionitem{logs}%TODO eventuell "Versionsverlauf" ändern
	\code{logs} enthält die formatierten Versionsverläufe.
	\descriptionitem{temp}
	Diese Variable enthält einen bestimmten Versionsverlauf, der gerade bearbeitet wird.
\end{description}
Die \code{if}-Anweisung in Zeile 6 überprüft, ob Dateien in der Versionsliste \code{versions} überhaupt vorhanden sind, um sie dann mithilfe der \code{for}-Schleife in Zeile 7 zu separieren. In Zeile 8 wird die \code{logs} Variable in eine neue Zeile geschrieben und um den neuen Pfad ergänzt. Weiters wird hier ein \code{=} angefügt, um den Pfad nachher noch von der Logdatei trennen zu können. In der nächsten Zeile wird die Variable temp erstellt und in der darauf folgenden \code{for}-Schleife um die Versionen, getrennt durch einen Beistrich, ergänzt. In Zeile 13 wird der erste Beistrich mit der Anweisung \code{temp.substring(1)} gekürzt und ein ";" angefügt, um die Versionen von den Geräten, auf denen die Datei synchronisert ist, zu trennen. Anschließend wird die Variable \code{temp} zurückgesetzt und die Geräte genau so wie die Versionen in die Variable \code{temp} geschrieben. Weiters wird, falls vorhanden, der Pfad zur Datei, von der diese Datei die Konfliktdatei ist, ebenfalls angegeben. Ob eine solche Datei existiert wird durch die \code{if}-Anweisung in Zeile 20 überprüft. In Zeile 21 wird diese, falls vorhanden, ebenfalls getrennt durch einen ";" angefügt. Schließlich wird der Zeilenumbruch am Anfang der Variable \code{logs} entfernt und die Datei wird in Zeile 26 geschrieben.

\subsection{Gerätedatei receivers.txt} \label{receivers}
\datei{receivers.txt} enthält eine Liste der Geräte, auf denen der \sblit-Ordner synchronisiert werden soll. Zu diesen Geräten wird optional ein Name gespeichert, um es dem Benutzer zu erleichtern, den Überblick über die Geräte zu bewahren. Diese werden in der Form \code{Adresse=Name} gespeichert, wobei die Adresse wieder eine hexadezimale Darstellung der Bytes ist und der Name vom Nutzer beliebig vergeben werden kann.

\subsection{rk.txt}
\datei{rk.txt} enthält den Private-Key des Gerätes. Dieser dient dazu, den Besitz Adresse zu beweisen und ist unbedingt geheim zu halten, da sonst fremde Geräte Identitätsklau begehen könnten. Der Private-Key wird in hexadezimaler Form in der Datei gespeichert. Die Datei ist außerdem versteckt.

\subsection{sblitDirectory.txt}
\datei{sblitDirectory.txt} enthält den Pfad zum \sblit-Ordner. Darin wird ein \code{String} gespeichert, der den Dateipfad enthält. 

\subsection{symmetricKey.txt} \label{symmetricKey}
In \datei{symmetricKey.txt} wird der symmetrische Schlüssel festgehalten. Dieser dient zum Entschlüsseln der Daten, die auf den Partnergeräten gespeichert werden. Der symmetrische Schlüssel wird, wie alle binären Daten, in hexadezimaler Form in der Datei gespeichert. Zusätzlich ist diese Datei versteckt.

\subsection{uk.txt}
\datei{uk.txt} enthält den Public-Key des Gerätes, der gleichzeitig die Geräteadresse ist. Beim Verbinden mit \gls{dcl} wird diese Adresse bekanntgegeben. Der Schlüssel werden in Hexadezimalzahlen umgewandelt und in der Datei gespeichert.
