
\subsection{Einleitung}
Das \gls{asproto} wird zur Kommunikation zwischen \glspl{service} und über DCL
kommunizierenden Anwendungen verwendet. Es dient zur Bekanntgabe von
öffentlichen Schlüsseln, die als Adresse verwendet werden, zum Beitritt zu
\glslink{net}{Netzwerken}, zum Senden von verbindungslosen Paketen, zur
Verbindung von \glspl{appch} und zur Übertragung von über \glspl{appch}
gesendeten Nutzdaten.

Dieses Unterkapitel beschreibt das \gls{asproto} in Revision 0.

\subsection{Sicherheit}
Das \gls{asproto} ist zur Kommunikation zwischen einem \gls{service} und einer
Anwendung, die auf der selben Maschine wie der \gls{service} läuft, vorgesehen.
Die Sicherheitsvorkehrungen des Protokolls sind für den Einsatz auf
Verbindungen basierend auf dem \acrfull{tcp}, die nur über das
Loopback-Interface des lokalen Rechners und nicht über ein Netzwerk übertragen
werden, angepasst.
Über das \gls{asproto} werden unter anderem Anweisungen zum Verschlüsseln von
Daten gegeben.
Bei solch einer Anweisung kommt es zuerst zur Klartextübertragung dieser Daten
und kurze Zeit später zur erneuten Übertragung dieser Daten in verschlüsselter
Form.
Die Möglichkeit, eine \gls{asconn} abzuhören, würde ein sehr großes
Sicherheitsrisiko darstellen, da das \gls{asproto} keinerlei Verschlüsselung
der \gls{asconn} vorsieht.
Die \gls{asconn} sollte deshalb immer nur auf dem Localhost erfolgen und niemals
über ein Netzwerk.

\subsection{Network Endpoint Slots}
Nach dem derzeitigem Entwicklungsstand des \gls{asproto} kann eine Anwendung
auf einer \gls{asconn} zwar nur eine Adresse bekanntgeben, mit dieser Adresse
kann jedoch mehreren \glspl{net} beigetreten werden.
Eine Kombination aus der bekanntgegebenen Adresse und einem \gls{net}, dem die
Anwendung mit dieser Adresse beigetreten ist, ergibt einen \gls{asendp}.
\glspl{asendp} innerhalb von \glspl{asconn} entsprechen den
\glslink{endp}{Endpunkten} innerhalb von \glspl{isch}.
Um sich in Messages einfach auf diese \glspl{asendp} beziehen zu können, werden
im \gls{asproto} \glspl{endpslot} verwendet. Dabei handelt es sich im
Wesentlichen um Zahlen, die jeweils auf genau einen \gls{asendp}, also eine
Kombination aus Adresse und \gls{net}, verweisen.

\subsection{Remote Keys}
\label{dcl-asproto-remotekeys}
Um zu verhindern, dass der private Schlüssel des Schlüsselpaars, das als Adresse
verwendet wird, zum \gls{service} übertragen werden muss und trotzdem die zum
Adressbeweis nötigen kryptographischen Operationen ausführen zu können, kommen
in der Implementierung des \gls{dcl} sogenannte \glspl{remotekey} zur Anwendung.

Dabei führt der \gls{service} die kryptographischen Operationen, die
beispielsweise für \glspl{cchlg} zum Beweis der Echtheit der Adresse notwendig
sind, nicht selbst aus, sondern leitet diese an die Anwendung weiter.

Konkret sendet der \gls{service} dazu \msgpl{\asprotokeyenc} zum Verschlüsseln
von unverschlüsselten Daten und \msgpl{\asprotokeydec} zum Entschlüsseln von
verschlüsselten Daten an die Anwendung. Diese Antwortet nach Ausführung der
Operation mit jeweils einer \msg{\asprotocryptoresponse}, die die
verschlüsselten bzw. die entschlüsselten Daten enthält.

In der Implementierung des \gls{service} werden \glspl{remotekey} dabei als
normale \code{Key}-Objekte dargestellt, die bei Aufruf der \code{encrypt()}- und
\code{decrypt()}-Methoden die Anfragen samt Daten an die Anwendung weiterleiten
und die Kommunikation in einem separaten Thread abwickeln.
Der Methodenaufruf blockt dabei so lange, bis die Anfrage von der Anwendung
bearbeitet wurde.

\subsection{Initialisierung}
Die Kommunikation über das \gls{asproto} wird mit der Aushandlung der zu
verwendenden Revision des Protokolls begonnen.
Dazu werden, beginnend mit der Anwendung, so lange \msgpl{\asprotorevision}
abwechselnd zwischen \gls{service} und Anwendung gesendet, bis die letzten
zwei übertragenen \msgpl{\asprotorevision} die selben Revisionsnummern
enthalten.

\subsection{Messageaufbau}
Eine Message im \gls{asproto} ist generell so aufgebaut, dass ein einzelnes Byte
am Anfang der Message den Typ angibt, anhand dessen die restliche Nachricht
interpretiert wird.

\asprotobytefield

\subsection{Messages}

\subsubsection{Revision}
\label{dcl-asproto-revision}
Die \msg{\asprotorevision} enthält eine Versionsnummer in Form eines
FlexNum-Components und wird benutzt, um am Beginn der Verbindung die verwendete
Revision des \gls{asproto} auszuhandeln.

\asprotorevisionbytefield


\subsubsection{Generate Key}
\label{dcl-asproto-genkey}
Die \msg{\asprotogenkey} kann von der Anwendung benutzt werden, wenn keine fixe
Adresse notwendig ist und der \gls{service} die Generierung einer temporären
Adresse übernehmen soll. Die Übertragung von Messages für kryptographische
Funktionen wie der \msg{\asprotokeyenc}, der \msg{\asprotokeydec} oder der
\msg{\asprotocryptoresponse} fällt dadurch weg.
Die \msg{\asprotogenkey} ersetzt außerdem die \msg{\asprotoaddrpubkey}.

\asprotogenkeybytefield


\subsubsection{Join Network}
\label{dcl-asproto-joinnet}
Die \msg{\asprotojoinnet} wird von der Anwendung benutzt, um mit der Adresse
einem bestimmten Netzwerk beizutreten.
Nach erfolgreichem Beitritt antwortet der \gls{service} mit einer
\msg{\asprotoslotassign}, die den \gls{endpslot} für die Adresse und dieses
\gls{net} angibt.

\asprotojoinnetbytefield


\subsubsection{Slot Assign}
\label{dcl-asproto-slotassign}
Die \msg{\asprotoslotassign} wird vom \gls{service} als Antwort auf eine
\msg{\asprotojoinnet} bzw. \msg{\asprotojoindefnets} gesendet und beinhaltet
den \gls{endpslot} des neu angelegten \gls{asendp}, dessen \gls{nt} und die
Daten, die innerhalb dieses \glslink{net}{Netzwerks} als skalierte Adresse zum
Routing verwendet werden. Letztere würden im Fall des \gls{cnet} also den Daten
entsprechen, die durch das unter \labellink{dcl-addr-scaling-cnet} auf
\pagelink{dcl-addr-scaling-cnet} angeführte Codebeispiel generiert werden.

\asprotoslotassignbytefield


\subsubsection{Data}
\label{dcl-asproto-data}
Die \msg{\asprotodata} wird sowohl von \gls{service} als auch Anwendung benutzt,
um verbindungslose Pakete zu übertragen.
Die Message enthält den \gls{endpslot}, von dem aus die Nachricht gesendet wird
bzw. an den das Paket adressiert ist, die skalierte Quelladresse des
verbindungslosen Pakets und die Nutzdaten des Pakets.

\asprotodatabytefield


\subsubsection{Address Public Key}
\label{dcl-asproto-addrpubkey}
Die \msg{\asprotoaddrpubkey} wird von der Anwendung benutzt, um den öffentlichen
Schlüssel bekanntzugeben, der als Adresse verwendet werden soll.
Alternativ kann durch Senden einer \msg{\asprotogenkey} auch der \gls{service}
aufgefordert werden, für die Anwendung einen neues, temporäres Schlüsselpaar
für die Verwendung als Adresse zu generieren.

\asprotoaddrpubkeybytefield


\subsubsection{Join Default Networks}
\label{dcl-asproto-joindefnets}
Die \msg{\asprotojoindefnets} kann von der Anwendung gesendet werden, wenn
an das \gls{net} keine besonderen Anforderungen gestellt werden und dem
gängigsten \gls{net} bzw. den gängigsten \glspl{net} beigetreten werden soll.
Nach erfolgreichem Beitritt antwortet der \gls{service} mit einer oder mehreren
\msgpl{\asprotoslotassign}, die die \glspl{endpslot} für die Adresse und die
\glspl{net} angibt.

\asprotojoindefnetsbytefield


\subsubsection{Key Encrypt}
\label{dcl-asproto-keyenc}
Die \msg{\asprotokeyenc} wird vom \gls{service} benutzt, um die Anwendung
aufzufordern, die enthaltenen Daten mit dem privaten Schlüssel zu verschlüsseln,
der aus dem selben Schlüsselpaar stammt wie der als Adresse verwendete und mit
der \msg{\asprotoaddrpubkey} bekanntgegebene öffentliche Schlüssel.
Die Anwendung übermittelt die verschlüsselten Daten durch Senden einer
\msg{\asprotocryptoresponse}.

\asprotokeyencbytefield


\subsubsection{Key Decrypt}
\label{dcl-asproto-keydec}
Die \msg{\asprotokeyenc} wird vom \gls{service} benutzt, um die Anwendung
aufzufordern, die enthaltenen Daten mit dem privaten Schlüssel zu entschlüsseln,
der aus dem selben Schlüsselpaar stammt wie der als Adresse verwendete und mit
der \msg{\asprotoaddrpubkey} bekanntgegebene öffentliche Schlüssel.
Die Anwendung übermittelt die entschlüsselten Daten durch Senden einer
\msg{\asprotocryptoresponse}.

\asprotokeydecbytefield


\subsubsection{Key Crypto Response}
\label{dcl-asproto-cryptoresponse}
Die \msg{\asprotocryptoresponse} wird von der Anwendung benutzt, um die
verschlüsselten Daten einer vorhergehenden \msg{\asprotokeyenc} bzw. die
entschlüsselten Daten einer vorhergehenden \msg{\asprotokeydec} zurück an den
\gls{service} zu übermitteln.

\asprotocryptoresponsebytefield


\subsubsection{Application Channel Outgoing Request}
\label{dcl-asproto-appchoutreq}
Die \msg{\asprotoappchoutreq} wird von der Anwendung benutzt, um einen
\gls{appch} anzufordern. Die Message überträgt den \gls{endpslot}, von dem aus
der \gls{appch} angefragt werden soll, den \gls{appchslot}, dem der neue
\gls{appch} zugewiesen werden soll, den \gls{actid} des neuen \gls{appch} und
der öffentliche Schlüssel, der vom Ziel als Adresse benutzt wird.

\asprotoappchoutreqbytefield


\subsubsection{Application Channel Incoming Request}
\label{dcl-asproto-appchinreq}
Die \msg{\asprotoappchinreq} wird vom \gls{service} benutzt, um die Anwendung
über eine eingehende Anfrage über einen neuen \gls{appch} zu benachrichtigen.
Die Message überträgt den \gls{endpslot}, an den die Anfrage gerichtet ist,
den \gls{actid} des neuen \gls{appch}, den öffentlichen Schlüssel, den die
Quelle der Anfrage als Adresse verwendet, die \gls{lla} des \gls{service}, von
dem die Anfrage gesendet wurde und die \gls{ignoredata}, die als Präfix für
\glspl{nattpkt} verwendet werden sollen.

\asprotoappchinreqbytefield


\subsubsection{Application Channel Accept}
\label{dcl-asproto-appchaccept}
Die \msg{\asprotoappchaccept} wird von der Anwendung benutzt, um eine eingehende
Anfrage über einen neuen \gls{appch} zu bestätigen.
Die Message überträgt den \gls{appchslot}, dem der neue \gls{appch} zugewiesen
werden soll. Außerdem überträgt die Message eine Kopie der Daten der ihr
vorhergehenden \msg{\asprotoappchinreq}, also den \gls{endpslot}, an den die
ursprüngliche Anfrage gerichtet war, den \gls{actid} des neuen \gls{appch},
den öffentlichen Schlüssel, den die Quelle der ursprünglichen Anfrage als
Adresse verwendet, die \gls{lla} des \gls{service}, von dem die Anfrage gesendet
wurde und die \gls{ignoredata}, die als Präfix für \glspl{nattpkt} verwendet
werden sollen.

\asprotoappchacceptbytefield


\subsubsection{Application Channel Connected}
\label{dcl-asproto-appchconnected}
Die \msg{\asprotoappchconnected} wird vom \gls{service} benutzt, um die
Anwendung darüber in Kenntnis zu setzen, dass der vom angegebenen
\gls{appchslot} referenzierte \gls{appch} erfolgreich verbunden wurde und
mittels \msg{\asprotoappchdata} Daten darüber übertragen werden können.

\asprotoappchconnectedbytefield


\subsubsection{Application Channel Data}
\label{dcl-asproto-appchdata}
Die \msg{\asprotoappchdata} wird von sowohl Anwendung als auch \gls{service}
bentutz, um Daten aus dem vom angegebenen \gls{appchslot} referenzierten
\gls{appch} zu übertragen.

\asprotoappchdatabytefield


\subsubsection{Key Encryption Block Size Request}
\label{dcl-asproto-keyencblocksizereq}
Die \msg{\asprotokeyencblocksizereq} wird vom \gls{service} benutzt, um von der
Anwendung die maximal zulässige Datenmenge abzufragen, die mit dem als Adresse
angegebenen öffentlichen Schlüssel in einem Zug verschlüsselt werden kann.
Die Anwendung antwortet unmittelbar darauf mit einer \msg{\asprotokeynum}, die
diese Datenmenge in Bytes enthält.

\asprotokeyencblocksizereqbytefield


\subsubsection{Key Number Response}
\label{dcl-asproto-keynum}
Die \msg{\asprotokeynum} wird von der Anwendung benutzt, um auf Anfragen
bezüglich Zahlenwerte des als Adresse verwendeten öffentlichen Schlüssel zu
antworten.
Der Zahlenwert wird in Form eines \gls{fnumc} übertragen.
Derzeit ist die einzige Anfrage, die eine Antwort durch eine
\msg{\asprotokeynum} erfordert, die \msg{\asprotokeyencblocksizereq}.

\asprotokeynumbytefield
