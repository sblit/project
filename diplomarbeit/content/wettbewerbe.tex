\chapter{Wettbewerbe}
\renewcommand{\kapitelautor}{Autor: Andreas Novak}

\section{Allgemein}
% Projektidee ist cool und wir erkennen das Potential hinter der Idee.
% innovativ, behandelt aktuelle Themen -> Netzwerksicherheit + Anonymität bzw. Schutz der Identität im Internet
% Wettbewerbe sind einfachste Methode des Marketings für Schulprojekte/Diplomarbeiten
% Anwendung für jedermann,
Jedes Jahr auf's Neue gibt es für SchülerInnen und StudentenInnen die Möglichkeit, ihre privaten-
oder die ihm Rahmen ihrer Ausbildung entstandenen Projekte, bei Projekt-Wettbewerben
einzureichen. Für Diplomarbeiten, wie \sblit, stellen diese Wettbewerbe eine einfache
Methode dar, um effektives Marketing durch das Vorstellen des Projekts bei Events eben solcher
Wettbewerbe zu betreiben.

\section{Wettbewerbe}
\subsection{Allgemein}
% An unserer Schule gibt es übliche Verdächtige, von denen wir auch informiert werden.


\subsection{Jugend Innovativ}
\subsubsection{Allgemein}
Jugend Innovativ ist der größte Wettbewerb für innovative Schülerprojekte in Österreich und
bietet die Möglichkeit,
\subsubsection{Einreichung}

\subsubsection{Ergebnis}
Vorerst (Stand April 2015) sind nur die Projekte bekannt, die es ins Halbfinalisten des
Wettbewerbs bekannt -- sblit ist eines von ihnen. Deshalb wurde das Diplomarbeitsteam
zum Halbfinal-Event eingeladen, um das Projekt mit einem eigenen Stand dort vertreten
zu können. Anwesend sind einige Wirtschaftsvertreter und darunter auch sicherlich viele
Interessenten.

Für den einfachen Austausch von Kontaktinformationen mit Interessenten, wurden Visitenkarten
designt und ausgedruckt siehe Visitenkarten.

\subsection{AXAWARD}
\subsubsection{Allgemein}

\subsubsection{Einreichung}

\subsubsection{Ergebnis}


\subsection{u19 -- Create your World}
\subsubsection{Allgemein}
u19 – CREATE YOUR WORLD ist der Kinder- und Jugenswettbewerb des \gls{Prix Ars Electronica}.
Hier wird einem die Möglichkeit geboten, die Welt von Morgen mitzugestalten und eigene Vorstellungen
und Ideen zu realisieren und zu präsentieren.


Das Motto: Mit Mut und Kreativität ist alles erreichbar!

\subsubsection{Einreichung}
Gefordert wurde
\subsubsection{Ergebnis}


\subsection{Internet of Things Cup}
\subsubsection{Allgemein}

\subsubsection{Einreichung}

\subsubsection{Ergebnis}
