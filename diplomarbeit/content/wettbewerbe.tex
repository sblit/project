\chapter{Wettbewerbe}
\renewcommand{\kapitelautor}{Autor: Andreas Novak}

\section{Allgemein}
% Projektidee ist cool und wir erkennen das Potential hinter der Idee.
% innovativ, behandelt aktuelle Themen -> Netzwerksicherheit + Anonymität bzw. Schutz der Identität im Internet
% Wettbewerbe sind einfachste Methode des Marketings für Schulprojekte/Diplomarbeiten
% Anwendung für jedermann,
Jedes Jahr auf's Neue gibt es für SchülerInnen und StudentenInnen die Möglichkeit, ihre privaten-
oder die ihm Rahmen ihrer Ausbildung entstandenen Projekte, bei Projekt-Wettbewerben
einzureichen. Für Diplomarbeiten wie \sblit, stellen diese Wettbewerbe eine einfache und effektive
Methode dar um Marketing zu betreiben und den Bekanntheitsgrad des Projekts zu steigern.

Mit einem Projekt wie \sblit, dass eine kreative Idee besitzt, aktuelle Themen behandelt und technischen Tiefgang hat,
finden sich leicht Interessenten unter den zahlreichen Besuchern von Projektwettbewerben.

Durch die HTLW3R-Matura-News wurden die Mitglieder aller Diplomarbeiten über diverse Vertreter
solcher Wettbewerbe informiert. Das \sblit-Diplomarbeitsteam hat daraufhin das Projekt bei folgenden
Wettbewerben eingereicht.

\section{Wettbewerbe}
% An unserer Schule gibt es übliche Verdächtige, von denen wir auch informiert werden.
\subsection{Jugend Innovativ}
\subsubsection{Allgemein}
Jugend Innovativ ist der größte Wettbewerb für innovative Schülerprojekte in Österreich und
bietet für Jugendliche die Möglichkeit, ihre innovativen Projekte von einer Fachjury bewerten zu lassen
und
%TODO

\subsubsection{Einreichung}
Für
%TODO

\subsubsection{Ergebnis}
Vorerst (Stand April 2015) sind nur die Projekte bekannt, die es ins Halbfinalisten des
Wettbewerbs geschafft haben. \sblit ist dabei das einzige von vielen eingereichten Projekten
der HTLW3 Rennweg, dass es bis in die nächste Runde geschafft hat. Aus diesem Grund wurde das Diplomarbeitsteam
zum Halbfinal-Event eingeladen, um das Projekt mit einem eigenen Stand dort vorzustellen.
Anwesend sind einige Wirtschaftsvertreter und darunter auch sicherlich viele
Interessenten.

Für den einfachen Austausch von Kontaktinformationen mit Interessenten, wurden Visitenkarten
designt und ausgedruckt siehe Visitenkarten.


\subsection{u19 -- Create your World}
\subsubsection{Allgemein}
u19 – CREATE YOUR WORLD ist der Kinder- und Jugenswettbewerb des \gls{Prix Ars Electronica}.
Hier wird einem die Möglichkeit geboten, die Welt von Morgen mitzugestalten und eigene Vorstellungen
und Ideen zu realisieren und zu präsentieren.


Das Motto: Mit Mut und Kreativität ist alles erreichbar!

\subsubsection{Einreichung}
Ursprünglich wurde das Projekt selbst als Dateiabgabe oder eine technische Dokumentation als Abgabe gefordert. Da \sblit zu diesem Zeitpunkt
aber noch nicht fertig war und auch nicht als einzelne Datei abgegeben hätte werden können,
einigte man sich mit den Organisatoren darauf, einen Projektbericht einzureichen, der die Projektidee
beschreibt und noch nähere technische Details zu den einzelnen Teilgebieten und den Projektstatus beinhaltet.
Siehe Anhang ...

\subsubsection{Ergebnis}
Die Ergebnisse wurden noch nicht bekanntgegeben und es wird auf weitere Informationen gewartet.

\subsection{AXAWARD}
\subsubsection{Allgemein}
AXAWARD steht für AUSTRIAN X.TEST AWARD und ist ein Wettbewerb für technische Schulprojekte und Forschungsarbeiten
von Teams mit bis zu 3 SchülerInnen/StudentInnen und wurde von der österreichischen Messtechnik-Firma x.test veranstaltet.

\subsubsection{Einreichung}
Auch beim AXAWARD wurde das Projekt selbst als Dateiabgabe oder eine technische
Dokumentation als Abgabe gefordert. Aber auch hier wurde mit dem Wettbewerbsausschuss ausgemacht,
dass ein Projektbericht eingereicht werden kann. Siehe Anhang ...

\subsubsection{Ergebnis}
Die Finalteams des AXAWARD's wurden am 9. April 2015 per Mail bekanntgegeben, wobei es \sblit nicht
unter die besten 10 geschafft hat. Genauere Platzierungen wurden nicht bekannt gegeben.


\subsection{Internet of Things Cup}
\subsubsection{Allgemein}
Der Internet of Things Cup ist auch ein Projektwettbewerb für SchülerInnen und StudentInnen,
um mit einer technischen Projektidee mit etwas Glück Preise zu gewinnen.

Dem Diplomarbeitsteam wurde dabei kostenlose Hard- und Software zur Verfügung in Form eines
Beaglebone-Mikrocomputers, einer Jahreslizenz für Microsofts Azure Datenbank und Conrad Gutscheine
zur Verfügung gestellt.

\subsubsection{Einreichung}
Bis 24. April ist die Abgabe des Projekt-Zwischenberichts. Weitere Infos Folgen.
%TODO

\subsubsection{Ergebnis}
Die Siegerehrung des IoT Cups findet am 3. Juli 2015 statt. Ergebnisse wurden noch nicht bekanntgegeben,
da dieser Wettbewerb noch am Laufen ist.
