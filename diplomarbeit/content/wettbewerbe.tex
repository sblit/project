\chapter{Wettbewerbe}
\renewcommand{\kapitelautor}{Autor: Andreas Novak}

\section{Allgemein}
% Projektidee ist cool und wir erkennen das Potential hinter der Idee.
% innovativ, behandelt aktuelle Themen -> Netzwerksicherheit + Anonymität bzw. Schutz der Identität im Internet
% Wettbewerbe sind einfachste Methode des Marketings für Schulprojekte/Diplomarbeiten
% Anwendung für jedermann,

\section{Wettbewerbe}
\subsection{Allgemein}
% An unserer Schule gibt es übliche Verdächtige, von denen wir auch informiert werden.
Jedes Jahr auf's Neue gibt es für SchülerInnen und StudentenInnen die Möglichkeit, ihre privaten-
oder die ihm Rahmen ihrer Ausbildung entstandenen Projekte, bei Projekt-Wettbewerben
einzureichen. Für Diplomarbeiten, wie \sblit, stellen diese Wettbewerbe eine einfache
Methode dar, um effektives Marketing durch das Vorstellen des Projekts bei Events solcher
Wettbewerbe zu betreiben.

\subsection{Jugend Innovativ}
\subsubsection{Allgemein}

\subsubsection{Einreichung}

\subsubsection{Ergebnis}


\subsection{AXAWARD}
\subsubsection{Allgemein}

\subsubsection{Einreichung}

\subsubsection{Ergebnis}


\subsection{u19 -- Create your World}
\subsubsection{Allgemein}
u19 – CREATE YOUR WORLD ist der Kinder- und Jugenswettbewerb des Prix Ars Electronica.
Hier habt ihr die Möglichkeit, die Welt von Morgen mitzugestalten und eure Vorstellungen
und Ideen zu realisieren und zu präsentieren. Mit Mut und Kreativität ist alles erreichbar!

\subsubsection{Einreichung}

\subsubsection{Ergebnis}


\subsection{Internet of Things Cup}
\subsubsection{Allgemein}

\subsubsection{Einreichung}

\subsubsection{Ergebnis}
