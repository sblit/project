\chapter{Wettbewerbe}
\renewcommand{\kapitelautor}{Autor: Andreas Novak}

\section{Allgemein}
Jedes Jahr gibt es für Studierende, Schülerinnen und Schüler die Möglichkeit, ihre privaten
oder die ihm Rahmen ihrer Ausbildung entstandenen Projekte bei Projekt-Wettbewerben
einzureichen. Für Diplomarbeiten wie \sblit stellen diese Wettbewerbe eine einfache und effektive
Methode dar um Marketing zu betreiben und den Bekanntheitsgrad des Projekts zu steigern.

Mit einem Projekt wie \sblit, das eine kreative Idee besitzt, aktuelle Themen behandelt und technischen Tiefgang hat,
finden sich leicht Interessenten unter den zahlreichen Besuchern von Projektwettbewerben.

Durch die Schule wurden die Mitglieder aller Diplomarbeiten über diverse Vertreter
solcher Wettbewerbe informiert. Das \sblit-Diplomarbeitsteam hat das Projekt daraufhin bei einigen
Wettbewerben eingereicht. Der folgende Abschnitt behandelt die Wettbewerbe, bei denen das Projekt
eingereicht und dafür erforderlichen Dokumente.

\section{Wettbewerbe}
% An unserer Schule gibt es übliche Verdächtige, von denen wir auch informiert werden.
\subsection{Jugend Innovativ}
\subsubsection{Allgemein}
Jugend Innovativ ist der größte Wettbewerb für innovative Schulprojekte in Österreich und
bietet Jugendlichen die Möglichkeit, ihre innovativen Projekte von einer Fachjury bewerten zu lassen
und mit etwas Glück unter den Gewinnern zu landen, denen ein Preisgeld von bis zu 2000€ zugesprochen wird.

\subsubsection{Einreichung}
Nachdem \sblit in der Kategorie idea.goes.app eingereicht  wurde musste für den Wettbewerb ein
Projektstatusbericht verfasst werden, der einen einen genauen Einblick in das Projekt geben soll
und zusätzlich auch die Ergebnisse aufzeigt, die bis dato erzielt wurden.

\subsubsection{Ergebnis}
Vorerst (Stand April 2015) sind nur die Halbfinalisten der teilnehmenden Projekten bekannt.
Aus diesem Grund wurde das Diplomarbeitsteam
zum gleichnamigen Halbfinal-Event eingeladen, um das Projekt vorort auszustellen.

Um den eigenen Projektstand zu schmücken wurde ein Plakat angefertigt, auf dem die Eckdaten unseres
Projekts einsehbar sind.
Zusätzlich wurden zum erleichterten Weiterreichen der Kontaktinformationen des Projektteams Visitenkarten
entworfen und gedruckt (\siehe{appendix}).

\subsection{u19 -- Create your World}
\subsubsection{Allgemein}
u19 – CREATE YOUR WORLD ist der Jugendwettbewerb des Prix Ars Electronica, einem der traditionsreichsten Medienkunstwettbewerbe Österreichs.
Hier wird die Möglichkeit geboten, eigene Vorstellungen und Ideen zu realisieren und zu präsentieren.

\subsubsection{Einreichung}
Ursprünglich wurde das Projekt selbst als Dateiabgabe oder eine technische Dokumentation als Abgabe gefordert. Da \sblit zu diesem Zeitpunkt
aber noch nicht fertiggestellt wurde und auch nicht als einzelne Datei abgegeben hätte werden können,
einigte man sich mit den Organisatoren darauf, einen Projektfolder einzureichen, der die Projektidee
beschreibt und noch nähere technische Details zu den einzelnen Teilgebieten und den Projektstatus beinhaltet (\siehe{appenix}).


\subsubsection{Ergebnis}
Die Ergebnisse wurden noch nicht bekanntgegeben (Stand April 2015).

\subsection{AXAWARD}
\subsubsection{Allgemein}
AXAWARD steht für AUSTRIAN X.TEST AWARD und ist ein Wettbewerb für technische Schulprojekte und Forschungsarbeiten
von Teams mit bis zu 3 Mitgliedern und wird von der österreichischen Messtechnik-Firma x.test veranstaltet.

\subsubsection{Einreichung}
Auch beim AXAWARD wurde das Projekt selbst als Dateiabgabe oder eine technische
Dokumentation als Abgabe gefordert. Aufgrund dessen wurde auch hier der Projektfolder eingereicht (\siehe{appenix}).

\subsubsection{Ergebnis}
Die Finalteams des AXAWARD 2015 wurden im April 2015 per Mail bekanntgegeben, wobei es \sblit nicht
unter die besten zehn Projekte geschafft hat. Genauere Platzierungen wurden nicht bekannt gegeben.


\subsection{Internet of Things Cup}
\subsubsection{Allgemein}
Der Internet of Things Cup ist ein Open Source Projektwettbewerb für Studierende, Schülerinnen und Schüler,
der den Fokus auf das namensgebende Internet der Dinge legt. sblit eignet sich insofern für den Wettbewerb,
als dass, basierend auf einzelnen Komponenten der Umsetzung, dezentrale Speicher- und Kommunikationsmöglichkeiten
geschaffen werden können, die in Internet of Things und Machine-to-Machine-Anwendungen integriert werden können.

Für das Projekt wurde dem Diplomarbeitsteam kostenlose Hard- und Software zur Verfügung gestellt: ein
Beaglebone-Mikrocomputers, eine Jahreslizenz für Microsofts Azure Datenbank und Conrad Gutscheine.

\subsubsection{Einreichung}
Im April war die Abgabe des Projekt-Zwischenberichts, welcher die erste Abgabe nach
der Einreichung darstellt. Der Bericht setzt sich aus einem Google-Drive-Fragebogen
zusammen, bei dem grundsätzliche Informationen an die Wettbewerbsleitung übergeben
werden, wie zum Beispiel der Projektstatus in Prozentangabe oder auch das Beilegen
von Grafiken des Projekts. Zudem waren noch relevante Links wie zum Beispiel zu der
Projekt-Website oder Social-Media-Seiten anzugeben.

\subsubsection{Ergebnis}
Ergebnisse wurden noch nicht bekanntgegeben, da dieser Wettbewerb noch am Laufen ist.
Die Siegerehrung des IoT Cups findet am 3. Juli 2015 statt.

\subsection{Computer Creative Wettbewerb}
\subsubsection{Allgemein}
Beim Computer Creative Wettbewerb, der von der Oesterreichischen Computer Gesellschaft
organisiert wird, ist die Ausarbeitung eines Projekts, das sich kreativ mit Informatik
auseinandersetzt das Ziel. Die Bewerber dürfen dabei nur jünger als 21 Jahre sein.

Die eingereichten Projekte kommen aus verschiedensten Teilbereichen der Informatik,
sodass sich diverse Arbeiten mit Multimedia, Internet, Robotik, Webseitengestaltung, Spielen oder
Programmieren als Thematik gegenüberstehen und miteinander verglichen werden.

Für die ersten fünf in jeder Kategorie gibt es Geldpreise zu gewinnen.

\subsubsection{Einreichung}
\sblit zählt zur Sekundarstufe II, also Oberstufe von 14 bis 20 Jahren. Zur Teilnahme
musste ein Online-Formular ausgefüllt und eine detaillierte Projektbeschreibung
verfasst und abgegeben werden. Auch bei diesem Wettbewerb wurde der Projektflyer
abgegeben (\siehe{appenix}).

\subsubsection{Ergebnis}
Einsendeschluss der Projekte war bis April, weshalb dieser Wettbewerb noch einige Zeit
andauern wird. Genauere Termine sind noch nicht bekannt.
