
\label{dcl-link}

\subsection{Einleitung}
Um \gls{natt} möglichst problemlos durchführen zu können, erfolgt die
Kommunikation zwischen \glspl{service} über das \gls{udp}.
Da \gls{udp} selbst jedoch keine verlässliche Übertragung und keine
Verschlüsselung bietet, werden die Datenströme zwischen \glspl{service} über
\glspl{link} übertragen.
Diese gewährleisten die verlässliche und geordnete Übertragung der Daten,
außerdem wird der Übertragungskanal verschlüsselt.

\subsection{Sicherheit}
\Glspl{link} werden mit dem \gls{aes} im \gls{gcm} verschlüsselt.
Das garantiert sowohl Vertraulichkeit, Authentizität als auch Integrität der
verschlüsselten Daten.

\subsection{Channels}
\label{dcl-link-channels}
Über einen \gls{link} können mehrere getrennte Datenströme gleichzeitig
übertragen werden. Um diese voneinander abzugrenzen, werden sie in
unterschiedlichen \glspl{channel} übertragen.

Auch die zur Verwaltung des \glslink{link}{Links} notwendigen Nachrichten
werden in einem \gls{channel}, dem \gls{mgmtch}, übertragen.

Der Link selbst definiert deshalb nur eine einzige Message, die Daten, einen
\gls{channelid} für die Zuweisung dieser Daten zu einem \gls{channel} und einen
\gls{dataid} für die Angabe der Reihenfolge dieser Daten innerhalb des
\glslink{channel}{Channels} überträgt.
Siehe dazu \figurelink{dcl-link-bytefield}.
Für \gls{fnumc}, siehe \link{dcl-packetcomponents-flexnum}.

\linkbytefield

Die Messages zur Verwaltung des \glslink{link}{Links} definiert das verwendete
\gls{mcp}.
Für die Initialisierung des \glslink{link}{Links} werden die Messages des
\gls{bmcp} verwendet.
\gls{bmcp} definiert aber auch Messages für den späteren Wechsel des verwendeten
\gls{mcp}.

\subsection{Basic Management Channel Protocol}
\label{dcl-link-bmcp}

\subsubsection{Einleitung}
Das \acrfull{bmcp} ist das standardmäßig verwendete \gls{mcp} für
\glslink{link}{Links}.
Es definiert Messages und Abläufe zum Aufbau der Verbindung, zum Öffnen von
\glspl{channel}, zur erneuten Übertragung von verlorenen Nachrichten und zur
\gls{fcnt}.

\subsubsection{Initialisierung}
Während der Initialisierung eines \glslink{link}{Links} wird entschieden, wie
dieser \gls{link} verschlüsselt wird und wie diese Verschlüsselung initialisiert
wird.
Der Initiator des \glslink{link}{Links} sendet dazu in der \msg{\bmcpconnectreq}
eine Liste an möglichen \glspl{cinitmethod}, aus der der Empfänger eine wählt
und diese in der \msg{\bmcpconnectrep} bestätigt.
Danach werden, je nach \gls{cinitmethod} unterschiedlich, eine Reihe an
\msgpl{\bmcpcryptoinit} gesendet, bis die Verschlüsselung initialisiert ist.
Zur Zeit existiert nur eine \gls{cinitmethod}: \gls{aes}/\gls{gcm} via
\gls{rsa}.
Bei dieser \gls{cinitmethod} werden, nach Austausch der öffentlichen
\gls{rsa}-Schlüssel, zufällig generierte \gls{aes}-Schlüssel mit den
öffentlichen \gls{rsa}-Schlüsseln verschlüsselt, übertragen und angewandt.

Für eine spätere Implementierung ist eine zusätzliche \gls{cinitmethod} geplant,
die die \gls{aes}-Schlüssel über das Diffie-Hellman-Merkle
Schlüsselaustauschverfahren generiert.

\subsubsection{Messageaufbau}
Die Messages des \acrlong{bmcp} sind generell so aufgebaut, dass ein einzelnes
Byte am Beginn der Message deren Typ angibt, anhand dessen die restliche
Nachricht interpretiert wird.

\bmcpbytefield

\subsubsection{Messages}

\subsubsection*{Connect Request}
\label{dcl-bmcp-connectreq}
Die \msg{\bmcpconnectreq} ist die Message zur Verbindungsanfrage und folglich
die erste Message eines \glslink{link}{Links}, die übertragen wird.
Die Message enthält eine Liste an \glspl{cimid} jener \glspl{cinitmethod}, die
der Sender akzeptiert.
Der Empfänger antwortet bei Annahme der Verbindung mit einer
\msg{\bmcpconnectrep}, die einen der angeführten \glspl{cimid} enthält.

\bmcpconnectreqbytefield


\subsubsection*{Connect Reply}
\label{dcl-bmcp-connectrep}
Die \msg{\bmcpconnectrep} wird als Antwort auf eine vorhergehende
\msg{\bmcpconnectreq} gesendet und enthält einen der darin angeführten
\glspl{cimid}.
Dieser bestimmt, wie der \gls{link} verschlüsselt wird und wie diese
Verschlüsselung initialisiert wird.
Zum weiteren Verbindungsaufbau folgt, je nach verwendeter \gls{cinitmethod},
eine Reihe an \msgpl{\bmcpcryptoinit}.

\bmcpconnectrepbytefield


\subsubsection*{Disconnect}
\label{dcl-bmcp-disconnect}
Die \msg{\bmcpdisconnect} dient zur Beendigung des \glslink{link}{Links}.
Der Empfänger bestätigt die Auflösung der Verbindung durch Senden einer
\msg{\bmcpkill}.

\bmcpdisconnectbytefield


\subsubsection*{Kill}
\label{dcl-bmcp-kill}
Die \msg{\bmcpkill} ist die letzte Nachricht eines \glslink{link}{Links}, die
übertragen wird und markiert das Ende der Verbindung.
Die Message wird als Antwort auf eine \msg{\bmcpdisconnect} gesendet, kann aber
bei unerwartetem Beenden des \glslink{link}{Links} durch einen
Verbindungspartner auch alleinstehend gesendet werden.

\bmcpkillbytefield


\subsubsection*{Crypto Init}
\label{dcl-bmcp-cryptoinit}
Die \msg{\bmcpcryptoinit} überträgt Daten zur Initialisierung der verwendeten
Verschlüsselung des \glslink{link}{Links} und wird zu Beginn der Verbindung, je
nach verwendeter \gls{cinitmethod} unterschiedlich oft, übertragen.

\bmcpcryptoinitbytefield


\subsubsection*{Ack}
\label{dcl-bmcp-ack}
Die \msg{\bmcpack} dient zur Bestätigung empfangener Anfragen, wie
beispielsweise einer \msg{\bmcpopenchreq}.
Die Message enthält den \gls{dataid} der bestätigten Anfrage.

\bmcpackbytefield


\subsubsection*{Change Protocol Request}
\label{dcl-bmcp-chprotoreq}
Die \msg{\bmcpchprotoreq} dient zum Wechsel des verwendeten \gls{mcp}.
Die Message enthält einen \gls{protoid} in Form eines Strings, der das neue
\gls{mcp} angibt.
Der Empfänger bestätigt den Wechsel des Protokolls mit einer \msg{\bmcpack},
die den \gls{dataid} dieser \msg{\bmcpchprotoreq} enthält.

\bmcpchprotoreqbytefield


\subsubsection*{Channel Block Status Request}
\label{dcl-bmcp-chblockstatreq}
Die \msg{\bmcpchblockstatreq} dient zur Abfrage der empfangenen \glspl{dataid}
eines oder mehrerer \glslink{channel}{Channels}.
Die Message enthält eine Liste der \glspl{channelid} der \glspl{channel}, die
abgefragt werden sollen.

Der Empfänger antwortet mit einer \msg{\bmcpchblockstatrep}, anhand der der
Sender dieser \msg{\bmcpchblockstatreq} feststellen kann, welche Pakete nicht
am Ziel angekommen sind und erneut übertragen werden müssen.

\bmcpchblockstatreqbytefield


\subsubsection*{Channel Block Status Report}
\label{dcl-bmcp-chblockstatrep}
Die \msg{\bmcpchblockstatrep} wird als Antwort auf eine
\msg{\bmcpchblockstatreq} gesendet und enthält eine Liste an
\glspl{chblockstatrep} für jeden der in der Anfrage angeführten \glspl{channel}.
Der Empfänger kann anhand dieser \glspl{chblockstatrep} erkennen, welche Pakete
nicht übertragen wurden und deshalb erneut gesendet werden müssen.

Ein \gls{chblockstatrep} enthält den \gls{channelid} des
\glslink{channel}{Channels} auf den er sich bezieht,
den niedrigsten und den höchsten \gls{dataid}, die auf diesem \gls{channel}
empfangen wurden, % TODO die empfangen wurden oder der empfangen wurde?
die Anzahl an unterschiedlichen \glspl{dataid}, die insgesamt auf diesem
\gls{channel} empfangen wurden
und je eine Liste an nicht empfangenen, einzelnen \glspl{dataid} sowie an
zusammenhängenden Blöcken aus nicht empfangenen \glspl{dataid} innerhalb des
niedrigsten und höchsten \glslink{dataid}{Data Identifiers} des
\glslink{channel}{Channels}.

\bmcpchblockstatrepbytefield


\subsubsection*{Open Channel Request}
\label{dcl-bmcp-openchreq}
Die \msg{\bmcpopenchreq} dient zur Anfrage über das Öffnen eines neuen
\glslink{channel}{Channels}.
Die Message enthält den \gls{channelid} des neuen \glslink{channel}{Channels}
sowie den \gls{protoid} des Protokolls, über das auf dem neuen \gls{channel}
kommuniziert werden soll.

Nimmt der Empfänger die Anfrage an, wird eine \msg{\bmcpack} gesendet, die den
\gls{dataid} dieser \msg{\bmcpopenchreq} enthält.

\bmcpopenchreqbytefield


\subsubsection*{Throttle}
\label{dcl-bmcp-throttle}
Die \msg{\bmcpthrottle} dient zur Limitierung der Sendeübertragungsrate des
Empfängers und wird für die \gls{fcnt} des \glslink{link}{Links} genutzt.
Die Message enthält die maximal zulässige Senderate in Bytes pro Sekunde.

\bmcpthrottlebytefield
