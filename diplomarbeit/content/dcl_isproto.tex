
\subsection{Zweck}
Das \gls{isproto} wird zur Kommunikation zwischen zwei direkt miteinander verbundenen \glspl{service}
verwendet. Es dient zur Übertragung von Nachrichten und von in \glspl{appch} gesendeten Nutzdaten,
zur Bekanntgabe von \glspl{endp} samt Adressen und beigetretenen \glspl{net}, sowie zum Einstieg in
\glspl{net} des \gls{dcl}.

Dieses Unterkapitel beschreibt das \gls{isproto} in Version 0.

\subsection{Initialisierung}
Die Kommunikation über das \gls{isproto} wird mit der Aushandlung einer Protokollversion begonnen.
Dazu werden, beginnend mit dem Initiator, also dem \gls{service}, der die Verbindung angefordert hat,
so lange \isprotoversion-Messages abwechselnd zwischen den \glspl{service} gesendet, bis die letzten
zwei übertragenen \isprotoversion-Messages die selben Versionsnummern enthalten.

%	\begin{figure*}[ht!]
%	  \centering
%	  \begin{bytefield}{32}
%	    \bitheader{31,24,23,16,15,8,7,0} \\
%	    \begin{rightwordgroup}{Network Header}
%	      \wordbox{1}{Destination} \\
%	      \wordbox{1}{Source} \\
%	      \wordbox{1}{Length} 
%	    \end{rightwordgroup} \\
%	    \begin{rightwordgroup}{Payload}
%	      \wordbox[lrt]{2}{$N$ data words} \\
%	      \skippedwords \\
%	      \wordbox[lrb]{2}{}
%	    \end{rightwordgroup} \\
%	    \begin{rightwordgroup}{Footer}
%	      \wordbox{1}{Checksum}
%	    \end{rightwordgroup}
%	  \end{bytefield}
%	  \caption{\label{fig:mwe_packet}Sample packet}
%	\end{figure*}
%
%	\begin{figure*}
%	  \centering
%	  \begin{bytefield}{32}
%	    \bitheader{31,24,23,16,15,8,7,0} \\
%	    \bitbox{1}{\tiny{M\\E\\M}} & \bitbox{3}{SEL} & \bitbox{23}{} & \bitbox{4}{Mem\\Type} & \bitbox{4}{ID} \\    
%	  \end{bytefield}
%	  \caption{\label{fig:mwe_cmd}Cmd word}
%	\end{figure*}

%	\begin{bytefield}[bitwidth=1.1em]{32}
%	\bitheader{0-31} \\
%	\begin{rightwordgroup}{RTP \\ Header}
%	\bitbox{2}{V=2} & \bitbox{1}{P} & \bitbox{1}{X}
%	& \bitbox{4}{CC} & \bitbox{1}{M} & \bitbox{7}{PT}
%	& \bitbox{16}{sequence number} \\
%	\bitbox{32}{timestamp}
%	\end{rightwordgroup} \\
%	\bitbox{32}{synchronization source (SSRC) identifier} \\
%	\wordbox[tlr]{1}{contributing source (CSRC) identifiers} \\
%	\wordbox[blr]{1}{$\cdots$} \\
%	\begin{rightwordgroup}{RTP \\ Payload}
%	\wordbox[tlr]{3}{MPEG-4 Visual stream (byte aligned)} \\
%	\bitbox[blr]{16}{}
%	& \bitbox{16}{\dots\emph{optional} RTP padding}
%	\end{rightwordgroup}
%	\end{bytefield}


\subsection{Messageaufbau}
Eine Message im \gls{isproto} ist generell so aufgebaut, dass ein FlexNum-Component am Anfang der
Message den Typ angibt, anhand dessen die restliche Nachricht interpretiert wird.

\isgeneralbytefield

%\begin{figure}[tbh]
%\begin{centering}
%
%\begin{bytefield}[bitwidth=3em]{8}
%	\\
%	\bitheader{0-7} \\
%	
%	\wordbox[tlr]{2}{\isprotomsgtype \\ \flexnumfield} \\
%	\skippedwords \\
%	\wordbox[lr]{1}{} \\
%	
%	\wordbox[tlr]{2}{\isprotomsgdata \\ $N$ Bytes} \\
%	\skippedwords \\
%	\wordbox[blr]{1}{}
%	
%	\begin{rightwordgroup}{RTP \\ Header}
%		\bitbox{2}{V=2} & \bitbox{1}{P} & \bitbox{1}{X} & \bitbox{4}{CC} \\
%		\bitbox{1}{M} & \bitbox{7}{PT} \\
%		\wordbox[tlr]{2}{sequence number} \\
%		\wordbox[tlr]{4}{timestamp}
%	\end{rightwordgroup} \\
%	\bitbox{32}{synchronization source (SSRC) identifier} \\
%	\wordbox[tlr]{1}{contributing source (CSRC) identifiers} \\
%	\wordbox[blr]{1}{$\cdots$} \\
%	\begin{rightwordgroup}{RTP \\ Payload}
%		\wordbox[tlr]{3}{MPEG-4 Visual stream (byte aligned)} \\
%		\bitbox[blr]{16}{}
%		& \bitbox{16}{\dots\emph{optional} RTP padding}
%	\end{rightwordgroup}
%\end{bytefield}
%
%\par\end{centering}
%\protect\caption{\gls*{isproto} -- Genereller Messageaufbau}
%\end{figure}

\subsection{Messages}

\subsubsection{Version}
\label{dcl-isproto-version}
Die \isprotoversion-Message enthält eine Versionsnummer in Form eines FlexNum-Components und wird
benutzt, um am Beginn der Verbindung die verwendete Version des \gls{isproto} auszuhandeln.

\isprotoversionbytefield

%\begin{figure}[tbh]
%\begin{centering}
%
%\begin{bytefield}[bitwidth=3em]{8}
%	\\
%	\bitheader{0-7} \\
%	
%	\begin{rightwordgroup}{\isprotomsgtype}
%		\wordbox[tlr]{1}{\code{0}}
%	\end{rightwordgroup} \\
%	
%	\begin{rightwordgroup}{\isprotomsgdata}
%		\wordbox[tlr]{2}{Version ID \\ \flexnumfield} \\
%		\skippedwords \\
%		\wordbox[blr]{1}{}
%	\end{rightwordgroup}
%	
%\end{bytefield}
%
%\par\end{centering}
%\protect\caption{\gls*{isproto} -- \isprotoversion-Message}
%\end{figure}

\subsubsection{LLA Request}
\label{dcl-isproto-llareq}
Die \isprotollareq-Message wird benutzt, um vom Gegenüber eine Liste an \glspl{lla} anzufordern.
Die Message enthält ein Feld für die Maximalanzahl an \glspl{lla}, die in der
\isprotollarep-Message zurückgesendet werden sollen.

\isprotollareqbytefield

\subsubsection{LLA Reply}
\label{dcl-isproto-llarep}
Blah blah blub.

\isprotollarepbytefield

\subsubsection{Trusted Switch}
\label{dcl-isproto-ts}
Blah blah blub.

\isprototsbytefield

\subsubsection{Crypto Challenge Request}
\label{dcl-isproto-ccreq}
Blah blah blub.

\isprotoccreqbytefield


