
\subsection{Zweck}
Das \gls{isproto} wird zur Kommunikation zwischen zwei direkt miteinander verbundenen \glspl{service}
verwendet. Es dient zur Übertragung von Nachrichten und von in \glspl{appch} gesendeten Nutzdaten,
zur Bekanntgabe von \glspl{endp} samt Adressen und beigetretenen \glspl{net}, sowie zum Einstieg in
\glspl{net} des \gls{dcl}.

Dieses Unterkapitel beschreibt das \gls{isproto} in Version 0.

\subsection{Address Slots}
Da \glspl{service} jeweils mehrere Adressen hosten können, ist es notwendig, diese mit dem
\gls{isproto} auch zu unterstützen. Jede Adresse wird hierbei einem eigenen \gls{addrslot} zugeordnet,
um sich in den Messages des Protokolls einfach auf einzelne Adressen beziehen zu können, ohne jedes
mal die ganze Adresse senden zu müssen.
%TODO Übergang zu connection base; Verbindung Addr Slots <-> Net Slots

\subsection{Network Slots}
Eine Adresse kann gleichzeitig in mehreren verschiedenen \glspl{net} vertreten sein. Um diese
\glspl{net} in den Messages des \gls{isproto} einfach referenzieren zu können, ohne jedes mal den
ganzen \gls{ntid} übertragen und parsen zu müssen, wird jedes \gls{net}, dem ein Service mit einer
oder mehreren Adressen beitritt einem \gls{netslot} zugeordnet.

Werden \glspl{netpkt} übertragen, so wird auch das \gls{net} in dem diese geroutet werden in der
\msg{\isprotonp} in Form des entsprechenden \gls{netslot} angeführt, um beim Empfänger die Zuordnung
zu ermöglichen.

\subsection{Connection Base}
Die \gls{connbase} gibt an, welche Messages von einem \gls{service} auf einem \gls{addrslot} eines
\gls{isch} akzeptiert werden. Derzeit sind zwei Werte für die \gls{connbase} implementiert:
\const{stranger} (\code{0}) und \const{trusted} (\code{1}).
Ein neuer \gls{addrslot} wird immer mit der \gls{connbase} \const{stranger} initialisiert, unter
welcher keine mit dem Routing in Verbindung stehenden Messages akzeptiert werden.
Erst ab der \gls{connbase} \const{trusted} werden die \msg{\isprotonjn}, die \msg{\isprotonln},
die \msg{\isprotonp}, die \msg{\isprotoireq}, die \msg{\isprotoacsa} und die \msg{\isprotoacd}
akzeptiert.

Als Voraussetzung für die \gls{connbase} \const{trusted} gilt der Beweise der Adresse auf diesem
\gls{addrslot}. Ein \gls{service} kann den Wechsel auf die \gls{connbase} \const{trusted} für einen
existenten \gls{addrslot} anfordern, indem er eine \msg{\isprotots} an sein Gegenüber sendet.
Der entfernte \gls{service} fordert danach durch Senden einer \msg{}

darüber, dass die \gls{cchlg} erfolgreich durchgeführt wurde, dass die
vom angegebenen \gls{addrslot} referenziertte Adresse des Empfängers somit vom Sender akzeptiert
wurde und dass der Empfänger der \msg{\isprotocbn} jetzt Nachrichten über den Sender routen kann.

\subsection{Initialisierung}
Die Kommunikation über das \gls{isproto} wird mit der Aushandlung einer Protokollversion begonnen.
Dazu werden, beginnend mit dem Initiator, also dem \gls{service}, der die Verbindung angefordert hat,
so lange \msgpl{\isprotoversion} abwechselnd zwischen den \glspl{service} gesendet, bis die letzten
zwei übertragenen \msgpl{\isprotoversion} die selben Versionsnummern enthalten.

\subsection{Messageaufbau}
Eine Message im \gls{isproto} ist generell so aufgebaut, dass ein FlexNum-Component am Anfang der
Message den Typ angibt, anhand dessen die restliche Nachricht interpretiert wird.

\isprotobytefield

\subsection{Messages}

\subsubsection{Version}
\label{dcl-isproto-version}
Die \isprotoversion Message enthält eine Versionsnummer in Form eines FlexNum-Components und wird
benutzt, um am Beginn der Verbindung die verwendete Version des \gls{isproto} auszuhandeln.

\isprotoversionbytefield

\subsubsection{LLA Request}
\label{dcl-isproto-llareq}
Die \msg{\isprotollareq} wird benutzt, um vom Gegenüber eine Liste an \glspl{lla} von anderen
\glspl{service} anzufordern.
Die Message enthält ein Feld für die Maximalanzahl an \glspl{lla}, die in der
\msg{\isprotollarep} zurückgesendet werden sollen.

\isprotollareqbytefield


\subsubsection{LLA Reply}
\label{dcl-isproto-llarep}
Die \msg{\isprotollarep} wird als Antwort auf eine zuvor empfangene \msg{\isprotollareq} gesendet
und enthält \glspl{lla} von anderen \glspl{service}.

\isprotollarepbytefield


\subsubsection{Trusted Switch}
\label{dcl-isproto-ts}
Die \msg{\isprotots} wird benutzt, um vom Gegenüber eine \gls{cchlg} für den als Adresse verwendeten,
angegebenen öffentlichen Schlüssel anzufordern. Dieser Schritt wird benötigt, um dem Gegenüber beweisen
zu können, dass die angegebene Adresse echt ist. Erst wenn die \gls{cchlg} erfolgreich durchgeführt
wurde, kann der Sender der \msg{\isprotots} damit am Routing jener \glspl{net} teilnehmen, denen er
mit der angegebenen Adresse beigetreten ist.
% TODO: "Erst wenn X, kann Y" oder "Erst wenn X kann Y"? (Beistrich)

Mit dieser Nachricht wird gleichzeitig die in Form eines öffentlichen Schlüssel angegebene Adresse
mit dem angegebenen \gls{addrslot} assoziiert.

\isprototsbytefield


\subsubsection{Crypto Challenge Request}
\label{dcl-isproto-ccreq}
Die \msg{\isprotoccreq} wird als Antwort auf eine \msg{\isprotots} gesendet und fordert das Gegenüber
auf, die enthaltenen Binärdaten im Rahmen einer \gls{cchlg} mit dem zur angegebenen Adresse
zugehörigen privaten Schlüssel zu signieren.

\isprotoccreqbytefield


\subsubsection{Crypto Challenge Reply}
\label{dcl-isproto-ccrep}
Die \msg{\isprotoccrep} wird als Antwort auf eine \msg{\isprotoccreq} gesendet und enthält die Daten
der gelösten \gls{cchlg}.

\isprotoccrepbytefield


\subsubsection{Connection Base Notice}
\label{dcl-isproto-cbn}
Die \msg{\isprotocbn} wird als Antwort auf eine \msg{\isprotoccrep} gesendet und benachrichtigt
den Empfänger über die aktuelle \gls{connbase} der vom angegebenen \gls{addrslot} referenzierten
Adresse.

\isprotocbnbytefield


\subsubsection{Network Join Notice}
\label{dcl-isproto-njn}
Die \msg{\isprotonjn} benachrichtigt den Empfänger darüber, dass der sendende \gls{service} mit der
vom angegebenen \gls{addrslot} referenzierten Adresse dem vom angegebenen \gls{ntid} referenzierten
\gls{net} beigetreten ist. Sollte der sendende \gls{service} diesem \gls{net} zuvor noch mit keiner
Adresse beigetreten sein, so wird der Empfänger mit dieser Nachricht außerdem darüber in Kenntnis
gesetzt, dass ab jetzt Nachrichten zum Routing in diesem \gls{net} an den sendenden \gls{service}
übermittelt werden können.

\isprotonjnbytefield

\subsubsection{Network Leave Notice}
\label{dcl-isproto-nln}
Die \msg{\isprotonln} benachrichtigt den Empfänger darüber, dass der sendende \gls{service} mit der
vom angegebenen \gls{addrslot} referenzierten Adresse aus dem vom angegebenen \gls{ntid}
referenzierten \gls{net} austritt. Sollte der sendende \gls{service} aktuell mit keiner anderen
Adresse diesem \gls{net} beigetreten sein, so wird der Empfänger mit dieser Nachricht außerdem
darüber in Kenntnis gesetzt, dass ab jetzt keine Nachrichten mehr zum Routing in diesem \gls{net}
an den Sender übermittelt werden können.

\isprotonlnbytefield


\subsubsection{Integration Request}
\label{dcl-isproto-ireq}
Die \msg{\isprotoireq} fordert den Empfänger auf, den sendenden \gls{service} in die gemeinsamen
\glspl{net} zu integrieren. Konkret bedeut das, dafür zu sorgen, dass der sendende \gls{service}
so mit anderen \glspl{service} verbunden wird, dass das Routing der Adressen des sendenden
\gls{service} möglichst effizient funktionieren kann.

\isprotoireqbytefield


\subsubsection{Integration Connect Request}
\label{dcl-isproto-icreq}
Die \msg{\isprotoicreq} fordert den Empfänger auf, eine Verbindung zur angegebenen \gls{lla}
aufzubauen und wird als Antwort auf eine frühere \msg{\isprotoireq} gesendet. Sie wird benötigt,
um neue Teilnehmer eines \glslink{net}{Netzwerks} mit ihren entsprechenden Nachbarn verbinden zu
können.

\isprotoicreqbytefield


\subsubsection{Network Packet}
\label{dcl-isproto-np}
Die \msg{\isprotonp} enthält ein \gls{netpkt}, das vom Empfänger in dem durch den angegebenen
\gls{netslot} referenzierten \gls{net} weitergeleitet werden soll.

\isprotonpbytefield


\subsubsection{Application Channel Slot Assign}
\label{dcl-isproto-acsa}
Die \msg{\isprotoacsa} benachrichtigt den Empfänger darüber, dass Nachrichten eines \gls{appch} % TODO Application ChannelS?
zwischen den beiden durch die angegebenen \glspl{addrslot} referenzierten Adressen auf dem
angegebenen \gls{appchslot} entgegen genommen werden.

\isprotoacsabytefield


\subsubsection{Application Channel Data}
\label{dcl-isproto-acd}
Die \msg{\isprotoacd} wird dazu benutzt, um Daten des durch den angegebenen \gls{appchslot} % TODO Application ChannelS?
referenzierten \gls{appch} zu übertragen.

\isprotoacdbytefield

