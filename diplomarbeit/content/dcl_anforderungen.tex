
Das \gls{p2pnet} des \gls{dcl} muss eine Reihe von Eigenschaften aufweisen,
um für die Anwendung eingesetzt werden zu können. Diese Eigenschaften decken sich im wesentlichen mit
üblichen Anforderungen an herkömmliche, nicht dezentrale Netzwerke und lauten wie folgt:
\begin{description}
	\descriptionitem{Adressierung}
		Den Teilnehmern müssen eindeutige Adressen zugewiesen werden können, die auf ihre Echtheit überprüfbar sind.


	\descriptionitem{Routing}
		Zwischen Teilnehmern müssen anhand ihrer Adressen Kommunikationskanäle aufgebaut werden können.

\end{description}

Durch den dezentralen Ansatz von \gls{dcl} gestaltet sich die Realisierung dieser Eigenschaften jedoch
anders als das beispielsweise für ein übliches Computernetzwerk oder das Internet der Fall wäre.
So können Adressen nicht von zentralen, dazu bemächtigten Behörden vergeben werden und das Routing
kann nicht von speziellen Teilnehmern des Netzwerks in einer hierarchischen Organisation erfolgen.
