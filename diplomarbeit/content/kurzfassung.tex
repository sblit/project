Diese Diplomarbeit zielt darauf ab, einen Dienst zur Synchronisation von Dateien
über das Internet in der Form freier Software zu schaffen.
Zentrale Stellen wie Server sollen dabei im Sinne der Sicherheit vor Eingriffen
durch Unbefugte nicht eingesetzt werden.
Statt der Zwischenspeicherung von Daten auf Cloudspeicher eines herkömmlichen
Anbieters werden Daten in verschlüsselten Blöcken auf den Geräten anderer Nutzer
zwischengespeichert.

An die Stelle einer klassischen Client-Server-Infrastruktur tritt ein
dezentrales Peer-to-Peer-Netzwerk.
Dadurch und durch die Verteilung von Daten auf Geräte anderer Teilnehmer
auftretende Problemstellungen sollen im Rahmen dieser Diplomarbeit gelöst
werden.
Dazu zählen dezentrale Adressierung, dezentrales Routing, Fairness von
gegenseitiger Speicherfreigabe, Sicherheit und Effizienz von Kommunikation und
Synchronisation sowie Behandlung von Synchronisationskonflikten.
