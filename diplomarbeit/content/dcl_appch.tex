\label{dcl-appch}
\subsection{Einleitung}
Um Anwendungen, deren Kommunikation über den \gls{dcl} geleitet werden soll,
nicht nur das Senden von verbindungslosen Paketen zu ermöglichen, sondern auch
die Übertragung von Daten durch verschlüsselte, performante und verlässliche
Verbindungen, gibt es im \gls{dcl} sogenannte \glspl{appch}.
Zur Zeit sind das direkte Kanäle zwischen jenen zwei \glspl{service}, die mit
den Anwendungen, zwischen denen der \gls{appch} besteht, via \glspl{asconn}
verbunden sind.
Eine Implementierung eines zweiten Typs von \glspl{appch}, bei dem die
\glspl{service}, die die Anwendungen an den \gls{dcl} anbinden, nicht direkt
miteinander verbunden sind, ist für die Zukunft nicht ausgeschlossen.

\subsection{Sicherheit}
Die durch einen \gls{appch} übertragenen Daten werden im \gls{isch} mit der
\msg{\isprotoacd} (\link{dcl-isproto-acd}) zwischen den \glspl{service}
übermittelt.
Der \gls{link}, über den der \gls{isch} übertragen wird, ist dabei
verschlüsselt.
Die Daten des \gls{appch} werden nicht weiter verschlüsselt.
Da \glspl{appch} aber immer zwischen genau zwei \glspl{service} übertragen
werden, ist neben der Verschlüsselung des \glslink{link}{Links} auch keine
weitere Verschlüsselung notwendig.
%Die durch einen \gls{appch} übertragenen Daten werden durch den sogenannten
%\gls{isch} übertragen.
%Der \gls{isch} wird wiederum durch den \gls{link} zwischen zwei \glspl{service}
%übertragen, bei dem es sich um eine direkte, verschlüsselte Verbindung handelt.
%Die Daten des \gls{appch} werden nicht weiter verschlüsselt.
%Da \glspl{appch} in der derzeitigen Implementierung aber immer zwischen genau
%zwei \glspl{service} übertragen werden, ist neben der Verschlüsselung des
%\glslink{link}{Links} auch keine weitere Verschlüsselung notwendig.

Selbstverständlich kann aber auf der Seite der Anwendung eine zusätzliche
Verschlüsselungsschicht umgesetzt werden, sodass ausschließlich die
verschlüsselten Daten über den \gls{appch} übertragen werden.

\subsection{Aufbau}
Für einen \gls{appch} müssen die beiden \glspl{service}, die die Anwendungen an
den \gls{dcl} anbinden, direkt miteinander verbunden sein und über einen
\gls{isch} miteinander kommunizieren.
Beim Aufbau eines \gls{appch} gibt es deshalb zwei Fälle zu unterscheiden:
der Fall, in dem die \glspl{service} bereits miteinander verbunden sind, und der
Fall, in dem die \glspl{service} noch nicht direkt miteinander verbunden sind.

Im zweiten Fall müssen die \glspl{service} zusätzlich zu den Aktionen, die
schon im ersten Fall notwendig sind, noch miteinander verbunden werden, wofür
der Austausch ihrer \glspl{lla} und evtl. die Durchführung von \gls{natt}
notwenig sind.

Um diese Fälle in der Umsetzung nicht weiter behandeln zu müssen, werden die
Anfragen für \glspl{appch} über das \acrfull{crisp} verschickt, welches unter
\labellink{dcl-crisp} auf \pagelink{dcl-crisp} beschrieben wird.

Der \gls{service}, der die Anwendung an den \gls{dcl} anbindet, die den
\gls{appch} angefragt hat, sendet zum Aufbau eine \acrshort{crisp}
\msg{\crispneighreq} (siehe \figurelinkp{dcl-crisp-neighreq-bytefield}) über das
\gls{net}, das die Anwendung in der \msg{\asprotoappchoutreq} über die
\gls{asconn} angegeben hat.
Der \gls{actid} aus der über die \gls{asconn} gestellten Anfrage der Anwendung
erhält das Präfix \code{org.dclayer.applicationchannel/} und wird in die
\msg{\crispneighreq} kopiert.
Außerdem wird die \gls{lla} des lokalen \gls{service} in die Message übernommen,
das Response-Flag auf \code{false} gesetzt und die \gls{ignoredata} eingefügt.
Die \msg{\crispneighreq} wird über das \gls{net} an die Adresse gesendet, die
die Anwendung als Zieladresse des \gls{appch} angegeben hat.

Der entfernte \gls{service}, der die \msg{\crispneighreq} empfängt, leitet die
Anfrage per \msg{\asprotoappchinreq} über die \gls{asconn} an die Anwendung
weiter.
Nimmt die Anwendung den \gls{appch} an, sendet diese als Antwort auf die
\msg{\asprotoappchinreq} eine \msg{\asprotoappchaccept} an den \gls{service}.
Dieser antwortet danach wiederum dem ersten \gls{service} mit einer
\acrshort{crisp} \msg{\crispneighreq}, in der das Response-Flag auf \code{true}
gesetzt wird, wodurch das Feld für die \gls{ignoredata} wegfällt.
In die Message wird die \gls{lla} dieses \gls{service} eingetragen und der
\gls{actid} aus der vorhergehenden \msg{\crispneighreq} kopiert.
Sofern die beiden \glspl{service} nicht bereits miteinander verbunden sind,
sendet der \gls{service} neben der \msg{\crispneighreq} gleichzeitig auch ein
\gls{nattpkt} an die \gls{lla} des ersten \gls{service}, das aus der
\gls{ignoredata} der zuvor empfangenen Message besteht, um ein möglicherweise
notwendiges \gls{natt} durchzuführen.
Sollte kein \gls{natdev} die Durchführung von \gls{natt} erfordern und dieses
\gls{nattpkt} deshalb den anderen \gls{service} erreichen, so wird das Paket
dort ignoriert, da es aus der zuvor übermittelten \gls{ignoredata} besteht.
Für nähere Informationen zu \gls{nat} und \gls{natt} siehe \link{dcl-natt}.

Anschließend sendet der erste \gls{service} eine normale Verbindungsanfrage an
die in der eben empfangenen \msg{\crispneighreq} angeführte \gls{lla}, sofern
die beiden \glspl{service} nicht bereits verbunden sind.
Die Anfrage erreicht aufgrund des zuvor durchgeführten
\glslink{natt}{NAT-Traversals} jedenfalls ihr Ziel und die Verbindung wird
aufgebaut.

Sobald die Verbindung aufgebaut ist und der \gls{isch} bereit ist,
benachrichtigen die \glspl{service} die Anwendungen durch Senden einer
\msg{\asprotoappchconnected} über die \gls{asconn} darüber, dass der \gls{appch}
verbunden wurde und Daten darüber übertragen werden können.

\subsection{CRISP}
\label{dcl-crisp}
Das \acrfull{crisp} ist ein Protokoll, das zur verbindungslosen Kommunikation
zwischen \glspl{service} dient.
Die Messages des Protokolls werden innerhalb des \gls{dcl} durch \glspl{net}
geroutet.
\acrshort{crisp} kann dabei unabhängig vom verwendeten \gls{net} genutzt werden.

Prinzipiell enthalten \acrshort{crisp}-Messages einen \gls{fnumc}, der die
Revisionsnummer des Protokolls angibt und einen weitere \gls{fnumc}, der den
Typ der Message angibt und anhand dessen der Rest der Message interpretiert
wird.

\crispbytefield

\label{dcl-crisp-neighreq}
Derzeit ist nur eine \acrshort{crisp}-Message definiert: die
\msg{\crispneighreq}.

Die \msg{\crispneighreq} dient zur Anforderung einer direkten Verbindung
zwischen zwei \glspl{service}, die derzeit nur verbindungslos über ein \gls{net}
kommunizieren können. Der primäre Verwendungszweck liegt derzeit im Aufbau von
\glspl{appch}, die Message kann aber auch zur \gls{netint} verwendet werden.

In der \msg{\crispneighreq} wird ein \gls{actid} übertragen, der den Grund für
die Verbindungsanfrage enthält.
Außerdem enthält die Message die \gls{lla} des Senders, ein Response-Flag,
das angibt, ob es sich bei der Nachricht um eine Anfrage oder um eine Antwort
handelt und \gls{ignoredata}, die für \glspl{nattpkt} verwendet werden können.

\crispneighreqbytefield


%TODO Action Identifiers
%TODO Ignore Data
%TODO CRISP?
