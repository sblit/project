
\subsection{Notwendigkeit}
Um im \gls{dcl} Synchronisationgruppen zu bilden, ist es nötig, die Teilnehmer dieser Gruppen als solche
erkennen zu können. Das erfordert wiederum die permanente und eindeutige Adressierung dieser Teilnehmer.
Da sich die öffentlichen IP-Adressen der meisten privaten Internetanschlüsse und somit des Großteils der
Zielgruppe von \sblit periodisch ändern, eignen sich diese jedoch nicht als Adressen im \gls{dcl}. Es
wird also ein anderes Adresskonzept benötigt, mit dem ohne höhere Behörde allen Teilnehmern eindeutige
Adressen zugewiesen werden können, die auf ihre Echtheit überprüfbar sind.

Es ist nicht ausreichend, die Teilnehmer ihre Adressen willkürlich selbst bestimmen zu lassen und eine
Funktion zu implementieren, die überprüft, ob eine neu generierte Adresse im Netzwerk schon existiert,
da dieser Ansatz keinerlei Sicherheit vor einer absichtlichen Übernahme der Adresse eines anderen
Teilnehmers durch einen Angreifer bietet.

% TODO schwierigkeit: keine zentrale Stelle, Eindeutigkeit

\subsection{Adressierung mit \gls{rsa}}
\glslink{aenc}{Asymmetrische Verschlüsselungsverfahren} wie \gls{rsa} eignen sich durch ihre Eigenschaften
ausgezeichnet für die Adressierung innerhalb eines Netzwerks, in dem alle Teilnehmer die gleichen
Berechtigungen haben und in dem keine höhere Instanz existiert, die Adressen vergeben und diese
verifizieren kann.
\tags{key-info, schlüsselpaar, asymmetrisch}

Bei asymmetrischen Verschlüsselungsverfahren kommen sogenannte Schlüsselpaare, bestehend aus zwei
Schlüsseln, zum Einsatz. Die Besonderheit liegt darin, dass eine Nachricht, die mit einem Schlüssel
aus dem Schlüsselpaar verschlüsselt wurde, bei asymmetrischen Verfahren im Gegensatz zu symmetrischen
Verfahren nicht mit dem selben Schlüssel auch wieder entschlüsselt werden kann, sondern ausschließlich
mit dem anderen Schlüssel des Schlüsselpaars.
Gleichzeitig kann aus einem Schlüssel eines Schlüsselpaars der dazugehörige andere Schlüssel des
Schlüsselpaars nicht in absehbarer Zeit berechnet werden.

Dadurch wird es möglich, ein Adressierungssystem umzusetzen, das die Anforderungen im Bezug auf
Überprüfbarkeit der Adressen erfüllt. Dazu wird einer der beiden Schlüssel aus dem Schlüsselpaar
als Adresse angenommen und somit bewusst veröffentlicht, während der andere Schlüssel aus dem
Schlüsselpaar geheim gehalten wird.
Der Schlüssel aus dem Schlüsselpaar, der veröffentlicht wird, wird auch \emph{Öffentlicher Schlüssel}
oder \emph{Public Key} genannt, der weiterhin geheim gehaltene Schlüssel \emph{Privater Schlüssel} oder
\emph{Private Key}.

Öffentliche Schlüssel als Adressen haben den Vorteil, dass sie ohne höhere Behörde oder zentrale Stelle
auf ihre Echtheit überprüft werden können und somit fälschungssicher sind. Ein Mechanismus zur
Überprüfung solch einer Adresse wird im nächsten Abschnitt beschrieben.

\subsection{Überprüfung von Adressen}
Dadurch, dass eine mit einem öffentlichen Schlüssel verschlüsselte Nachricht nicht mit wieder mit dem
öffentlichen Schlüssel entschlüsselt werden kann, sondern ausschließlich mit dem dazugehörigen privaten
Schlüssel, kann der Besitz des gesamten Schlüsselpaars bewiesen werden, ohne mehr als den öffentlichen
Schlüssel preisgeben zu müssen: Eine beliebige Folge von Daten wird vom überprüfenden Teilnehmer
generiert, mit der Adresse, also dem öffentlichen Schlüssel des zu überprüfenden Teilnehmers
verschlüsselt und anschließend an diesen übermittelt. Dort wird die empfangene Nachricht vom zu
überprüfenden Teilnehmer wieder entschlüsselt und zurück an den überprüfenden Teilnehmer gesendet.
Decken sich die ursprünglich vom überprüfenden Teilnehmer generierten Daten mit denen, die vom zu
überprüfenden Teilnehmer entschlüsselt wurden, ist der Besitz des gesamten Schlüsselpaars, und nicht
lediglich des öffentlichen Schlüssels, bewiesen.

%TODO konkretes Beispiel

%TODO Erklärung der Challenge mit Nikolas mergen

\subsection{Eindeutigkeit von Adressen}
Die Eindeutigkeit der generierten \gls{rsa}-Schlüsselpaare und somit der Adressen kann zwar nicht
garantiert werden, eine Kollision ist jedoch aufgrund der Länge der verwendeten Schlüssel und der
Anzahl der dadurch möglichen Schlüsselpaare dermaßen unwahrscheinlich, dass davon ausgegangen
werden kann, dass eine Kollision praktisch nicht auftreten wird. \cite{crypto.stackexchange.com/a/2559:rsa-key-collision}

