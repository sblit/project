
\label{dcl-natt}

\subsection{Einleitung}
Unter \gls{nat} versteht man das Übersetzen von Netzwerkadressen.
\Gls{nat} kommt bei praktisch allen Routern privater Internetanschlüsse zum
Einsatz und ermöglicht es dort, mehrere Hosts über eine einzige öffentliche
IP-Adresse an das Internet anzubinden.

Realisiert wird das durch Übersetzen der privaten Quell-IP-Adressen zur
öffentlichen IP-Adresse bei Paketen, die das lokale Netzwerk verlassen, und
Rückübersetzen der öffentlichen Ziel-IP-Adresse zu den privaten IP-Adressen
bei Paketen, die aus dem Internet in das lokale Netzwerk eintreten.

Bei ankommenden Paketen wird dabei anhand des Ziel-Ports entschieden, an welche
lokale IP-Adresse das Paket weitergeleitet wird.
Um es mehreren lokalen Hosts zu ermöglichen, vom gleichen Port aus Pakete zu
senden, werden die Quell-Ports von Paketen, die das lokale Netzwerk verlassen,
zusätzlich zur Quell-IP-Adresse, auch übersetzt.
Beim Eintritt von Antwortpaketen in das lokale Netzwerk werden die Ziel-Ports
entsprechend rückübersetzt. Man spricht dann von der sogenannten Port Address
Translation, die eine Unterkategorie der \acrlong{nat} bildet.

Im \gls{natdev} werden die aktuellen Übersetzungen in einer Datenstruktur
gespeichert, die lokale IP-Adressen und Ports auf globale IP-Adressen und Ports
abbildet.
Dabei werden Datenreihen im folgenden Format gespeichert:

\begin{equation*}
    (\text{ lokale IP-Adresse}, \text{ lokaler Port}, \text{ globale IP-Adresse}, \text{ globaler Port })
\end{equation*}

Verlässt beispielsweise ein Paket das lokale Netzwerk, das von der lokalen
IP-Adresse \code{10.0.0.1} und vom Port \code{28785} aus gesendet wurde,
könnte die dafür von einem \gls{natdev} mit der öffentlichen IP-Adresse
\code{93.184.216.34} angelegte Datenreihe wie folgt aussehen:

\begin{equation*}
    (\texttt{ 10.0.0.1}, \texttt{ 28785}, \texttt{ 93.184.216.34}, \texttt{ 28785 })
\end{equation*}

An dieser Stelle sei angemerkt, dass der Quellport des Pakets in diesem Beispiel
nicht übersetzt wird.
Ist der Host mit der IP-Adresse \code{10.0.0.1} der einzige im lokalen Netzwerk,
der von Port \code{28785} aus sendet, muss eine Übersetzung des Ports auch nicht
zwingend stattfinden.
Nimmt man jedoch an, dass sich ein zweiter Host im lokalen Netzwerk befindet,
der beispielsweise die IP-Adresse \code{10.0.0.2} besitzt und ebenfalls von Port
\code{28785} aus sendet, so würde eine weitere Datenreihe angelegt, die wie
folgt aussieht:

\begin{equation*}
    (\texttt{ 10.0.0.2}, \texttt{ 28785}, \texttt{ 93.184.216.34}, \texttt{ 28786 })
\end{equation*}

Für Pakete des Hosts mit der IP-Adresse \code{10.0.0.2}, die das lokale Netzwerk
verlassen, wird also nicht nur die Quell-IP-Adresse übersetzt, sondern auch der
Quellport von \code{28785} auf \code{28786}. Dieser Schritt ist notwendig, da
sonst bei Paketen, die aus dem Internet in das lokale Netzwerk eintreten, nicht
unterschieden werden kann, ob diese an den Host mit der IP-Adresse
\code{10.0.0.1} oder an den Host mit der IP-Adresse \code{10.0.0.2}
weitergeleitet werden sollen. Die Übersetzung des Quellports ermöglicht es,
das lokale Ziel von Antwortpaketen anhand von deren Zielport zu bestimmen.

%TODO erwähnen, dass öffentliche IP+Port bei unterschiedlichen Zielen gleich
%bleibt?

\subsection{Probleme für Peer-to-Peer-Verbindungen}
Verbindungen zwischen zwei über \glspl{natdev} angebundenen Hosts werden durch
\acrlong{nat} erschwert, da die Pakete zur Initialisierung der Verbindung
aufgrund fehlender Übersetzungsinformationen in den
\glslink{natdev}{NAT-Geräten} nicht an den Zielhost weitergeleitet werden
können.

Um trotzdem Peer-to-Peer-Verbindungen zwischen solchen Hosts aufbauen zu können,
müssen Verfahren zum \gls{natt} angewandt werden.
Hierfür existieren unterschiedliche Protokolle zur Einrichtung einer
Portweiterleitung auf dem lokalen \gls{natdev}, welche eine Verbindung von außen
ermöglicht.

Die derzeitige Implementierung des \gls{service} wendet das sogenannte
\gls{nathp} an, für das kein spezielles Netzwerkprotokoll notwendig ist.

\subsection{NAT Hole Punching}
Bei \gls{nathp} wird für eine neue Verbindung ein Kommunikationskanal zwischen
den Verbindungspartnern benötigt, der schon vor Aufbau der Verbindung existieren
muss.
Diesen Kommunikationskanal kann ein dritter Host bereitstellen, zu dem beide
Verbindungspartner bereits verbunden sind, oder, wie im Fall von \gls{dcl},
ein \glslink{net}{DCL-Netzwerk}, über das beide Kommunikationspartner bereits
kommunizieren können.
In beiden Fällen muss bereits eine Verbindung zu einem Host außerhalb des
lokalen Netzwerks bestehen.

Einer der beiden Hosts, die direkt miteinander verbunden werden sollen, im
Folgenden Host A,
ermittelt seine öffentliche IP-Adresse und seinen öffentlichen Port mithilfe
eines dritten Hosts, zu dem bereits eine Verbindung besteht.
Diese Information wird über den bestehenden Kommunikationskanal an den Host
übermittelt, zu dem eine Verbindung aufgebaut werden soll, im Folgenden Host B.
Host B sendet an die eben erhaltene Adresse von Host A ein Paket, das das lokale
Netzwerk zwar verlässt, am \gls{natdev} von Host A jedoch verworfen wird.
Der Grund dafür liegt darin, dass das Ziel, an das es im lokalen Netzwerk von
Host A weitergeleitet werden soll, vom dortigen \gls{natdev} nicht
bestimmt werden kann, da kein passender Eintrag in der Datenstruktur der
Übersetzungen existiert.
Das \gls{natdev} von Host B erstellt für dieses Paket jedoch trotzdem
einen Übersetzungseintrag für die Weiterleitung von späteren Antwortpaketen in
seiner Datenstruktur, da nicht erkannt werden kann, dass das Paket sein Ziel
nicht erreichen wird.

Gleichzeitig ermittelt Host B ebenfalls seine öffentliche Adresse über einen
dritten Host, zu dem bereits eine Verbindung besteht.
Diese sendet er zusätzlich zu dem soeben in der Übertragung gescheiterten Paket
über den bestehenden Kommunikationskanal an Host A.

Host A sendet, sobald er die öffentliche Adresse des Gegenübers empfangen hat,
ein normales Paket zur Verbindungsinitialisierung an Host B.
Dieses Paket verlässt das lokale Netzwerk, das lokale \gls{natdev} speichert
einen Übersetzungseintrag in seiner Datenstruktur und das Paket erreicht das
\gls{natdev} von Host B.
Dort wird jener Übersetzungseintrag gefunden, der durch das in der Übertragung
bewusst gescheiterte Paket von Host B and Host A angelegt wurde.
Anhand dieses Eintrags kann das Paket zur Verbindungsinitialisierung, das von
Host A and Host B gesendet wurde, im lokalen Netzwerk weitergeleitet werden und
die Hosts können ab diesem Zeitpunkt normal miteinander kommunizieren.

\subsection{Einschränkungen}
Um \gls{nathp} durchführen zu können, muss jedenfalls bereits mindestens eine
Verbindung zu einem Host außerhalb des lokalen Netzwerks existieren.
Die erste Verbindung, die ein über ein \gls{natdev} angebundenes Gerät aufbaut,
muss deshalb jedenfalls zu einem Host erfolgen, der selbst nicht über ein
\gls{natdev} angebunden ist und somit Verbindungen problemlos annehmen kann.
Im Fall von \gls{dcl} sind das \glspl{service}, die auf Hosts laufen, die dieses
Kriterium erfüllen und zusätzlich über eine statische IP-Adresse verfügen.
In der Praxis handelt es sich bei diesen Hosts um Server.
Um einen möglichst problemlosen Einstieg in den \gls{dcl} gewährleisten zu
können, ist es nötig, auf eine der Anzahl an Teilnehmern im \gls{dcl}
entsprechend große Menge an \glspl{service} zurückgreifen zu können, die auf
solchen Hosts laufen.
