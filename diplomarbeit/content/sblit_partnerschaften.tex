\subsection{Allgemein}
Um nicht immer mindestens zwei Geräte eingeschalten haben zu müssen, werden die Daten nicht nur auf den eigenen Geräten gespeichert, sondern auch auf sogenannte Partnergeräte. Dabei speichert man auf 10 Geräten, um die Zeit, in der keine Geräte verfügbar sind, möglichst gering ist.

\subsection{Partnergeräte} \label{Partnergerät}
Ein Partnergerät zeichnet sich dadurch aus, dass entweder meine Daten auf dem Partnergerät oder dessen Daten am eigenen Gerät zwischengespeichert werden. Um den benötigten Speicherplatz möglichst gering zu halten, synchronisiert \sblit nur Delta-Daten. Unter Delta-Daten kann man sich die Änderungen an einer Datei vorstellen. Das heißt, wenn sich eine Datei ändert wird diese so lange auf den Partnergeräten gespeichert, bis diese auf allen Geräten verfügbar ist.

\subsection{Dauer einer Partnerschaft}
Da manche Geräte von Zeit zu Zeit mehr oder weniger genutzt werden, können Partnerschaften bei Inaktivität eines Nutzers gekündigt werden. Dazu löscht man einfach das Partnergerät aus der Liste. Beim Löschen kommt das Partnergerät auf eine Blacklist. Bei der nächsten Anfrage des Partnergeräts wird die Verbindung nicht mehr zugelassen und das Gerät merkt, dass die Partnerschaft nicht mehr gewünscht ist. Danach wird die Partnerschaft auch vom Partnergerät gelöscht.

\subsection{Wunschpartnerschaften}
Kennen sich zwei \sblit-User und haben einen ähnlichen Verwendungsrythmus, können diese natürlich eine Partnerschaft miteinander schließen. Dabei müssen die beiden nur ihre Adressen austauschen und zur Liste der Partnergeräte hinzufügen. Im weiteren Verlauf verhalten sich Wunschpartnerschaften nicht anders als andere Partnerschaften. Diese können, wie auch andere Partnerschaften, bei Bedarf gelöscht werden.

\subsection{Beispiel}
Susanne und Wilfried benötigen jeweils einen Gigabyte Synchronisationsspeicher. Susanne wählt eines ihrer Geräte aus, auf dem sie mehr als ein Gigabyte Speicher frei und genug Bandbreite zur Verfügung hat (im Folgenden Gerät A(S)). Wilfried sucht sich ebenfalls ein Gerät aus, das die Bedingungen erfüllt (im Folgenden Gerät A(W)). Nun tauschen die Geräte A(S) und A(W) ihre Adressen aus und tragen sie beim jeweils anderen in die Liste der Partnergeräte ein. Sobald beide Geräte gleichzeitig online sind, werden alle Geräte von Susanne auf Gerät A(W) und umgekehrt alle Geräte von Wilfried auf Gerät A(S) gespeichert. Weiters wird die Adresse von A(W) auf Susanne's Geräte und die Adresse von A(S) auf Wilfried's Geräte gespeichert. Ab diesem Zeitpunkt werden alle Änderungen von Susanne's Dateien auf Gerät A(W) gesichert, bis sie auf alle Geräte von Susanne verteilt wurden. Sobald Susanne's Geräte die Änderung erhalten, wird Gerät A(W) wieder aufgefordert, die Daten zu löschen. Das gleiche gilt natürlich auch für Wilfried und das Gerät A(S). 