Um nicht immer mindestens zwei Geräte eingeschalten haben zu müssen, werden die Daten nicht nur auf den eigenen Geräten gespeichert, sondern auch auf sogenannte Partnergeräte. Diese sind Geräte von fremden Personen. \\
Dazu ein Beispiel: Susanne und Wilfried benötigen jeweils einen Gigabyte Synchronisationsspeicher. Susanne wählt eines ihrer Geräte aus, auf dem sie mehr als einen Gigabyte Speicher frei und genug Bandbreite zur Verfügung hat (im Folgenden Gerät A(S)). Wilfried sucht sich ebenfalls ein Gerät aus, das die Bedingungen erfüllt (im Folgenden Gerät A(W)). Nun tauschen die Geräte A(S) und A(W) ihre Adressen aus und tragen sie beim jeweils anderen in die Liste der Partnergeräte ein \referenz{freceivers}.