Um nicht immer mindestens zwei Geräte eingeschalten haben zu müssen, werden die Daten nicht nur auf den eigenen Geräten gespeichert, sondern auch auf sogenannte Partnergeräte. Diese sind Geräte von fremden Personen, auf denen \sblit installiert ist. \\
Dazu ein Beispiel: Susanne und Wilfried benötigen jeweils einen Gigabyte Synchronisationsspeicher. Susanne wählt eines ihrer Geräte aus, auf dem sie mehr als einen Gigabyte Speicher frei und genug Bandbreite zur Verfügung hat (im Folgenden Gerät A(S)). Wilfried sucht sich ebenfalls ein Gerät aus, das die Bedingungen erfüllt (im Folgenden Gerät A(W)). Nun tauschen die Geräte A(S) und A(W) ihre Adressen aus und tragen sie beim jeweils anderen in die Liste der Partnergeräte ein. Sobald beide Geräte gleichzeitig online sind, werden alle Geräte von Susanne auf Gerät A(W) und umgekehrt alle Geräte von Wilfried auf Gerät A(S) gespeichert. Weiters wird die Adresse von A(W) auf alle Geräte von Susanne und die Adresse von A(S) auf Wilfried's Geräte gespeichert. Ab diesem Zeitpunkt werden alle Änderungen von Susanne's Dateien auf Gerät A(W) gesichert, bis sie auf alle Geräte von Susanne verteilt wurden. Sobald Susanne's Geräte die Änderung erhalten, wird Gerät A(W) wieder aufgefordert, die Daten zu löschen. Das gleiche gilt natürlich auch für Wilfried und das Gerät A(S). 