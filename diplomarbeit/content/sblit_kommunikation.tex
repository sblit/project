\subsection{Allgemein}
Um zu verhindern, dass Daten mitgelesen werden, verwendet \sblit den sicheren \nameref{Applicationchannel} (siehe Seite \pageref{Applicationchannel}) der Kommunikationsschicht.\\
Dabei unterscheidet man grundsätzlich zwischen zwei Typen von Nachrichten: Nachrichten, die an Partnergeräte \referenz{Partnergerät} geschickt und Nachrichten, die an die eigenen Geräte geschickt werden. Weiters gibt es Nachrichten, die bei beiden Geräten gleich sind.

\subsection{Nachrichten an alle Geräte}
\subsubsection{Allgemein}
Wenn ein fremdes Gerät eine Verbindung aufbauen will, muss auf jeden Fall überprüft werden, ob das Gerät auch das ist, welches es behauptet zu sein. Daher werden folgende Nachrichten versendet, bevor die restliche Kommunikation stattfindet:
\begin{itemize}
	\item \gls{authreq}s
	\item \gls{authres}s
\end{itemize}

\subsubsection{\gls{authreq}}
Angenommen zwei Geräte eines Besitzers wollen Daten austauschen. Wie können sich diese sicher sein, dass es sich auch um das richtige Gerät handelt? Hier kommt die Authentifizierung ins Spiel. \gls{authreq}s dienen zur Sicherstellung der Authentizität des Gerätes (im Folgenden Gerät A), mit dem ein anderes Gerät (im Folgenden Gerät B) eine Verbindung aufbaut. Dazu schickt das Gerät B zufällige Daten an Gerät A mit der Aufforderung, diese zu verschlüsseln. 
\messagestart
	\bitheader{0-7} \\
	
	\begin{rightwordgroup}{\isprotomsgtype}
		\wordbox[tlr]{1}{0}
	\end{rightwordgroup} \\
	
	\begin{rightwordgroup}{\isprotomsgdata}
		\wordbox[tblr]{4}{Zufallsdaten, 64 Bytes} 
	\end{rightwordgroup}
	
\messageend{authreq}
%TODO Umformulieren
Die Gesamtlänge der Daten, die mit RSA-2048 verschlüsselt werden, darf maximal 128 Byte lang sein. Um Gerät A 64 Byte zur Verfügung zu stellen, beträgt die Länge der von Gerät B gesendeten Daten 64 Byte. Diese 64 Byte werden von Gerät A benötigt, um zu verhindern, dass Gerät B sich gewünschte Werte von Gerät A verschlüsseln lässt.

\subsubsection{\gls{authres}}
Bevor die Daten wieder zurückgeschickt werden, müssen diese von Gerät B verschlüsselt werden. Dies geschieht mit dem Private-Key des Gerätes B. Dabei wird vorher noch ein zufälliger Wert an die empfangenen Daten angefügt, um das unter Authentifizierungsanfragen beschriebene Problem zu lösen.
\messagestart
	\bitheader{0-7} \\
	
	\begin{rightwordgroup}{\isprotomsgtype}
		\wordbox[tlr]{1}{1}
	\end{rightwordgroup} \\
	
	\begin{rightwordgroup}{\isprotomsgdata}
		\wordbox[tblr]{4}{Verschlüsselte Daten, 128 Bytes} 
	\end{rightwordgroup}
	
\messageend{authres}
Wird dieses Paket nun von Gerät B empfangen, kann der Inhalt mit dem Public-Key des Gerätes A, also dessen Adresse, entschlüsselt werden. Von den erhaltenen Daten werden die ersten 64 Byte mit den ursprünglich gesendeten 64 Byte verglichen. Stimmen die beiden Werte überein, kann das Gerät seine Authentizität beweisen. Stimmen diese jedoch nicht überein, handelt es sich um einen Betrüger, der offensichtlich den richtigen Private-Key zur von ihm angegebenen Adresse nicht kennt.
		
\subsection{Nachrichten an eigene Geräte}
\subsubsection{Allgemein}
\sblitg verwendet zur Kommunikation zwischen den authentifizierten eigenen Geräten vier verschiedene Nachrichten:
\begin{itemize}
	\item \gls{filereq}s
	\item \gls{fileres}s
	\item \gls{filemsg}s
	\item \gls{filedel}s
	\item \gls{refdev}s
\end{itemize}

\subsubsection{\gls{filereq}} \label{Dateianfrage}
Bevor eine Datei an ein anderes Gerät (im Folgenden Gerät B) verschickt wird, schickt das Gerät, das die Datei besitzt (im Folgenden Gerät A), eine Dateianfrage an das Gerät B. Dies hat 2 Gründe: Erstens muss eruiert werden, ob die Datei überhaupt von Gerät B benötigt wird, oder ob besagtes Gerät schon diese Datei besitzt. Zweitens besteht die Möglichkeit eines Konfliktes \referenz{Konflikt}. 
\messagestart
	\bitheader{0-7} \\
	
	\begin{rightwordgroup}{\isprotomsgtype}
		\wordbox[tlr]{1}{2}
	\end{rightwordgroup} \\
	
	\begin{rightwordgroup}{\isprotomsgdata}
		\wordbox[tlr]{2}{Dateipfad, variable Länge} \\
		\skippedwords \\
		\wordbox[lr]{1}{} \\
		\wordbox[tlr]{2}{Versionsverlauf, variable Länge} \\
		\skippedwords \\
		\wordbox[blr]{1}{}
	\end{rightwordgroup}
\messageend{filereq}

\begin{description} 
	\descriptionitem{Dateipfad}
		Hierbei wird der zu \sblit's Hauptordner relative Dateipfad mitgeschickt. Der absolute Dateipfad wird aus dem Grund nicht mitgeschickt, da der Ort des \sblit-Ordners nicht auf allen Geräten gleich sein muss. Befindet sich der Ordner auf Gerät A beispielsweise unter C:/Users/Susanne/ kann sich der Ordner auf Gerät B auch unter /home/susanne/dateien/ befinden.
	\descriptionitem{Versionsverlauf}
		Der Versionsverlauf beinhaltet alle Hashes einer Datei seit der letzten komplett synchronisierten Version. Das heißt, dass auf jedem Gerät aktuell entweder diese oder eine neuere Version gespeichert ist.
\end{description}

		
\subsubsection{\gls{fileres}} \label{fileres}

\messagestart
	\bitheader{0-7} \\
	
	\begin{rightwordgroup}{\isprotomsgtype}
		\wordbox[tlr]{1}{3}
	\end{rightwordgroup} \\
	
	\begin{rightwordgroup}{\isprotomsgdata}
		\wordbox[tlr]{1}{Need-Flag, 1 Byte} \\
		\wordbox[tlr]{2}{Dateipfad, variable Länge} \\
		\skippedwords \\
		\wordbox[lr]{1}{} \\
		\wordbox[tlr]{2}{Hash, variable Länge} \\
		\skippedwords \\
		\wordbox[blr]{1}{}
	\end{rightwordgroup}
\messageend{fileres}

\begin{description}
	\descriptionitem{Need-Flag}
		Dieses Feld enthält einen Hexadezimalwert, der darüber Auskunft gibt, ob die Datei benötigt wird oder nicht. Steht in diesem Feld der Hexadezimalwert 0x00, wird die Datei nicht benötigt. Steht hier hingegen der Hexadezimalwert 0x01, wird die Datei benötigt. Dieses Byte hilft den Datenverkehr zu reduzieren. So muss nicht eine ganze Datei verschickt werden muss, obwohl diese gar nicht gebraucht wird.
	\descriptionitem{Dateipfad}
		Wie auch bei der Dateianfrage wird in der Antwort auf die Dateianfrage der zu \sblit's Hauptordner relative Pfad mitgeschickt.
	\descriptionitem{Hash}
		Hier wird noch einmal der letzte Hash des Versionsverlaufs der Dateianfrage verschickt, um sicherzustellen, dass die Datei in der Zwischenzeit nicht geändert wurde. 
\end{description}
Nach Empfang der Dateianfrage wird zunächst geprüft, ob die Datei vorhanden und in der aktuellst ist.  Ist die lokale Datei nicht aktuell, wird das Need-Flag, auf den Hexadezimalwert 0x01 gesetzt. Ist die Datei auf dem aktuellsten Stand, wird besagtes Flag auf den Hexadezimalwert 0x00 gesetzt. Außerdem wird überprüft, ob ein Konflikt aufgetreten ist \referenz{Konflikterkennung}.\\
Nach Empfang der Antwort wird zunächst überprüft, ob der Hashwert der Datei mit dem erhaltenen Pfad übereinstimmt. Stimmt der Hashwert in der Anfrage nicht mit dem aktuellen Hashwert überein, wird die Antwort verworfen. Dies kann beispielsweise passieren, wenn in der Zwischenzeit eine neuere Version der Datei erzeugt wurde. Stimmt dieser jedoch überein, kann die Datei nun im nächsten Schritt verschickt werden.
		
\subsubsection{Die eigentliche Übertragung der Datei}
\messagestart
	\bitheader{0-7} \\
	
	\begin{rightwordgroup}{\isprotomsgtype}
		\wordbox[tlr]{1}{5}
	\end{rightwordgroup} \\
	
	\begin{rightwordgroup}{\isprotomsgdata}
		\wordbox[tlr]{2}{Dateiinhalt, variable Länge} \\
		\skippedwords \\
		\wordbox[lr]{1}{} \\
		\wordbox[tlr]{2}{Dateipfad, variable Länge} \\
		\skippedwords \\
		\wordbox[lr]{1}{} \\
		\wordbox[tlr]{2}{Versionsverlauf, variable Länge} \\
		\skippedwords \\
		\wordbox[lr]{1}{} \\
		\wordbox[tlr]{2}{Geräte mit der aktuellen Verision, variable Länge} \\
		\skippedwords \\
		\wordbox[blr]{1}{}
	\end{rightwordgroup}
	
\messageend{filemsg}

\begin{description}
	\descriptionitem{Dateiinhalt}
		Der Dateiinhalt wird als binäres \code{Data}-Objekt verschickt.
	\descriptionitem{Geräte mit der aktuellen Version}
		Hier stehen die Adressen aller Geräte, die die neuste Version schon haben. Ist die Version auf allen Geräten aktuell, kann diese von den Partnergeräten gelöscht werden. Daher wird diese Liste an Geräten immer mit der Datei mitgeschickt. Außerdem können somit unnötige Anfragen an Geräte, die die Datei schon besitzen verhindert werden.
	\descriptionitem{Dateipfad}
		Der Dateipfad wird benötigt, damit die Datei auf dem zu synchronisierenden Gerät weiß, an welchen Ort die Datei gespeichert werden soll. Dies ist der gleiche relative Ort, wie auf dem Gerät, das die Anfrage geschickt hat.
	\descriptionitem{Versionsverlauf}
		Der Versionsverlauf wird mitgeschickt, um diesen auf allen Geräten zu vereinheitlichen. Das verhindert das auftreten von Konflikten, welche gar keine sind. Weiters dient der aktuellste Hash dazu,  sicherzustellen, dass die Datei, die versendet wurde, auch so ankommt, wie sie versendet wurde. Verhasht man die neue Datei, muss das Ergebnis mit dem aktuellsten mitgeschickten Hash übereinstimmen. Andernfalls muss die Datei neu gesendet werden.
\end{description}
		
\subsubsection{\gls{filedel}}
\messagestart
	\bitheader{0-7} \\
	
	\begin{rightwordgroup}{\isprotomsgtype}
		\wordbox[tlr]{1}{5}
	\end{rightwordgroup} \\
	
	\begin{rightwordgroup}{\isprotomsgdata}
		\wordbox[tlr]{2}{Dateipfad, variable Länge} \\
		\skippedwords \\
		\wordbox[blr]{1}{} 
	\end{rightwordgroup}
	
\messageend{filedel}
Eine Löschanfrage beinhaltet den Pfad der zu löschenden Datei. Nach Empfang der Löschanfrage wird die Datei gelöscht.

\subsubsection{\gls{refdev}}
\messagestart
	\bitheader{0-7} \\
	
	\begin{rightwordgroup}{\isprotomsgtype}
		\wordbox[tlr]{1}{6}
	\end{rightwordgroup} \\
	
	\begin{rightwordgroup}{\isprotomsgdata}
		\wordbox[tlr]{1}{File-Flag, 1 Byte} \\
		\wordbox[tlr]{2}{Geräte, variable Länge} \\
		\skippedwords \\
		\wordbox[blr]{1}{} 
	\end{rightwordgroup} 
	
\messageend{refdev}
\begin{description}
	\descriptionitem{File-Flag}
	Das File-Flag gibt darüber Auskunft, ob es sich um die Datei für Partnergeräte oder die Datei für die eigenen Geräte handelt.
	\descriptionitem{Geräte}
	Hier stehen die Adressen aller Geräte, die in der im File-Flag angegebenen Datei vorhanden sind. Falls es sich um die Adressen der Partnergeräte handelt, werden nur diese mitgeschickt, die auch die eigenen Daten speichern. Im Falle, dass es sich um die eigenen Adressen handelt, werden außerdem die Namen der Geräte, die man in der Konfiguration angegeben hat, mitgeschickt.
\end{description}

\subsection{Nachrichten an Partnergeräte}
\begin{itemize}
	\item \gls{partfilereq}s
	\item \gls{partfileres}s
	\item \gls{partfilemsg}s
	\item \gls{partfiledel}s
	\item \gls{filedel}s
\end{itemize}

\subsubsection{\gls{partfilereq}} \label{partnerfilerequest}
\messagestart
	\begin{rightwordgroup}{\isprotomsgtype}
		\wordbox[tlr]{1}{7}
	\end{rightwordgroup} \\
	
	\begin{rightwordgroup}{\isprotomsgdata}
		\wordbox[tlr]{2}{Versionsverlauf, variable Länge} \\
		\skippedwords \\
		\wordbox[blr]{1}{}
	\end{rightwordgroup}
\messageend{partfilereq}
\begin{description}
	\descriptionitem{Versionsverlauf}
	Der Versionsverlauf wird, wie beim \gls{filereq} mitgeschickt, damit der Partner die Datei identifizieren kann. Dies hat den Vorteil, dass der Partner nicht den Dateipfad besitzt und somit nicht auf den Inhalt der Datei schließen kann. Weiters kann aus diesem Versionsverlauf, der lediglich Hashes enthält, ebenfalls nicht der Inhalt der Datei herausgefunden werden, da das Verhashen eine Einwegfunktion ist. 
\end{description}

\subsubsection{\gls{partfileres}}
\messagestart
	\begin{rightwordgroup}{\isprotomsgtype}
		\wordbox[tlr]{1}{8}
	\end{rightwordgroup} \\
	
	\begin{rightwordgroup}{\isprotomsgdata}
		\wordbox[tlr]{2}{Versionsverlauf, variable Länge} \\
		\skippedwords \\
		\wordbox[blr]{1}{}\\
		\wordbox[blr]{1}{Need-Flag}
	\end{rightwordgroup}
\messageend{partfileres}

\begin{description}
	\descriptionitem{Versionsverlauf}
	Da die Datei auf den Partnergeräten mithilfe des Versionsverlaufs identifiziert wird, muss dieser auch hier mitgeschickt werden um Verwechslungen vorzubeugen.
	\descriptionitem{Need-Flag}
	Das Need-Flag im \gls{partfileres} ist äquivalent zum Need-Flag im \gls{fileres} \referenz{fileres}.
\end{description}

\subsubsection{\gls{partfilemsg}}
\messagestart
	\begin{rightwordgroup}{\isprotomsgtype}
		\wordbox[tlr]{1}{9}
	\end{rightwordgroup}\\
	
	\begin{rightwordgroup}{\isprotomsgdata}
		\wordbox[tlr]{2}{Versionsverlauf, variable Länge} \\
		\skippedwords \\
		\wordbox[lr]{1}{} \\
		\wordbox[tlr]{2}{Dateiinhalt, variable Länge} \\
		\skippedwords \\
		\wordbox[lr]{1}{} \\
		\wordbox[tlr]{2}{Dateipfad, variable Länge} \\
		\skippedwords \\
		\wordbox[lr]{1}{} \\
		\wordbox[tlr]{2}{Geräte mit der aktuellen Verision, variable Länge} \\
		\skippedwords \\
		\wordbox[blr]{1}{}
	\end{rightwordgroup}
\messageend{partfilemsg}
\begin{description}
	\descriptionitem{Versionsverlauf}
	Da der Versionsverlauf zur Identifikation der Datei benötigt wird, wird dieser auch bei der \gls{partfilemsg} mitgeschickt. Gleichzeitig dient der letzte Hash im Versionsverlauf als Dateiname auf dem Partnergerät.
	\descriptionitem{Dateiinhalt}
	Um den Inhalt der Datei vor fremdem Zugriff zu schützen, wird dieser mit dem symmetrischen Schlüssel \referenz{symmetricKey} verschlüsselt. 
	\descriptionitem{Dateipfad}
	Wie auch beim \gls{partfilereq} wird der Dateipfad verschlüsselt übertragen. 
	\descriptionitem{Geräte mit der aktuellen Version}
	Damit alle Geräte auch auf den gleichen Stand kommen, wie durch direkte Kommunikation, werden auch die Geräte mit der aktuellen Version übertragen. 
\end{description}

\subsubsection{\gls{partfiledel}}
\messagestart
	\begin{rightwordgroup}{\isprotomsgtype}
		\wordbox[tlr]{1}{10}
	\end{rightwordgroup} \\
	
	\begin{rightwordgroup}{\isprotomsgdata}
		\wordbox[tlr]{2}{Dateipfad, variable Länge} \\
		\skippedwords \\
		\wordbox[blr]{1}{} \\
	\end{rightwordgroup}
\messageend{partfiledel}
\begin{description}
	\descriptionitem{Dateipfad}
	Das einzige Attribut der \gls{partfiledel} ist der verschlüsselte Dateipfad. Dieser wird eine Woche lang gespeichert und dann gelöscht.
\end{description}
\subsubsection{\gls{filedel}}
Um Platz auf den Partnergeräten zu sparen, werden die Daten gelöscht, sobald diese auf allen Geräten verteilt sind. Dies wird mit der \gls{filedel} initiiert. In der \gls{filedel} zwischen Partnergerät und eigenem Gerät wird jedoch wird der Versionsverlauf statt, wie in der \gls{filedel} zwischen zwei eigenen Geräten, des Pfades angegeben.
\messagestart
	\begin{rightwordgroup}{\isprotomsgtype}
		\wordbox[tlr]{1}{11}
	\end{rightwordgroup}\\
	
	\begin{rightwordgroup}{\isprotomsgdata}
		\wordbox[tlr]{2}{Versionsverlauf, variable Länge} \\
		\skippedwords \\
		\wordbox[blr]{1}{}
	\end{rightwordgroup}
\messageend{filedel}
\begin{description}
	\descriptionitem{Versionsverlauf}
	Der Versionsverlauf gibt die Datei an, die vom Partnergerät gelöscht werden soll.
\end{description}