
Um die Hauptanforderung an die Umsetzung, den Verzicht auf zentrale Server im System, realisieren zu können,
wird ein \gls{p2pnet} benötigt. Über dieses läuft die Kommunikation der Synchronisationsanwendung.
Dabei muss das Netzwerk vollständig dezentral aufgebaut sein, um ohne Server funktionieren zu können.
Gleichzeitig müssen alle Teilnehmer des Netzwerks dieselben Berechtigungen haben, weswegen es keinen
Teilnehmer geben kann, der besondere Befugnisse hat. Deshalb ist es notwendig, dass sich das Netzwerk
und die Teilnehmer selbst organisieren und verwalten, ohne dabei auf ein zentrales Organ angewiesen
zu sein.

Dieses \gls{p2pnet} ist in einer separaten \gls{cl}, dem \gls{dcl}, umgesetzt.
Das ermöglicht es einerseits, \gls{dcl} für Anwendungen von fremden Entwicklern zu öffnen, sodass diese auf das schon bestehende Netzwerk
zurückgreifen können und nicht erst ein eigenes umsetzen müssen, und andererseits erleichtert die klare Abgrenzung der Funktionen
zwischen \gls{cl} und eigentlicher Synchronisationsanwendung die Umsetzung beider erheblich.
