
\label{dcl-packetcomponents}

\subsection*{FlexNum}
\label{dcl-packetcomponents-flexnum}
Ein \gls{fnumc} ist ein \gls{pktc} zur Übertragung einer bis zu 64 Bits großen
Ganzzahl.
Die Besonderheit liegt darin, dass der \gls{fnumc} zwar Zahlen bis zu $2^{64}-1$ übertragen kann,
für Zahlen bis $2^{7}-1$, welche bereits mit 7 Bits darstellbar sind, aber trotzdem nur ein Byte
zur Übertragung benötigt.

Das funktioniert durch Kodierung der Anzahl der folgenden Bytes im ersten Byte. Hierbei zeigt die
Anzahl der führenden Bits mit dem Wert 1 im ersten Byte an, wie viele Bytes für die Darstellung der
Zahl folgen.
Ein Bit mit dem Wert 0 wird im ersten Byte also jedenfalls benötigt, um die Abfolge der führenden
Bits mit dem Wert 1 zu beenden. Im Fall von Zahlen von 0 bis 127 ($2^{7}-1$) hat das erste Bit im
ersten Byte den Wert 0 und zeigt damit an, dass keine Bytes folgen. Die restlichen 7 Bits werden
zur Darstellung der Zahl verwendet.

Nachfolgend ist eine Tabelle angeführt, die einen Überblick über die mit einem \gls{fnumc}
darstellbaren Wertbereiche, das Bitpräfix im Initialisierungsbyte, die Anzahl der folgenden Bytes
sowie die Effizienz der Speichernutzung gibt.

Die Effizienz berechnet sich durch $\dfrac{log_2(n_{max})}{N_{bytes} \cdot 8}$, wobei $n_{max}$ der
mit der jeweiligen Anzahl an Folgebits maximal darstellbare Wert ist und $N_{bytes}$ der
Gesamtanzahl an benötigten Bytes entspricht.

\begin{table}[H]
\begin{centering}
\begin{tabular}{|c|l|c|r|}
	\hline 
	Erstes Byte & Wertbereich & Folgebytes & Effizienz\tabularnewline
	\hline 
	\hline 
	\code{0XXXXXXX} & $0$ -- $127$ & $0$ & $87,50\%$\tabularnewline
	\hline
	\code{10XXXXXX} & $128$ -- $16511$ & $1$ & $87,57\%$\tabularnewline
	\hline
	\code{110XXXXX} & $16512$ -- $2113663$ & $2$ & $87,55\%$\tabularnewline
	\hline
	\code{1110XXXX} & $2113664$ -- $270549119$ & $3$ & $87,54\%$\tabularnewline
	\hline
	\code{11110XXX} & $270549120$ -- $34630287487$ & $4$ & $87,53\%$\tabularnewline
	\hline
	\code{111110XX} & $34630287488$ -- $4432676798591$ & $5$ & $87,52\%$\tabularnewline
	\hline
	\code{1111110X} & $4432676798592$ -- $567382630219903$ & $6$ & $87,52\%$\tabularnewline
	\hline
	\code{11111110} & $567382630219904$ -- $72624976668147839$ & $7$ & $87,52\%$\tabularnewline
	\hline
	\code{11111111} & $72624976668147840$ -- $18519369050377699455$ & $8$ & $88,90\%$\tabularnewline
	\hline
\end{tabular}
\par\end{centering}
\protect\caption{FlexNum -- Darstellbare Werte}
\end{table}

Nachfolgend ist jener Teil des Quellcodes von \gls{dcl} angeführt, welcher zum Lesen eines
\gls{fnumc} angewandt wird.\\

\javalisting
\begin{minipage}{\linewidth}
\begin{lstlisting}[caption={Lesen eines \gls*{fnumc} (Java)},captionpos=b]
long readFlexNum() {
	
	byte iByte = read();
	int additionalBytes = 8;
	long num = 0;
	long offset = 0;
	
	for(int i = 0; i < 8; i++) {
		if((iByte & (0x80 >> i)) == 0) {
			additionalBytes = i;
			break;
		}
	}
	
	num |= (iByte & ((0x100 >> additionalBytes) - 1));
	for(int i = 0; i < additionalBytes; i++) {
		num <<= 8;
		num |= (read() & 0xFF);
		offset += (0x80L << 7*i);
	}
	
	return num + offset;
	
}
\end{lstlisting}
\end{minipage}



