
\subsection{Notwendigkeit}
Da es bereits ab einer geringen Anzahl an Teilnehmern im \gls{p2pnet} nicht mehr praktikabel ist,
ein vollständig vermaschtes Netz zu führen, weil die Anzahl an dafür notwendigen Verbindungen die
Kapazitäten der Teilnehmer überschreitet, ist es nötig, einen Routingmechanismus in das Netzwerk
zu implementieren. Damit können Nachrichten im \gls{p2pnet} von jedem Teilnehmer zu jedem beliebigen
anderen Teilnehmer gesendet werden, ohne dass die beiden kommunizierenden Teilnehmer direkt
miteinander verbunden sein müssen.

Dieses Unterkapitel beschreibt das Routing im \gls{cnet}.

\subsection{Schwierigkeit}
Im Gegensatz zu herkömmlichen Computernetzwerken sind die Verbindungen zwischen den einzelnen
Teilnehmern in einem \gls{p2pnet} variabel und ändern sich ständig. Die Wege für Pakete statisch
vorzugeben funktioniert deshalb nicht, stattdessen ist es notwendig, dynamisches Routing zu
implementieren.

Die meisten im Einsatz befindlichen Verfahren zum dynamischen Routing betrachten die gesamte
Topologie und suchen darin Wege für Pakete. Bei zu großen Topologien werden Teile des Netzwerks
zusammengefasst und die Teilnehmer innerhalb dieses Teils von Außerhalb als ein einziger Teilnehmer
betrachtet. Der Pfadfindungsprozess kann dann effizienter gestaltet werden, indem das Routing in
2 Schritten erfolgt: zuerst außerhalb bis zum zusammengefassten Teil und anschließend innerhalb
des zusammengefassten Teils des Netzwerks, wo das Routing zum endgültigen Ziel erfolgt.

Diese Technik funktioniert jedoch nur in einem hierarchisch aufgebauten Netzwerk, in dem Teilnehmer
mit ähnlichen Adressen logisch nah beieinander sind, wie das beispielsweise in einem IPv4-Subnet
der Fall ist. In einem dezentralen \gls{p2pnet}, dessen Topologie sich ständig ändert und in dem
die Adresse eines Teilnehmers nicht mit seiner Position in der Topologie zusammenhängt, können
keine Teilnehmer zusammengefasst werden.

Gleichzeitig ist die Topologie des \gls{p2pnet} so groß und ändert sich so oft, dass es praktisch 
unmöglich ist, bei jedem Teilnehmer des Netzwerks eine Kopie aller bestehenden Verbindungen zu
speichern und diese aktuell zu halten, um basierend darauf die Wege zur Weiterleitung von
Nachrichten zu finden.

\subsection{Distanzbasiertes Routing}
Im \gls{cnet} des \gls{dcl} werden nicht die Wege anhand der Topologie gesucht, sondern die
Topologie anhand der gewünschten Wege aufgebaut. Das bedeutet konkret, dass die Teilnehmer, mit
denen sich ein Teilnehmer verbindet, nicht zufällig gewählt werden, sondern ein Teilnehmer sich
tendentiell mit mehr Teilnehmern verbindet, die eine ähnliche Adresse haben, als mit solchen,
deren Adresse sich von der eigenen stark unterscheidet.

Ähnlich sind zwei Adressen dann, wenn ihre numerische Differenz gering ist. Dabei gilt es, auch
den Überlauf beim Überschreiten des mit der Adresslänge maximal darstellbaren Wertes zu beachten.
Nimmt man zur vereinfachten Darstellung eine Adresslänge von 16 Bits an, dann weisen die
hexadezimal angegebenen Adressen \code{E539} und \code{1092} eine Distanz von \code{2B59} auf,
obwohl \code{E539 - 1092 = D4A7}, weil die Distanz über den Überlauf kürzer ist:
\code{10000 - E539 + 1092 = 2B59}.

Des Weiteren werden Teilnehmer, die exakt die selbe Adresse aufweisen, stets direkt miteinander
verbunden, um sicherzustellen, dass Nachrichten deren Zieladresse auf mehrere Teilnehmer verweist,
an alle zugestellt werden. Dass zwei Teilnehmer die selbe Adresse bekommen ist jedoch erstens
extrem unwahrscheinlich und bedeutet zweitens nicht unbedingt, dass ihre Adressen auf den selben
öffentlichen Schlüsseln basieren. Siehe dazu \link{dcl-addr-uniqueness}.

Dadurch, dass die Teilnehmer des Netzwerks jeweils mehr Verbindungen zu Teilnehmern mit Adressen
ähnlich zur eigenen haben, als Verbindungen zu Teilnehmern mit Adressen, die eine hohe Differenz
zur eigenen aufweisen, können Nachrichten einfach an den Teilnehmer weitergeleitet werden, dessen
Adresse die niedrigste Differenz zur Zieladresse der Nachricht aufweist. Auf diese weise gelangt
die Nachricht, logisch betrachtet, immer näher an ihr Ziel und wird schlussendlich an den Adressat
zugestellt.

\subsection{Umsetzung im Quellcode von DCL}
Nachfolgend ist jener Teil des Quellcodes von \gls{dcl} angeführt, welcher im \code{\gls{cnt}}
zum Routinglookup angewandt wird.

\javalisting
\begin{minipage}{\linewidth}
\begin{lstlisting}[caption={Routing im \gls*{cnt} (Java)},captionpos=b]
Nexthops lookup( Data scaledDestinationAddress,
						Address originAddress,
						Object originIdentifierObject ) {
	Nexthops nexthops
	 = localEndpointRoutes.get(scaledDestinationAddress);
	if(nexthops != null) {
		Nexthops twinNeighbors
		 = remoteServiceRoutes.get(scaledDestinationAddress);
		if(twinNeighbors != null) {
			boolean append = true;
			for(ForwardDestination forwardDestination
				 : twinNeighbors) {
				if( forwardDestination.getIdentifierObject()
					 == originIdentifierObject ) {
					append = false;
					break;
				}
			}
			if(append) {
				nexthops.append(twinNeighbors);
			}
		}
		return nexthops;
	}
	nexthops = remoteServiceRoutes
			   .getClosest(scaledDestinationAddress);
	if(nexthops == null) {
		return null;
	}
	for(ForwardDestination forwardDestination : nexthops) {
		if(forwardDestination.getIdentifierObject()
			 == originIdentifierObject) {
			return null;
		}
	}
	return nexthops.getLast();
}
\end{lstlisting}
\end{minipage}

\begin{description}
	
	\item [{\code{scaledDestinationAddress}:}]
		Das \javaarg \code{scaledDestinationAddress} referenziert ein \code{\gls{Data}}-Objekt,
		welches die Zieladresse enthält, für die ein Routinglookup durchgeführt werden soll.
	
	\item [{\code{originAddress}:}]
		Das \javaarg \code{originAddress} referenziert ein \code{\gls{Address}}-Objekt, welches
		den öffentlichen Schlüssel des Teilnehmers enthält, von welchem die zu weiterleitende
		Nachricht empfangen wurde.
	
	\item [{\code{originIdentifierObject}:}]
		Das \javaarg \code{originIdentifierObject} referenziert ein Objekt, das den Teilnehmer
		identifiziert, von welchem die zu weiterleitende Nachricht empfangen wurde. Dieses wird
		benötigt, um später sicherzustellen, dass die Nachricht nicht zum selben Teilnehmer
		weitergeleitet wird, von welchem sie schon empfangen wurde.
	
	\item [{\code{\gls{Nexthops}}:}]
		\glsdesc{Nexthops}.
	
	\item [{\code{\gls{ForwardDestination}}:}]
		\glsdesc{ForwardDestination}.
	
	\item [{\code{localEndpointRoutes}:}]
		Die \javainstvar \code{localEndpointRoutes} referenziert ein Objekt, das zu Schlüsseln
		in Form von \code{\gls{Data}}-Objekten Werte in Form von \code{\gls{Nexthops}}-Objekten
		speichert. Dieses wird benutzt, um zu Adressen die entsprechenden
		\code{\gls{Nexthops}}-Objekte finden zu können, an die Nachrichten weitergeleitet werden
		können. In diesem Fall handelt es sich dabei ausschließlich um Adressen, deren Endpunkte
		sich am lokalen \gls{service} befinden.
	
	\item [{\code{remoteServiceRoutes}:}]
		Die \javainstvar \code{localEndpointRoutes} referenziert ein Objekt, das zu Schlüsseln
		in Form von \code{\gls{Data}}-Objekten Werte in Form von \code{\gls{Nexthops}}-Objekten
		speichert. Dieses wird benutzt, um zu Adressen die entsprechenden
		\code{\gls{Nexthops}}-Objekte finden zu können, an die Nachrichten weitergeleitet werden
		können. In diesem Fall handelt es sich dabei ausschließlich um Adressen, deren Endpunkte
		sich auf direkt verbundenen, aber nicht am lokalen \gls{service} befinden.
	
\end{description}

Mit \code{localEndpointRoutes.get(scaledDestinationAddress)} wird zu Beginn des Routinglookups
nach lokalen \glslink{endp}{Endpunkten} mit der genauen Zieladresse dieser Nachricht gesucht.
Werden ein oder mehrere \glspl{endp} gefunden, so wird weiters geprüft, ob es entfernte Endpunkte
gibt, die die selbe Adresse haben. Gibt es solche \glspl{endp}, dann wird sichergestellt, dass
keiner dieser \glspl{endp} derjenige \gls{endp} ist, von dem die Nachricht empfangen wurde und
dass die Nachricht somit nicht wieder zu diesem \gls{endp} weitergeleitet wird.
Anschließend werden die \glspl{endp} mit der gleichen Adresse wie der lokale \gls{endp} zur Liste
der \glspl{endp}, an die die Nachricht weitergeleitet werden soll, hinzugefügt.
Das geschieht, indem das von \code{twinNexthops} referenzierte \code{\gls{Nexthops}}-Objekt
mit \code{nexthops.append(twinNexthops)} an das von \code{nexthops} referenzierte
\code{\gls{Nexthops}}-Objekt angehängt wird.

Wurden keine lokalen \glspl{endp} mit der Zieladresse der Nachricht gefunden, dann wird mit
\code{remoteServiceRoutes.getClosest(scaledDestinationAddress)} nach direkt verbundenen
\glslink{endp}{Endpunkten} gesucht, deren Adresse der Zieladresse am ähnlichsten ist.
Anschließend wird, wie schon bei den lokalen \glslink{endp}{Endpunkten}, sichergestellt,
dass die Liste von \glspl{endp} nicht den \gls{endp} enthält, von dem die Nachricht empfangen
wurde.
Falls das der Fall ist, wird \code{null} zurückgegeben und die Nachricht in Folge verworfen.

Anderenfalls wird nur der letzte \gls{endp} in der von \code{nexthops} referenzierten
Liste der \glspl{endp} zurückgegeben: Enthält die Liste mehrere Einträge, so haben die darin
enthaltenen \glspl{endp} die selbe Adresse. Die Nachricht soll in diesem Fall abernur an einen
dieser \glspl{endp} weitergeleitet werden, da sie von dort an die restlichen \glspl{endp} mit der
gleichen Adresse weitergeleitet wird.




