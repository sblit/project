
\newglossaryentry{gls_udp}{
	name=User Datagram Protocol,
	description={Verbindungsloses Übertragungsprotokoll aus der Internet protocol suite \cite{wikipedia:udp}}
}

\newacronym[see={[Glossar:]{gls_udp}}]{udp}{UDP}{User Datagram Protocol\glsadd{gls_udp}}

\newglossaryentry{gls_tcp}{
	name=Transmission Control Protocol,
	description={Verbindungsorientiertes, verlässliches Übertragungsprotokoll aus der Internet protocol suite \cite{wikipedia:tcp}}
}

\newacronym[see={[Glossar:]{gls_tcp}}]{tcp}{TCP}{Transmission Control Protocol\glsadd{gls_tcp}}

\newglossaryentry{cl}{
	name=Kommunikationsschicht,
	description={Eine Schicht im netzwerktechnischen Sinne, die zur Kommunikation verwendet wird},
	see={[Siehe:]{gls_dcl}}
}

\newglossaryentry{p2pnet}{
	name=Peer-to-Peer-Netzwerk,
	plural=Peer-to-Peer-Netzwerke,
	description={Ein Kommunikationsnetzwerk, in dem die einzelnen Kommunikationspartner im Gegensatz zu einer klassischen Client-Server-Infrastruktur direkt miteinander kommunizieren}
}

\newglossaryentry{gls_dcl}{
	name=Decentralized Communication Layer,
	description={Die auf einem dezentralen \gls{p2pnet} basierende \gls{cl}, die \sblit zur Kommunikation benutzt}
}

\newacronym[see={[Glossar:]{gls_dcl}}]{dcl}{DCL}{Decentralized Communication Layer\glsadd{gls_dcl}}

\newglossaryentry{aenc}{
	name=Asymmetrisches Verschlüsselungsverfahren,
	description={Verschlüsselungsverfahren, bei dem zur Entschlüsselung einer Nachricht ein anderer Schlüssel verwendet wird, als zur Verschlüsselung},
	see={[Siehe:]{rsa}}
}

\newglossaryentry{rsa}{
	name=RSA,
	description={\gls{aenc}, benannt nach und erstmals veröffentlich von Ron Rivest, Adi Shamir and Leonard Adleman \cite{wikipedia:rsa}},
	see={[Siehe:]{aenc}}
}

\newglossaryentry{gls_aes}{
	name=Advanced Encryption Standard,
	description={Standard-Verschlüsselungsverfahren für \glspl{link}},
	see={[Siehe:]{link}}
}

\newacronym[see={[Glossar:]{gls_aes}}]{aes}{AES}{Advanced Encryption Standard\glsadd{gls_aes}}

\newglossaryentry{gls_gcm}{
	name=Galois/Counter Mode,
	description={Standard-Betriebsmodus für die Verschlüsselung von \glspl{link} mit \acrshort{aes}},
}

\newacronym[see={[Glossar:]{gls_gcm}}]{gcm}{GCM}{Galois/Counter Mode\glsadd{gls_gcm}}

\newglossaryentry{hash}{
	name=Hashalgorithmus,
	plural=Hashalgorithmen,
	description={Algorithmus, der eine große Eingabemenge auf eine kleinere Zielmenge (\gls{hashval}) abbildet \cite{wikipedia:hash}}
}

\newglossaryentry{hashval}{
	name=Hashwert,
	plural=Hashwerte,
	description={Ergebnis der Anwendung eines \gls{hash}}
}

\newglossaryentry{digestlength}{
	name=Digest Length,
	description={Die Länge der \glspl{hashval} eines \gls{hash}}
}

\newglossaryentry{sha1}{
	name=SHA-1,
	description={Secure Hash Algorithm. Im \gls{dcl} standardmäßig verwendeter \gls{hash}},
	see={[Siehe:]{hash}}
}

\newglossaryentry{remotekey}{
	name=Remote Key,
	description={Schlüssel, der lokal nicht bekannt ist und zu dessen besitzender Anwendung eine Verbindung besteht, über die Verschlüsselungs- und Entschlüsselungsanfragen weitergeleitet werden. Siehe \link{dcl-asproto-remotekeys}}
}

\newglossaryentry{net}{
	name=Netzwerk,
	plural=Netzwerke,
	description={Ein \gls{p2pnet} innerhalb des \gls{dcl}},
	see={[Siehe:]{cnet}}
}

\newglossaryentry{nt}{
	name=Network Type,
	plural=Network Types,
	description={Beschreibt ein \gls{net} innerhalb des \gls{dcl} samt Adresskonzept und Routingverfahren},
	see={[Siehe:]{net}}
}

\newglossaryentry{ntid}{
	name=Network Type Identifier,
	plural=Network Type Identifiers,
	description={Der eindeutige Name eines \gls{nt}s},
	see={[Siehe:]{nt}}
}

\newglossaryentry{cnet}{
	name=Circle Network,
	description={Standard \gls{net} im \gls{dcl}, \gls{ntid} \code{\cnetid{}}},
	see={[Siehe:]{net}}
}

\newglossaryentry{cnt}{
	name=CircleNetworkType,
	description={Im Quellcode von \gls{dcl} definierte Klasse zur Darstellung des \gls{nt} des \gls{cnet}},
	see={[Siehe:]{nt}}
}

\newglossaryentry{netint}{
	name=Netzwerkintegration,
	description={Das Aufbauen von Verbindungen mit anderen \glspl{service}, um optimal in das Routing eines \glslink{net}{Netzwerks} eingebunden zu sein}
}

\newglossaryentry{service}{
	name=DCL-Service,
	description={Ein Teilnehmer des \gls{dcl}. Kann mehrere Adressen in mehreren unterschiedlichen \glslink{net}{Netzwerken} hosten}, % TODO
	see={[Siehe:]{dcl}}
}

\newglossaryentry{endp}{
	name=Endpunkt,
	plural=Endpunkte,
	description={Eine auf dem lokalen bzw. einem direkt verbundenen \gls{service} gehostete Adresse in einem bestimmten \gls{net}},
	see={[Siehe:]{Nexthops}}
}

\newglossaryentry{asendp}{
	name=Network Endpoint,
	description={Eine auf einer \gls{asconn} bekanntgegebene Adresse in einem \gls{net}, dem die Anwendung mit dieser Adresse beigetreten ist}
}

\newglossaryentry{isproto}{
	name=Interservice-Protokoll,
	description={Das zur Kommunikation zwischen zwei direkt verbundenen \glspl{service} verwendete Protokoll}
}

\newglossaryentry{link}{
	name=Link,
	description={Verlässlicher und verschlüsselter Übertragungskanal basierend auf \acrshort{udp}. Basis für Verbindungen zwischen \glspl{service}. Siehe \link{dcl-link}}
}

\newglossaryentry{channel}{
	name=Channel,
	description={Getrennter Datenkanal eines \glslink{link}{Links}. Siehe \link{dcl-link-channels}}
}

\newglossaryentry{mgmtch}{
	name=Management Channel,
	description={\Gls{channel} eines \glslink{link}{Links}, der zur Übertragung von Nachrichten zur Verwaltung des \glslink{link}{Links} genutzt wird},
	see={[Siehe:]{channel}}
}

\newglossaryentry{channelid}{
	name=Channel Identifier,
	description={Zahl zur eindeutigen Identifizierung eines \glslink{channel}{Channels} auf einem \gls{link}}
}

\newglossaryentry{dataid}{
	name=Data Identifier,
	description={Fortlaufende Zahl zur eindeutigen Identifizierung eines über einen \gls{channel} eines \glslink{link}{Links} übertragenen Datenblocks und zur Angabe der Reihenfolge dieses Datenblocks im Datenstrom des \glslink{channel}{Channels}}
}

\newglossaryentry{chblockstatrep}{
	name=Channel Block Status Report,
	description={Statusinformation über die empfangenen \glspl{dataid} eines \glslink{channel}{Channels}, anhand der fehlende Pakete erkannt und erneut übertragen werden können},
	see={[Siehe:]{channel}}
}

\newglossaryentry{fcnt}{
	name=Flusskontrolle,
	description={Einschränkung der über einen Kommunikationskanal übertragenen Menge an Daten zur Vermeidung von Verstopfungen}
}

\newglossaryentry{protoid}{
	name=Protocol Identifier,
	description={Eindeutige Kennung des Protokolls, über das auf einem \gls{channel} eines \glslink{link}{Links} kommuniziert wird}
}

\newglossaryentry{mcp}{
	name=Management Channel Protocol,
	description={Protokoll zur Verwaltung eines \glslink{link}{Links}},
	see={[Siehe:]{gls_bmcp}}
}

\newglossaryentry{gls_bmcp}{
	name=Basic Management Channel Protocol,
	description={Standardmäßig verwendetes \gls{mcp} zur Verwaltung von \glspl{link}. Siehe \link{dcl-link-bmcp}}
}

\newacronym[see={[Glossar:]{gls_bmcp}}]{bmcp}{BMCP}{Basic Management Channel Protocol\glsadd{gls_bmcp}}

\newglossaryentry{cinitmethod}{
	name=Crypto Initialization Method,
	description={Art der Verschlüsselung eines mit \acrshort{bmcp} verwalteten \glslink{link}{Links} und Methode zur Initialisierung dieser Verschlüsselung},
	see={[Siehe:]{gls_bmcp}}
}

\newglossaryentry{cimid}{
	name=Crypto Initialization Method Identifier,
	description={Zahl zur eindeutigen Kennung einer \gls{cinitmethod}},
	see={[Siehe:]{cinitmethod}}
}

\newglossaryentry{asproto}{
	name=Application-to-Service-Protokoll,
	description={Das zur Kommunikation zwischen einer Anwendung und einem \gls{service} verwendete Protokoll}
}

\newglossaryentry{asconn}{
	name=Application-to-Service-Verbindung,
	plural=Application-to-Service-Verbindungen,
	description={Verbindung über \acrshort{tcp} zwischen einer Anwendung und einem \gls{service}, auf der mittels \gls{asproto} kommuniziert wird}
}

\newglossaryentry{appch}{
	name=Application Channel,
	description={Ein verlässlicher Übertragungskanal zur Kommunikation zwischen zwei Instanzen einer auf dem \gls{dcl} aufbauenden Anwendung}
}

\newglossaryentry{cchlg}{
	name=Crypto Challenge,
	description={Anforderung, mit dem einem bekannten öffentlichen Schlüssel zugehörigen privaten Schlüssel Daten zu signieren, um so den Besitz des gesamten Schlüsselpaars zu beweisen}
}

\newglossaryentry{pktc}{
	name=Packet Component,
	description={Teil einer Message, beispielsweise Ganzzahl oder String. Siehe \link{dcl-packetcomponents}}
}

\newglossaryentry{fnumc}{
	name=FlexNum-Component,
	description={\gls{pktc} zur Übertragung einer bis zu 64 Bits großen Ganzzahl. Siehe \link{dcl-packetcomponents-flexnum}}
}

\newglossaryentry{gls_lla}{
	name=Lower Level Address,
	plural=Lower Level Addresses,
	description={Die Adresse eines \gls{service} auf der Netzwerkschicht unter \gls{dcl}, de facto die IP-Adresse und Port, unter der ein \gls{service} erreichbar ist}
}

\newacronym[plural={LLAs}, longplural={Lower Level Addresses}, see={[Glossar:]{gls_lla}}]{lla}{LLA}{Lower Level Address\glsadd{gls_lla}}

\newglossaryentry{netpkt}{
	name=Network Packet,
	description={Geroutete Nachricht innerhalb eines \glslink{net}{Netzwerks}}
}

\newglossaryentry{keyc}{
	name=KeyComponent,
	description={Basis für \glspl{pktc} zur Übertragung von Schlüsseldaten}
}

\newglossaryentry{actid}{
	name=Action Identifier,
	description={String, der in Anfragen übertragen wird und Auskunft über deren Grund geben soll}
}

\newglossaryentry{addrslot}{
	name=Address Slot,
	description={Zahl zur Referenzierung von auf einem \gls{service} gehosteten Adressen}
}

\newglossaryentry{netslot}{
	name=Network Slot,
	description={Zahl zur Referenzierung von \glslink{net}{Netzwerken}, denen ein \gls{service} auf einem \gls{isch} mit einer oder mehreren Adressen beigetreten ist}
}

\newglossaryentry{appchslot}{
	name=Application Channel Slot,
	description={Zahl zur Referenzierung von \glspl{appch} in \glspl{isch}}
}

\newglossaryentry{endpslot}{
	name=Network Endpoint Slot,
	description={Zahl zur Referenzierung von \glspl{asendp} auf \glspl{asconn}}
}

\newglossaryentry{isch}{
	name=Interservice Channel,
	description={\Gls{channel} eines \glslink{link}{Links} zwischen zwei \glspl{service}, innerhalb dessen mit dem \gls{isproto} kommuniziert wird}
}

\newglossaryentry{connbase}{
	name=Connection Base,
	description={Berechtigungslevel für einen \gls{addrslot} eines \gls{isch}} % TODO
}

% DCL source classes

\newglossaryentry{Data}{
	name=Data,
	description={Im Quellcode von \gls{dcl} definierte Klasse zur Speicherung von Binärdaten}
}

\newglossaryentry{Hash}{
	name=Hash,
	description={Im Quellcode von \gls{dcl} definierte Klasse zur Anwendung von \glspl{hash} auf \code{\gls{Data}}-Objekte}
}

\newglossaryentry{Address}{
	name=Address,
	description={Im Quellcode von \gls{dcl} definierte Klasse zur Speicherung von öffentlichen Schlüsseln anderer Teilnehmer}
}

\newglossaryentry{Nexthops}{
	name=Nexthops,
	description={Im Quellcode von \gls{dcl} definierte Klasse zur Speicherung von verbundenen \glslink{endp}{Endpunkten} in der Form von \code{\gls{ForwardDestination}}-Objekten}
}

\newglossaryentry{ForwardDestination}{
	name=ForwardDestination,
	description={Im Quellcode von \gls{dcl} definierte Klasse zur Speicherung eines verbundenen \glslink{endp}{Endpunkts},
			an den eine Nachricht weitergeleitet werden kann, inklusive dessen \code{\gls{Address}}- und
			\code{\gls{InterserviceChannel}}-Objekten},
	see={[Siehe:]{endp}}
}

\newglossaryentry{InterserviceChannel}{
	name=InterserviceChannel,
	description={Im Quellcode von \gls{dcl} definierte Klasse zur Verwaltung einer Verbindung zwischen zwei \glspl{service}},
	see={[Siehe:]{isch}}
}

\newglossaryentry{natdev}{
	name=NAT-Gerät,
	plural=NAT-Geräte,
	description={Gerät, das \acrlong{nat} durchführt}
}

\newglossaryentry{gls_nat}{
	name=Network Address Translation,
	description={Auf \glslink{natdev}{NAT-Geräten} durchgeführte Übersetzung von Netzwerkadressen. Siehe \link{dcl-natt}}
}

\newacronym[see={[Glossar:]{gls_nat}}]{nat}{NAT}{Network Address Translation\glsadd{gls_nat}}

\newglossaryentry{natt}{
	name=NAT-Traversal,
	description={Umgehung von \glslink{natdev}{NAT-Geräten}, um Peer-to-Peer-Verbindungen aufbauen zu können}
}

\newglossaryentry{nattpkt}{
	name=NAT-Traversal-Paket,
	plural=NAT-Traversal-Pakete,
	description={Paket, das im Zuge von \gls{natt} gesendet wird},
	see={[Siehe:]{natt}}
}

\newglossaryentry{nathp}{
	name=NAT Hole Punching,
	description={Verfahren zum \gls{natt}, das in der Implementierung des \gls{service} angewandt wird},
	see={[Siehe:]{natt}}
}

\newglossaryentry{ignoredata}{
	name=Ignore Data,
	plural=Ignore Data,
	description={Daten, die als Präfix für \glspl{nattpkt} verwendet werden können, da Pakete, die diese Daten enthalten, vom Empfänger ignoriert werden},
	see={[Siehe:]{nattpkt}}
}

\newglossaryentry{gls_crisp}{
	name=Common Routed Interservice Protocol,
	description={Protokoll, das netzwerkübergreifend zur Kommunikation zwischen nicht direkt miteinander verbundenen \glspl{service} verwendet wird. Siehe \link{dcl-crisp}}
}

\newacronym[see={[Glossar:]{gls_crisp}}]{crisp}{CRISP}{Common Routed Interservice Protocol\glsadd{gls_crisp}}
