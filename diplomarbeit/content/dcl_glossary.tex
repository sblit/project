
\newglossaryentry{cl}{
	name=Kommunikationsschicht,
	description={Eine Schicht im netzwerktechnischen Sinne, die zur Kommunikation verwendet wird},
	see={[Siehe:]{gls_dcl}}
}

\newglossaryentry{p2pnet}{
	name=Peer-to-Peer-Netzwerk,
	description={Ein Kommunikationsnetzwerk, in dem die einzelnen Kommunikationspartner im Gegensatz zu einer klassischen Client-Server-Infrastruktur direkt miteinander kommunizieren}
}

\newglossaryentry{gls_dcl}{
	name=Decentralized Communication Layer,
	description={Die auf einem dezentralen \gls{p2pnet} basierende \gls{cl}, die \sblit zur Kommunikation benutzt}
}

\newacronym[see={[Glossar:]{gls_dcl}}]{dcl}{DCL}{Decentralized Communication Layer\glsadd{gls_dcl}}

\newglossaryentry{aenc}{
	name=Asymmetrisches Verschlüsselungsverfahren,
	description={Verschlüsselungsverfahren, bei dem zur Entschlüsselung einer Nachricht ein anderer Schlüssel verwendet wird, als zur Verschlüsselung verwendet wurde},
	see={[Siehe:]{rsa}}
}

\newglossaryentry{rsa}{
	name=RSA,
	description={\gls{aenc}, benannt nach und erstmals veröffentlich von Ron Rivest, Adi Shamir and Leonard Adleman \cite{wikipedia:rsa}},
	see={[Siehe:]{aenc}}
}

\newglossaryentry{hash}{
	name=Hashalgorithmus,
	plural=Hashalgorithmen,
	description={Algorithmus, der eine große Eingabemenge auf eine kleinere Zielmenge (\gls{hashval}) abbildet \cite{wikipedia:hash}}
}

\newglossaryentry{Hash}{
	name=Hash,
	description={Im Quellcode von \gls{dcl} definierte Klasse zur Anwendung von \glspl{hash} auf \gls{Data}-Objekte}
}

\newglossaryentry{hashval}{
	name=Hashwert,
	plural=Hashwerte,
	description={Ergebnis bei Anwendung eines \gls{hash}}
}

\newglossaryentry{digestlength}{
	name=Digest Length,
	description={Die Länge der \glspl{hashval} eines \gls{hash}}
}

\newglossaryentry{cnt}{
	name=CircleNetworkType,
	description={Standard \gls{nt} im \gls{dcl}},
	see={[Siehe:]{nt}}
}

\newglossaryentry{nt}{
	name=Network Type,
	description={Beschreibt ein Netzwerk innerhalb des \gls{dcl} samt Adresskonzept und Routingverfahren},
	see={[Siehe:]{cnt}}
}

\newglossaryentry{Data}{
	name=Data,
	description={Im Quellcode von \gls{dcl} definierte Klasse zur Speicherung von Binärdaten}
}
