
\newglossaryentry{cl}{
	name=Kommunikationsschicht,
	description={Eine Schicht im netzwerktechnischen Sinne, die zur Kommunikation verwendet wird},
	see={[Siehe:]{gls_dcl}}
}

\newglossaryentry{p2pnet}{
	name=Peer-to-Peer-Netzwerk,
	plural=Peer-to-Peer-Netzwerke,
	description={Ein Kommunikationsnetzwerk, in dem die einzelnen Kommunikationspartner im Gegensatz zu einer klassischen Client-Server-Infrastruktur direkt miteinander kommunizieren}
}

\newglossaryentry{gls_dcl}{
	name=Decentralized Communication Layer,
	description={Die auf einem dezentralen \gls{p2pnet} basierende \gls{cl}, die \sblit zur Kommunikation benutzt}
}

\newacronym[see={[Glossar:]{gls_dcl}}]{dcl}{DCL}{Decentralized Communication Layer\glsadd{gls_dcl}}

\newglossaryentry{aenc}{
	name=Asymmetrisches Verschlüsselungsverfahren,
	description={Verschlüsselungsverfahren, bei dem zur Entschlüsselung einer Nachricht ein anderer Schlüssel verwendet wird, als zur Verschlüsselung verwendet wurde},
	see={[Siehe:]{rsa}}
}

\newglossaryentry{rsa}{
	name=RSA,
	description={\gls{aenc}, benannt nach und erstmals veröffentlich von Ron Rivest, Adi Shamir and Leonard Adleman \cite{wikipedia:rsa}},
	see={[Siehe:]{aenc}}
}

\newglossaryentry{hash}{
	name=Hashalgorithmus,
	plural=Hashalgorithmen,
	description={Algorithmus, der eine große Eingabemenge auf eine kleinere Zielmenge (\gls{hashval}) abbildet \cite{wikipedia:hash}}
}

\newglossaryentry{hashval}{
	name=Hashwert,
	plural=Hashwerte,
	description={Ergebnis bei Anwendung eines \gls{hash}}
}

\newglossaryentry{digestlength}{
	name=Digest Length,
	description={Die Länge der \glspl{hashval} eines \gls{hash}}
}

\newglossaryentry{sha1}{
	name=SHA-1,
	description={Im \gls{dcl} standardmäßig verwendeter \gls{hash}. SHA steht für Secure Hash Algorithm},
	see={[Siehe:]{hash}}
}

\newglossaryentry{net}{
	name=Netzwerk,
	plural=Netzwerke,
	description={Ein \gls{p2pnet} innerhalb des \gls{dcl}},
	see={[Siehe:]{cnet}}
}

\newglossaryentry{nt}{
	name=Network Type,
	plural=Network Types,
	description={Beschreibt ein \gls{net} innerhalb des \gls{dcl} samt Adresskonzept und Routingverfahren},
	see={[Siehe:]{net}}
}

\newglossaryentry{ntid}{
	name=Network Type Identifier,
	plural=Network Type Identifiers,
	description={Der eindeutige Name eines \gls{nt}},
	see={[Siehe:]{nt}}
}

\newglossaryentry{cnet}{
	name=Circle Network,
	description={Standard \gls{net} im \gls{dcl}, \gls{ntid} \code{\cnetid{}}},
	see={[Siehe:]{net}}
}

\newglossaryentry{cnt}{
	name=CircleNetworkType,
	description={Im Quellcode von \gls{dcl} definierte Klasse zur Darstellung des \gls{nt} des \gls{cnet}},
	see={[Siehe:]{nt}}
}

\newglossaryentry{service}{
	name=DCL-Service,
	description={Ein Teilnehmer des \gls{dcl}. Kann mehrere Adressen in mehreren unterschiedlichen \glslink{net}{Netzwerken} hosten}, % TODO
	see={[Siehe:]{dcl}}
}

\newglossaryentry{endp}{
	name=Endpunkt,
	plural=Endpunkte,
	description={Eine auf dem lokalen bzw. einem direkt verbundenen \gls{service} gehostete Adresse in einem bestimmten \gls{net}},
	see={[Siehe:]{Nexthops}}
}

\newglossaryentry{isproto}{
	name=Interservice-Protokoll,
	description={Das zur Kommunikation zwischen zwei direkt verbundenen \glspl{service} verwendete Protokoll}
}

\newglossaryentry{appch}{
	name=Application Channel,
	description={Ein verlässlicher Übertragungskanal zur Kommunikation zwischen zwei Instanzen einer auf dem \gls{dcl} aufbauenden Anwendung}
}

% DCL source classes

\newglossaryentry{Data}{
	name=Data,
	description={Im Quellcode von \gls{dcl} definierte Klasse zur Speicherung von Binärdaten}
}

\newglossaryentry{Hash}{
	name=Hash,
	description={Im Quellcode von \gls{dcl} definierte Klasse zur Anwendung von \glspl{hash} auf \gls{Data}-Objekte}
}

\newglossaryentry{Address}{
	name=Address,
	description={Im Quellcode von \gls{dcl} definierte Klasse zur Speicherung von öffentlichen Schlüsseln anderer Teilnehmer}
}

\newglossaryentry{Nexthops}{
	name=Nexthops,
	description={Im Quellcode von \gls{dcl} definierte Klasse zur Speicherung von verbundenen \glslink{endp}{Endpunkten} in
			der Form von \gls{ForwardDestination}-Objekten}
}

\newglossaryentry{ForwardDestination}{
	name=ForwardDestination,
	description={Im Quellcode von \gls{dcl} definierte Klasse zur Speicherung eines verbundenen \glslink{endp}{Endpunkts},
			an den eine Nachricht weitergeleitet werden kann, inklusive dessen \gls{Address}- und
			\gls{InterserviceChannel}-Objekten},
	see={[Siehe:]{endp}}
}

\newglossaryentry{InterserviceChannel}{
	name=InterserviceChannel,
	description={Im Quellcode von \gls{dcl} definierte Klasse zur Verwaltung einer Verbindung zwischen zwei \glspl{service}}
}

\newglossaryentry{FlexNum}{
	name=FlexNum,
	description={Im \gls{dcl} definierter \gls{PacketComponent} zur Übertragung einer bis zu 64 Bits großen Ganzzahl. Siehe \link{dcl-packetcomponents-flexnum}}
}

\newglossaryentry{PacketComponent}{
	name=PacketComponent,
	description={Im \gls{dcl} definierte abstrakte Klasse zum Lesen und Schreiben bestimmter Teile von Messages}
}
