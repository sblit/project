\newglossaryentry{partnership}{
	name=Partnerschaft,
	plural=Partnerschaften,
	description={Um Dateien/Dateiblöcke extern speichern zu können, muss eine
	sogenannte Partnerschaft mit anderen Geräten eingangen werden, bei der man
	Dateiblöcke auf den jeweilig anderen Geräten speichert, während man selber
	Speicherplatz für diese Partner freigibt, in dem deren verschlüsselte
	Dateiblöcke	gespeichert werden}
}

\newglossaryentry{syncpartner}{
	name=Synchronisationspartner,
	plural=Synchronisationspartner,
	description={Gerät mit dem der \sblit-Ordner synchronisiert wird}
}

\newglossaryentry{p2plink}{
	name=Peer-to-Peer-Link,
	plural=Peer-to-Peer-Links,
	description={Ein direkte Verbindung zwischen zwei Endgeräten}
}

\newglossaryentry{syncconflict}{
	name=Synchronisationskonflikt,
	plural=Synchronisationskonflikte,
	description={Wenn eine synchronisierte Datei von zwei Rechnern bearbeitet wird,
	ohne, dass sie in der Zwischenzeit synchronisert werden konnte. Es existieren
	dann zwei Versionen der gleichen Datei. Man spricht dann von einem Konflikt}
}

\newglossaryentry{filecloud}{
	name=Filecloud,
	description={Externer Speicher im Internet, auf dem Dateien gespeichert werden
	können. Meistens bestehend aus einem oder mehreren Servern}
}
