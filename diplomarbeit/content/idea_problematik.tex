Einen Server als Dreh- und Angelpunkt eines Dienstes einzusetzen scheint die am
einfachsten umzusetzende und kosteneffizienteste Methode zu sein, birgt aber Sicherheitsrisiken
für den Benutzer.

In der Vergangenheit kam es schon öfters zu Hackerangriffen auf große Firmen, bei
denen große Mengen an Kundendaten gestohlen wurden. Vorfälle wie diese zeigen, wie leicht
Mengen an privaten Daten in falsche Hände fallen können, wenn diese zentral an einem Ort
gespeichert werden.

Die Speicherung privater Daten von Usern auf zentralen Servern eines Unternehmens ebnet
Geheimdiensten und staatlichen Sicherheitsbehörden außerdem den Weg, flächendeckende Überwachung
durchzuführen und damit massiv in die Privatsphäre aller User des Synchronisationsdienstes
einzugreifen, da das Unternehmen als Anbieter einer solchen \gls{filecloud} unter Umständen zur
Herausgabe der Nutzerdaten gesetzlich verpflichtet ist.

Hauptursache für diese Problematik ist die unverschlüsselte und vor allem zentrale
Speicherung der Daten.

Bei \sblit werden die angesprochenen Probleme umgangen, ohne den durch Synchronisationsdienste
gebotenen Komfort zu mindern.
