In der Vergangenheit kam es schon öfters zu Hackerangriffen auf große Firmen, bei
denen große Mengen an Kundendaten gestohlen wurden. Vorfälle wie diese zeigen wie leicht
Mengen anprivaten Daten in falsche Hände fallen können, wenn diese zentral an einem Ort
gespeichert werden.

Geheimdienste oder staatliche Sicherheitsbehörden haben außerdem ein leichtes
Spiel an Userdaten heranzukommen, wenn sie gesammelt auf den Servern eines
Unternehmens gespeichert werden. Diese Unternehmen sind als Anbieter solcher
\gls{cloudstorage} unter Umständen zur Herausgabe der Nutzerdaten gesetzlich
verpflichtet.

Und selbst wenn der duchschnittliche Benutzer nichts zu verbergen hat, ist der
dreiste Eingriff in die Privatsphäre doch als äußerst problematisch einzustufen
und leider gestaltet sich dieser zum Bedauern der betroffenen Benutzer, für
Fremde oft viel zu einfach.

Hauptursache für diese Problematik ist die oft unverschlüsselte, zentrale
Speicherung der Daten. Damit aber nicht auf den Komfort verzichtet werden muss,
der von üblichen Dateisynchronisationsdiensten geboten wird, werden die zuvor
angesprochenen Probleme bei \sblit umgangen.
