Ereignisse wie "The Fappening" oder diverse Hackerangriffe auf große Firmen, bei
denen Unmengen an Kundendaten gestohlen werden, zeigen wie leicht Mengen an
privaten Daten in falsche Hände fallen können, wenn diese zentral an einem Ort
gespeichert werden.

Geheimdienste oder staatliche Sicherheitsbehörden haben außerdem ein leichtes
Spiel an Userdaten heranzukommen, wenn sie gesammelt auf den Servern eines
Unternehmens gespeichert werden. Diese Unternehmen sind als Anbieter solcher
Cloudspeicher unter Umständen zur Herausgabe der Nutzerdaten gesetzlich
verpflichtet.

Und selbst wenn der duchschnittliche Benutzer nichts zu verbergen hat, ist der
dreiste Eingriff in die Privatsphäre doch als äußerst problematisch einzustufen
und leider gestaltet sich dieser zum Bedauern der betroffenen Benutzer, für
Fremden oft viel zu einfach.
