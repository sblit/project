\subsection{Einführung}
Das Standard Widget Toolkit ist eine Open-Source-Bibliothek für grafische
Benutzeroberflächen entwickelt von Eclipse. Es unterstützt viele Plattformen,
darunter Windows 2000/XP/Vista/7/8, Linux und Mac OS X und bietet eine
Programmschnittstelle, die es eralubt, auf die nativen Widgets des
Betriebssystems zuzugreifen, sofern dies möglich ist. Zusätzlich werden noch
weitere Widgets angeboten, die nicht nativ existieren, aber in dem
Betriebssystem-spezifischen Design gehalten sind, wie zum Beispiel das Tree-
oder Table-Widget.

%
% TODO: Bilder
%
\subsection{Vorteile}
Damit eine möglichst große Reichweite mit \sblit erreicht werden kann, wurde bei
der Entwicklung besoders auf Plattformunabhängigkeit geachtet. um es mit wenig
bis keinem Anpassungsaufwand auf die verschiedenen Plattformen transportieren zu
können. SWT hat nun den Vorteil, dass es auf den verschiedenen Plattformen das
spezifische Aussehen des jeweiligen Betriebssystems hat, obwohl das selbe
Programm ausgeführt und die selben Elemente verwendet werden.

\subsection{Nachteile}
Sich mit einem neuen Framework zu beschäftigen kostet Zeit und Energie. Um
diesen Schritt zu erleichtern, gibt es für die meisten Frameworks für grafische
Oberflächen einen \gls{wysiwygeditor}. Nicht so bei SWT, weshalb
sich die Einarbeitungsphase länger gestaltet, als dies bei anderen Frameworks
der Fall ist. Dennoch überwiegen bei der Implementierung in \sblit die Vorteile,
sodass es den Mehraufwand wert ist.
