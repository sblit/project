\subsection{Einführung}
Für das Gestalten der grafischen Oberfläche wurde bei \sblit \acrshort{swt} verwendet.
Das Standard Widget Toolkit ist eine Open-Source-Bibliothek für grafische
Benutzeroberflächen, entwickelt von Eclipse. Es unterstützt viele Plattformen,
darunter Windows 2000/XP/Vista/7/8, Linux und Mac OS X und bietet eine
Programmschnittstelle, die es erlaubt, auf die nativen Widgets des
Betriebssystems zuzugreifen, sofern dies möglich ist. Zusätzlich werden noch
weitere Widgets angeboten, die nicht nativ existieren, aber in dem
Betriebssystem-spezifischen Design gehalten sind. Beispiele wären das \code{Tree}-
oder \code{Table}-Widget.\cite{swt:tutorial}

\subsection{Vorteile}
\begin{description}
	\descriptionitem{Behält Design der jeweiligen Plattform}
    Um mit \sblit eine möglichst große Reichweite zu erzielen, wurde
    besonders auf Plattformunabhängigkeit geachtet, um \sblit mit wenig
    Anpassungsaufwand auf verschiedensten Betriebssystemen laufen lassen zu können.
	\acrshort{swt} hat den Vorteil, dass grafische Oberflächen auf den verschiedenen Plattformen das
    spezifische Aussehen des jeweiligen Betriebssystems haben, obwohl überall dasselbe
    Programm ausgeführt und die selben Elemente verwendet werden.

	\descriptionitem{Einsteigerfreundlicher Editor}
    Sich mit einem neuen Framework zu beschäftigen kostet Zeit und Energie. Um
    diesen Schritt zu erleichtern, gibt es für die meisten Frameworks für grafische
    Oberflächen einen \gls{wysiwygeditor}. So auch bei \acrshort{swt}, weshalb
    sich die Einarbeitungsphase einfacher gestaltet, als dies bei manch anderen Frameworks
    der Fall ist.
\end{description}

\subsection{Nachteile}
\begin{description}
	\descriptionitem{Ist auf Linux weniger performant}
    Auf allen Plattformen außer Windows (wie zum Beispiel Linux) ist \acrshort{swt} manchmal
    weniger performant, da bestimmte Widgets nicht auf diesen Plattformen vorhanden sind und erst emuliert werden müssen.\cite{swt:disadvantages}

	\descriptionitem{Ist nicht \acrshort{mvc} basiert}
    SWT basiert nicht auf einem \acrshort{mvc}-Design, bietet mit JFace allerdings eine
    \acrshort{mvc}-Abstraktion als Erweiterung.\cite{swt:disadvantages}
\end{description}

\subsection{Aufbau}
Der Aufbau bei \acrshort{swt} ist sehr einheitlich. Es gibt zwei Hauptobjekte:
\code{Display} und \code{Shell}. Jedes Element hat ein Elternelement als Parameter
und jedes Widget noch dazu ein Stil-Bit, das dem Element eine vordefinierte Optik
verleiht. Zudem gibt es noch Container-Klassen auf die ein Layout angewendet werden
kann, um die ihm untergeordneten Elemente zu positionieren.
\listingstart{Beispielklasse für grafische Oberfläche \datei{erklaerungsGUI}}
public class erklaerungsGUI {
	Display display;
	Shell shell;

	public static void main(String[] args) {
		new erklaerungsGUI();
	}

	public erklaerungsGUI(){
		display = new Display();
		shell = new Shell(display);

		GridLayout layout = new GridLayout(2, false);
		shell.setLayout(layout);

		Button pushButton = new Button(shell, SWT.PUSH);
		GridData layoutData = new GridData(SWT.FILL, SWT.FILL, true, true, 1, 1);
		pushButton.setLayoutData(layoutData);
		pushButton.setText("Push");

		Button arrowButton = new Button(shell, SWT.ARROW);
		layoutData = new GridData(SWT.FILL, SWT.FILL, true, true, 1, 1);
		arrowButton.setLayoutData(layoutData);

		shell.setSize(200, 100);
		shell.open();

    // Eventloop
		while (!shell.isDisposed()) {
			if (!display.readAndDispatch())
				display.sleep();
		}
		display.dispose();
	}
}
\end{lstlisting}

\begin{description}
	\descriptionitem{Display}
    Die \code{Display}-Klasse ist eine der Kernkomponenten innerhalb einer \acrshort{swt}-\acrshort{gui}.
    Sie verwaltet Event-Loops, Schriftarten, Farben und die generelle Kommunikation zwischen dem \acrshort{gui}-Thread
    und den anderen Threads. Jedes Programm mit einer \acrshort{swt}-\acrshort{gui} muss ein \code{Display}-Objekt haben.

    \descriptionitem{Shell}
    Die \code{Shell}-Klasse ist die zweite Kernkomponenten einer \acrshort{swt}-\acrshort{gui}.
    Sie stellt das Programmfenster dar und wird das oberste Elternelement aller eingefügten Widgets.
	Das \code{Display}-Objekt muss ein oder mehrere \code{Shell}-Objekte als Unterobjekte haben.
    Erst mit der Funktion \code{shell.open()} wird das Programmfenster gezeichnet.

    \descriptionitem{GridLayout}
    Layouts können auf \code{Composite}-Objekte angewandt werden, um die darin befindlichen
	Elemente zu positionieren. Ein Beispiel für ein solches Objekt ist die \code{Shell}.
    Es gibt je nach Struktur, in der man die Elemente anordnen will, verschiedenste Layouttypen.
    Um jedem Widget eine Layoutinformation, also unter anderem eine Position zuordnen zu können,
	wird die jeweilige \code{LayoutData} benötigt.

    \descriptionitem{Button}
    Hier werden zwei typische Buttons als Beispiel-Widgets verwendet. Unter den Parametern wird die Shell als
    Elternelement angegeben und \code{SWT.PUSH} beziehungsweise \code{SWT.ARROW}
    als Stil-Bit. Das ungleiche Stil-Bit erklärt die unterschiedliche Optik der Buttons.

    \descriptionitem{Eventloop}
    Damit das \acrshort{gui} nicht sofort geschlossen wird, gibt es eine Eventloop, eine Schleife,
	die wartet, bis ein gewisses Event eintritt. In diesem Fall wird
	darauf gewartet, dass das Fenster, also das \code{Shell}-Objekt disposed, zu deutsch
	geschlossen wird.

\end{description}

\begin{figure}[htb]
	\centering
	\includegraphics[]{images/erklaerungsgui.png}
	\label{erklaerungsgui}
  \caption{Beispiel-\acrshort{gui} zur Erklärung}
\end{figure}
