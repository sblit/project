/section{Grafische Oberfläche}

%
%Um dem Benutzer eine einfache zu bedienende Interaktionsmöglichkeit mit sblit zu bieten, gibt es die Grafische Oberfläche. Während der Aufbau des Peer-to-Peer-Links  komplett im Hinterrund %abläuft und sich der User nicht damit auseindandersetzen muss, hat er Kontrolle über die Dateisynchronisation.
%

Mit dem Gedanken, dass ein Dateisynchronisationstool hauptsächlich im Hintergrund arbeitet, ist die grafische Benutzeroberfläche zurückhaltend und einfach gestaltet. Dem Nutzer wird nicht unnötig viel Konfigurationsmöglichkeiten gegeben um leicht den Überblick behalten zu können. 
Beim Starten von sblit erscheint das Icon im System Tray, welches als Ausgang für jegliche Nutzerinaktion dient.

{Bild von System Tray mit Icon}

%
%  TODO: Lücke mit gescheitem Übergang füllen
%

Hier hat der User Zugriff auf die folgenden Dinge(!):
-letzte Änderung innerhalb des konfigurierten sblit-Ordners (1)
-Information über den Fortschritt der Übertragung von einer möglich laufenden Synchronisation (2)
-Möglichkeit zur Abbruch dieser (3)
-Anzeige von aufgetretenen Fehlern (4)
-Link zu sblit-Ordner (5)
-Öffnen des Konfigurationsmenüs  (6) + Abb.3.3

{Bild mit Beschriftung}

%
%  Genauer auf jeden Punkt eingehen
%
/subsubsection{letzte Änderung innerhalb des konfigurierten sblit-Ordners (1)}

/subsubsection{Information über den Fortschritt der Übertragung von einer möglich laufenden Synchronisation (2)}

/subsubsection{Möglichkeit zur Abbruch laufender Synchronisationen (3)}

/subsubsection{Anzeige von aufgetretenen Fehlern (4)}

/subsubsection{Link zu sblit-Ordner (5)}

/subsubsection{Konfigurationsmenü}
Der User kann hier folgende Konfigurationen vornehmen
-Geräte zu den Synchronisationspartnern hinzufügen/entfernen (1)
-Möglichkeit für manuell eingetragene Partnerschaften
-Verschieben des Ortes des sblit-Ordners

{Bild mit Beschriftung}

%
%  Genauer auf jeden Punkt eingehen
%
/subsubsubsection{Geräte zu den Synchronisationspartnern hinzufügen/entfernen}

/subsubsubsection{Möglichkeit für manuell eingetragene Partnerschaften}

/subsubsubsection{Verschieben des Ortes des sblit-Ordners}

Technologien/Plattformunabhängigkeit
[-Java falls noch nicht erwähnt]
-SWT
  -Grundlagen
  -Was ist anders?

-Bouncycastle
  -?