
Um eine Anleitung zur Einbindung von Kommunikationsfunktionalitäten des
\gls{dcl} in projektexterne Anwendungen zur Verfügung zu stellen, wurde das
DCL Integration Tutorial erstellt.
Damit dieses nicht nur von deutschsprachigen Entwicklerinnen und Entwicklern
verwendet werden kann, wurde es in englischer Sprache verfasst.
Das Tutorial soll die Grundzüge des Konzepts hinter \gls{dcl} erklären und
beschreiben, wie in Java-Programmen sowohl verbindungslose Kommunikation über
\glspl{net} des \gls{dcl}, als auch verbindungsorientierte Kommunikation durch
\glspl{appch} umgesetzt werden können.

Um einen schnellen Überblick über das Projekt und die technischen Elemente geben
zu können, wurde ein Folder angefertigt.
Dieser wurde zur Einreichung des Projekts bei diversen Wettbewerben genutzt.

Der Anhang auf den folgenden Seiten enthält das DCL Integration Tutorial, den
Projektfolder sowie die Visitenkarten des Projekts.
