\subsection{Website}
Zu Beginn des Projektes war es Bestandteil der Diplomarbeit eine Website als
Onlinepräsenz zu gestalten, auf der die Projektidee, die Teammitglieder und der
Projektstatus einsehbar sind.

Primär lag aber die Applikation im Fokus unserer Diplomarbeit, weshalb wir uns
an Bootstrap bedient haben, einem offenen CSS-Framework, das viele vordefinierte
Auszeichnungen für HTML-Elemente aller Art bietet.

Viele Frontend-Entwickler veröffentlichen ihre eigenen Bootstrap-Themen und 
machen es Anwendern möglich, diese bestehenden Frontend-Webseiten für eigene
Zwecke umzugestalten und so schnell für einen schönen und einfachen Web-Auftritt
zu sorgen. Namenhafte Webseiten, auf denen verschiedenste Bootstrap-Themen
ausgestellt und -- unter anderem -- zur Verwendung frei gestellt sind wären
startbootstrap.com
