\chapter{Website}
\section{Allgemein}
Zu Beginn der Diplomarbeit war eine Website zu gestalten, auf der die Projektidee,
die Teammitglieder und der Projektstatus einsehbar sind.

\section{Bootstrap}
Primär lag der Fokus in der Diplomarbeit auf der Applikation, weshalb die Website
einfach gehalten ist. Mit dem Einsatz von Bootstrap, einem offenen CSS-Framework,
ist es allerdings trotz geringem Arbeitsaufwand möglich, moderne Webseiten zu gestalten,
indem es viele vordefinierte Klassen zum Gestalten von diversesten HTML-Elementen bietet.

Viele Frontend-Entwickler veröffentlichen ihre eigenen Bootstrap-Themen, also
mit bootstrap-Klassen gestaltete statische HTML-Seiten und machen es Drittanwendern
möglich, diese bestehenden Frontend-Webseiten für eigene Zwecke umzugestalten,
um somit einen schönen und unkomplizierten Web-Auftritt zu ermöglichen. Namenhafte
Webseiten, auf denen verschiedenste Bootstrap-Themen ausgestellt und unter
anderem zur Verwendung frei gestellt sind wären \href{http://startbootstrap.com/}{startbootstrap.com},
\href{http://bootstrapzero.com/}{bootstrapzero.com} oder \href{http://blacktie.co/}{blacktie.co}.

\section{Diplomarbeitswebsite}
Der Webauftritt der Diplomarbeit ist aus solch einem Bootstrap-Thema von
\href{http://startbootstrap.com/}{startbootstrap.com} entstanden, welches optisch und inhaltsmäßig auf unsere
Diplomarbeit angepasst worden ist. Dabei wurde ein, für Bootstrap typisches
One-Page-Design verwendet.

Aufrufbar ist die Diplomarbeitswebsite unter \href{http://sblit.net/da/}{sblit.net/da}

\section{Produktwebsite}
Mit dem Ende der Diplomarbeit hat \sblit nicht nur einen Webauftritt für das Projekt,
sondern auch einen für das Produkt selbst, abseits der schulinternen Richtlinien für Diplomarbeitswebsiten.

Unter \href{http://sblit.net/}{sblit.net} ist die englischsprachige Produktwebsite aufrufbar,
die Besucher der Seite über die wichtigsten Eckdaten unseres Produktes informiert und bei
Abschluss des Projektes auch eine Download-Sektion haben wird, um sich einen Installationsclient herunterladen zu können.

Auch bei dieser Website wurde das für Bootstrap typische One-Page-Design verwendet.
