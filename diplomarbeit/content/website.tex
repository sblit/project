\subsection{Website}
Zu Beginn des Projektes war es Bestandteil der Diplomarbeit eine Website als
Onlinepräsenz zu gestalten, auf der die Projektidee, die Teammitglieder und der
Projektstatus einsehbar sind.

Primär lag aber die Applikation im Fokus unserer Diplomarbeit, weshalb wir uns
an Bootstrap bedient haben, einem offenen CSS-Framework, das viele vordefinierte
Auszeichnungen für HTML-Elemente aller Art bietet.

Viele Frontend-Entwickler veröffentlichen ihre eigenen Bootstrap-Themen, also
mit bootstrap-Klassen desginte, statische HTML-Seiten und machen es Anwendern
möglich, diese bestehenden Frontend-Webseiten für eigene Zwecke umzugestalten
und schnell für einen schönen und einfachen Web-Auftritt zu sorgen. Namenhafte
Webseiten, auf denen verschiedenste Bootstrap-Themen ausgestellt und unter
anderem zur Verwendung frei gestellt sind wären startbootstrap.com,
bootstrapzero.com oder blacktie.co.

Der Webauftritt der Diplomarbeit ist aus solch einem Bootstrap-Thema von
startbootstrap.com entstanden, welches optisch und inhaltsmäßig auf unsere
Diplomarbeit angepasst worden ist. Dabei wurde ein, für Bootstrap typisches
One-Page-Design verwendet.

% TODO: Website-Screenshot miteinbinden.
% + Websiten zu Hyperlinks machen
