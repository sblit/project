
\subsection{Notwendigkeit}
Um \gls{dcl} für eine breite Menge an Anwendungen benutzbar zu machen, muss es möglich sein, alle
Eigenschaften, die ein \gls{p2pnet} zur Kommunikation zwischen den Instanzen dieser Anwendungen
haben muss, erfüllen zu können. Das lässt sich am besten Realisieren, indem der \gls{dcl} so umgesetzt
wird, dass mehrere unterschiedliche \glspl{p2pnet} darüber aufgebaut werden können.
Das hat den Vorteil, auch mit \glslink{p2pnet}{Peer-to-Peer-Netzwerken}, die andere Anforderungen
als bisher existente \glspl{net} haben, auf die bereits bestehende Menge an Hosts aus dem \gls{dcl}
zurückgreifen zu können. So können neue \glspl{net} schnell und stabil zur Verfügung gestellt werden
und neue Teilnehmer erhöhen nicht nur die Verfügbarkeit ihres \glslink{net}{Netzwerks}, sondern auch
die aller anderen auf \gls{dcl} aufbauender \glspl{net}.

\subsection{Definition}
Ein \gls{nt} beschreibt ein \gls{net} innerhalb des \gls{dcl} durch Festlegung folgender Eigenschaften:

\begin{description}
	
	\descriptionitem{Adresskonzept}
		Beinhaltet die Länge von Adressen und die Verfahren zur Bildung sowie zur Überprüfung
		von Adressen.
	
	\descriptionitem{Routingverfahren}
		Beinhaltet den Vorgang zur Weiterleitung von Nachrichten, die an andere Teilnehmer des
		\glslink{net}{Netzwerks} adressiert sind.
	
\end{description}

\subsection{Notation}
Für jeden \gls{nt} gibt einen sogenannten \gls{ntid}, mit dem er identifiziert wird. Dieser setzt sich aus
einem Typstring im Stil von Java Package Names und einem Attributstring für Untereigenschaften zusammen.

\subsection{\gls*{cnet}}
Das im \gls{dcl} standardmäßig verwendete \gls{net} ist das \gls{cnet}. Der Typstring des \gls{nt} des
\gls{cnet} ist \code{\cnettype}.

Adressen werden im \gls{cnet} standardmäßig durch Anwendung des \gls{hash} \gls{sha1} auf die öffentlichen
\gls{rsa}-Schlüssel erzeugt und haben eine Länge von 20 Bytes, was der \gls{digestlength} von \gls{sha1}
entspricht.

Das Format des Attributstrings des \gls{nt} des \gls{cnet} ist
\sloppy{\mbox{\code{\gls{hash}/Adresslänge}}}.
Der standardmäßige Attributstring für den \gls{nt} des \gls{cnet} ist also
\sloppy{\mbox{\code{\cnetattr}}}.

Daraus ergibt sich der standardmäßige \gls{ntid} des \gls{cnet},
\sloppy{\mbox{\code{\cnetid}}}.
