\chapter{Decentralized Communication Layer}
\renewcommand{\kapitelautor}{Autor: Martin Exner}

Um die Hauptanforderung an die Umsetzung, den Verzicht auf zentrale Server im System, realisieren zu können,
wird ein Peer-to-Peer-Netzwerk benötigt. Über dieses läuft die Kommunikation der Synchronisationsanwendung.
Dabei muss das Netzwerk vollständig dezentral aufgebaut sein, um ohne Server funktionieren zu können.

Dieses Peer-to-Peer-Netzwerk ist in einer separaten Kommunikationsschicht, genannt Decentralized Communication Layer, umgesetzt.
Das ermöglicht es einerseits, DCL für Anwendungen von fremden Entwicklern zu öffnen, sodass diese auf das schon bestehende Netzwerk
zurückgreifen können und nicht erst ein eigenes umsetzen müssen, und andererseits erleichtert die klare Abgrenzung der Funktionen
zwischen Kommunikationsschicht und eigentlicher Synchronisationsanwendung die Umsetzung beider erheblich.

\section{Anforderungen an die Kommunikationsschicht}
