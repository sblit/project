\chapter{Decentralized Communication Layer}
\renewcommand{\kapitelautor}{Autor: Martin Exner}

\section{Einleitung}
% TODO niemand hat besondere berechtigungen, niemand kann die funktion des netzwerks beeinträchtigen
Um die Hauptanforderung an die Umsetzung, den Verzicht auf zentrale Server im System, realisieren zu können,
wird ein \gls{p2pnet} benötigt. Über dieses läuft die Kommunikation der Synchronisationsanwendung.
Dabei muss das Netzwerk vollständig dezentral aufgebaut sein, um ohne Server funktionieren zu können.

Dieses \gls{p2pnet} ist in einer separaten \gls{cl}, \gls{dcl}, umgesetzt.
Das ermöglicht es einerseits, \gls{dcl} für Anwendungen von fremden Entwicklern zu öffnen, sodass diese auf das schon bestehende Netzwerk
zurückgreifen können und nicht erst ein eigenes umsetzen müssen, und andererseits erleichtert die klare Abgrenzung der Funktionen
zwischen \gls{cl} und eigentlicher Synchronisationsanwendung die Umsetzung beider erheblich.

\section{Anforderungen an die \gls{cl}}

Das \gls{p2pnet} des \gls{dcl} muss eine Reihe von Eigenschaften aufweisen,
um für die Anwendung eingesetzt werden zu können. Diese Eigenschaften decken sich im wesentlichen mit
üblichen Anforderungen an herkömmliche, nicht dezentrale Netzwerke und lauten wie folgt:
\begin{description}
	\item [{Adressierung:}]
		
		Den Teilnehmern müssen eindeutige Adressen zugewiesen werden können, die auf ihre Echtheit überprüfbar sind.
	
	
	\item [{Routing:}]
		
		Zwischen Teilnehmern müssen anhand ihrer Adressen Kommunikationskanäle aufgebaut werden können.

\end{description}

Durch den dezentralen Ansatz von \gls{dcl} gestaltet sich die Realisierung dieser Eigenschaften jedoch
anders als das beispielsweise für ein übliches Computernetzwerk oder das Internet der Fall wäre.
So können Adressen nicht von zentralen, dazu bemächtigten Behörden vergeben werden und das Routing
kann nicht von speziellen Teilnehmern des Netzwerks in einer hierarchischen Organisation erfolgen.



\section{Adressierung}

\subsection{Notwendigkeit}
Um in einer auf dem \gls{dcl} basierenden Anwendung Synchronisationgruppen bilden zu können, ist es
nötig, die Teilnehmer dieser Gruppen als solche erkennen zu können.
Das erfordert wiederum die permanente und eindeutige Adressierung dieser Teilnehmer.
Da sich die öffentlichen IP-Adressen der meisten privaten Internetanschlüsse und somit des Großteils der
Zielgruppe von \sblit periodisch ändern, eignen sich diese jedoch nicht als Adressen im \gls{dcl}. Es
wird also ein anderes Adresskonzept benötigt, mit dem ohne höhere Behörde allen Teilnehmern eindeutige
Adressen zugewiesen werden können, die auf ihre Echtheit überprüfbar sind.

Es ist nicht ausreichend, die Teilnehmer ihre Adressen willkürlich selbst bestimmen zu lassen und eine
Funktion zu implementieren, die überprüft, ob eine neu generierte Adresse im Netzwerk schon existiert,
da dieser Ansatz keinerlei Sicherheit vor einer absichtlichen Übernahme der Adresse eines anderen
Teilnehmers durch einen Angreifer bietet.

Dieses Unterkapitel beschreibt die Adressierung im \gls{cnet}.

% TODO schwierigkeit: keine zentrale Stelle, Eindeutigkeit

\subsection{Adressierung mit \gls*{rsa}}
\glslink{aenc}{Asymmetrische Verschlüsselungsverfahren} wie \gls{rsa} eignen sich durch ihre Eigenschaften
ausgezeichnet für die Adressierung innerhalb eines Netzwerks, in dem alle Teilnehmer die gleichen
Berechtigungen haben und in dem keine höhere Instanz existiert, die Adressen vergeben und diese
verifizieren kann.
\tags{key-info, schlüsselpaar, asymmetrisch}

Bei asymmetrischen Verschlüsselungsverfahren kommen sogenannte Schlüsselpaare, bestehend aus zwei
Schlüsseln, zum Einsatz. Die Besonderheit liegt darin, dass eine Nachricht, die mit einem Schlüssel
aus dem Schlüsselpaar verschlüsselt wurde, bei asymmetrischen Verfahren im Gegensatz zu symmetrischen
Verfahren nicht mit dem selben Schlüssel auch wieder entschlüsselt werden kann, sondern ausschließlich
mit dem anderen Schlüssel des Schlüsselpaars.
Gleichzeitig kann aus einem Schlüssel eines Schlüsselpaars der dazugehörige andere Schlüssel des
Schlüsselpaars nicht in absehbarer Zeit berechnet werden.

Dadurch wird es möglich, ein Adressierungssystem umzusetzen, das die Anforderungen im Bezug auf
Überprüfbarkeit der Adressen erfüllt. Dazu wird eine Adresse angenommen, die sich aus einem der
beiden Schlüssel aus dem Schlüsselpaar ableitet. Dieser Schlüssel wird bewusst veröffentlicht,
während der andere Schlüssel aus dem Schlüsselpaar geheim gehalten wird.
Der Schlüssel aus dem Schlüsselpaar, der veröffentlicht wird, wird auch \emph{Öffentlicher Schlüssel}
oder \emph{Public Key} genannt, der weiterhin geheim gehaltene Schlüssel \emph{Privater Schlüssel} oder
\emph{Private Key}.

Öffentliche Schlüssel als Grundlage für Adressen haben den Vorteil, dass die Adressen ohne höhere
Behörde oder zentrale Stelle auf ihre Echtheit überprüft werden können und somit fälschungssicher sind.
Ein Mechanismus zur Überprüfung solch einer Adresse wird im nächsten Abschnitt beschrieben.

\subsection{Überprüfung von Adressen}
Dadurch, dass eine mit einem öffentlichen Schlüssel verschlüsselte Nachricht nicht mit wieder mit dem
öffentlichen Schlüssel entschlüsselt werden kann, sondern ausschließlich mit dem dazugehörigen privaten
Schlüssel, kann der Besitz des gesamten Schlüsselpaars bewiesen werden, ohne mehr als den öffentlichen
Schlüssel preisgeben zu müssen: Eine beliebige Folge von Daten wird vom überprüfenden Teilnehmer
generiert, mit der Grundlage der Adresse des zu überprüfenden Teilnehmers, also seinem öffentlichen
Schlüssel verschlüsselt und anschließend an diesen übermittelt. Dort wird die empfangene Nachricht vom zu
überprüfenden Teilnehmer mit seinem privaten Schlüssel, den nur dieser Teilnehmer besitzt, wieder
entschlüsselt und zurück an den überprüfenden Teilnehmer gesendet.

Decken sich die ursprünglich vom überprüfenden Teilnehmer generierten Daten mit denen, die vom zu
überprüfenden Teilnehmer entschlüsselt wurden, ist der Besitz des gesamten Schlüsselpaars, und nicht
lediglich des öffentlichen Schlüssels, bewiesen. In anderen Worten, die vom überprüften Teilnehmer
bekanntgegebene Adresse ist echt.

%TODO konkretes Beispiel

%TODO Erklärung der Challenge mit Nikolas mergen

\subsection{Eindeutigkeit von Adressen}
\label{dcl-addr-uniqueness}
Die Eindeutigkeit der generierten \gls{rsa}-Schlüsselpaare und somit der Adressen kann zwar nicht
garantiert werden, eine Kollision ist jedoch aufgrund der Länge der verwendeten Schlüssel und der
Anzahl der dadurch möglichen Schlüsselpaare dermaßen unwahrscheinlich, dass davon ausgegangen
werden kann, dass eine Kollision praktisch nicht auftritt. \cite{crypto.stackexchange.com/a/2559:rsa-key-collision}

\subsection{Adressformat}

\subsubsection{\gls*{rsa}-Schlüssellänge}
Die für die Adressen im \gls{dcl} verwendeten öffentlichen \gls{rsa}-Schlüssel haben eine Länge von \addrkeybits
Bits, was \addrkeybytes Bytes entspricht.
IPv4 und IPv6-Adressen haben mit jeweils 32 Bits (4 Bytes) bzw. 128 Bits (16 Bytes) vergleichsweise kurze
Adressen. Trotzdem ist eine Adressierung mit 128 Bits mehr als ausreichend und wesentlich längere
Adressen, wie sie beispielsweise bei direkter Verwendung von RSA-Schlüsseln entstehen, bringen
lediglich Nachteile und keinerlei Vorteile mit sich, da der dabei entstehende Overhead bei Nachrichten
eine wesentliche Reduktion der Übertragungseffizienz mit sich bringt.

\subsubsection{Adressverkürzung}
Die im \gls{dcl} verwendeten Adressen decken sich nicht mit den öffentlichen \gls{rsa}-Schlüsseln,
sondern basieren lediglich darauf. Da ohne Weiterverarbeitung der Schlüssel zu lange Adressen entstünden,
wird auf die Daten der öffentlichen Schlüssel ein \gls{hash} angewandt, welcher die Daten in einem ersten
Schritt wesentlich verkürzt. In einem optionalen zweiten Schritt können die bereits verkürzten
Daten weiter verkürzt werden, um unnötig lange Adressen zu eliminieren.

Nachfolgend ist jener Teil des Quellcodes von \gls{dcl} angeführt, welcher im \code{\gls{cnt}}
zur Adressverkürzung angewandt wird.

\javalisting
\begin{minipage}{\linewidth}
\begin{lstlisting}[caption={Adressverkürzung im \gls*{cnt} (Java)},captionpos=b]
Data scaleAddress(Address address) {
	
	Data fullData = address.toData();
	Data hashedData = hash.update(fullData).finish();
	
	if(byteLength == hashedData.length()) return hashedData;
	
	Data scaledData = new Data(byteLength);
	
	for(int i = 0; i < hashedData.length(); i++) {
		int scaledIndex = i % scaledData.length();
		scaledData.setByte(
			scaledIndex,
			(byte)( scaledData.getByte(scaledIndex)
					^ hashedData.getByte(i))
		);
	}
	
	return scaledData;
	
}
\end{lstlisting}
\end{minipage}

\begin{description}
	
	\descriptionitem{\code{address}}
		Das \javaarg \code{address} referenziert ein \code{\gls{Address}}-Objekt, welches den
		als Grundlage für eine Adresse verwendeten öffentlichen Schlüssel enthält.
	
	\descriptionitem{\code{\gls{Data}}}
		\glsdesc{Data}.
	
	\descriptionitem{\code{hash}}
		Die \javainstvar \code{hash} referenziert ein \code{\gls{Hash}}-Objekt, welches den
		im \gls{nt} angegebenen \gls{hash} implementiert.
	
	\descriptionitem{\code{byteLength}}
		Die \javainstvar \code{byteLength} enthält die Ziellänge der Adressen, in bytes.
	
\end{description}

Mit \code{address.toData()} wird der öffentliche Schlüssel der übergebenen Adresse zuerst in Binärdaten
umgewandelt, auf welche dann ein \gls{hash} angewandt wird. Entspricht die Ziellänge von Adressen im
\gls{nt} der \gls{digestlength} des verwendeten \gls{hash}, so wird der \gls{hashval} \code{hashedData}
direkt zurückgegeben.

Anderenfalls wird der \gls{hashval} aus \code{hashedData} auf Binärdaten der Länge \code{byteLength},
gespeichert in \code{scaledData}, abgebildet und diese zurückgegeben.



