\includegraphics[]{systemtray.png}

Mit einem einfachen Klick auf das Icon öffnet sich die Übersicht, in der der Benutzer auf die folgenden Optionen Zugriff bekommt:

\begin{description}
	
	\item[{Letzte Änderungen innerhalb des sblit-Ordners \includegraphics[]{images/legendennr_1_text}]
		Dem Benutzer wird hier eine Auflistung der zuletzt hinzugefügten Dateien geboten. Neben dem an den Dateityp angepassten Bild, wird auch der Dateiname und Ordnerpfad angegeben.

	\item[{Fortschrittsbalken laufender Synchronisationsvorgänge \includegraphics[]{images/legendennr_2_text}]
		Bei laufenden Synchronisationen hat der User die Möglichkeit, den Fortschritt zu verfolgen.

	\item[{Button für das Abbrechen der Synchronisation \includegraphics[]{images/legendennr_3_text}]
		Bei irrtümlichem Hinzufügen von Dateien oder Ähnlichem, hat der Benutzer die Möglichkeit, die laufende Synchronisation mithilfe des Löschen-Buttons abzubrechen.

	\item[{Anzeige von aufgetretenen Fehlern \includegraphics[]{images/legendennr_4_text}]
		Bei Versionskonflikten, die auftreten, wenn 2 Synchronisationspartner die selbe Datei gleichzeitig bearbeiten, sowie bei diversen anderen Fehlern, wird der User benachrichtigt. 

	\item[{Link zum sblit-Ordner \includegraphics[]{images/legendennr_5_text}]
		Um dem Benutzer schnellen Zugriff auf seinen konfigurierten sblit-Ordner zu gestatten, gibt es den Ordner-Button, mit dem sich der sblit-Ordner im Datei-Browser öffnet.

	\item[{Öffnen des Konfigurationsmenüs \includegraphics[]{images/legendennr_6_text}]
		Mit einem Klick auf den Optionen-Button öffnet sich das \ref{sec:Konfigurationsmenü}.
\end{description}