%% LyX 2.1.2 created this file.  For more info, see http://www.lyx.org/.
%% Do not edit unless you really know what you are doing.
\documentclass[a4paper,ngerman,naustrian,DIV=12,BCOR=1cm]{scrbook}
\usepackage[T1]{fontenc}
\usepackage[utf8]{inputenc}
\usepackage{fancyhdr}
\pagestyle{fancy}
\setcounter{secnumdepth}{3}
\usepackage{babel}
\usepackage{textcomp}
\usepackage{url}
\usepackage{makeidx}
\makeindex
\usepackage{graphicx}
\PassOptionsToPackage{normalem}{ulem}
\usepackage{ulem}
\usepackage[unicode=true,
 bookmarks=true,bookmarksnumbered=false,bookmarksopen=false,
 breaklinks=true,pdfborder={0 0 0},backref=false,colorlinks=false]
 {hyperref}
\hypersetup{pdftitle={Diplomarbeit Titel},
 pdfauthor={Wer auch immer},
 pdfsubject={Diplomarbeit},
 pdfkeywords={dies, das}}

\makeatletter

%%%%%%%%%%%%%%%%%%%%%%%%%%%%%% LyX specific LaTeX commands.
\pdfpageheight\paperheight
\pdfpagewidth\paperwidth

%% Because html converters don't know tabularnewline
\providecommand{\tabularnewline}{\\}

%%%%%%%%%%%%%%%%%%%%%%%%%%%%%% Textclass specific LaTeX commands.
\newcommand{\strong}[1]{\textbf{#1}}
\newcommand{\code}[1]{\texttt{#1}}

%%%%%%%%%%%%%%%%%%%%%%%%%%%%%% User specified LaTeX commands.
%%%%%%%%%%%%
% Latex-Vorspann
\usepackage{lastpage}
\usepackage{listings}
\usepackage{blindtext}

%% geht nicht mit jeder Latex Variante, gibt aber ein schöneres Layout
\usepackage{microtype} 

%% Aufzählungen nicht so weit einrücken
\usepackage{enumitem}
%\setitemize{leftmargin=*} 

%\usepackage{caladea}
%\usepackage[T1]{fontenc}
\usepackage{lmodern}

%% für pandoc
%% maximale Breite der Bilder
\setkeys{Gin}{width=0.90\linewidth,keepaspectratio}

\makeatother

\usepackage{listings}
\addto\captionsnaustrian{\renewcommand{\lstlistingname}{Listing}}
\addto\captionsngerman{\renewcommand{\lstlistingname}{Listing}}
\renewcommand{\lstlistingname}{Listing}

\begin{document}
%%%%%%
% Weitere Einstellungen siehe Latex-Vorspann%%%%%%%%%%%%%%%%%%%%%%%%%%%%%%%%%%%%%%%%%%%%%%%%%%%%%%%%%%%%%%%%%%%%%%%%%%%%%%%%%%
% falls man die erste Zeile der Absätze nicht einrücken will
% dann sollte man aber etwas mehr Abstand zwischen den Absätzen erlauben
%%\setlength{\parindent}{0pt}
%%\setlength{\parskip}{1.5ex plus0.5ex minus0.5ex}
% Auch Fußnoten bündig ausrichten
\deffootnote[]{1em}{1em}{\textsuperscript{\thefootnotemark\ }}
% Listen etwas wenige einrücken, erfordert enumitem
\setitemize{leftmargin=*}

%%%%%%%%%%%%%%%%%%%%%%%%%%%%%%%%%%%%%%%%%%%%%%%%%%%%%%%%%%%%%%%%%%%%%%%%%%%%%%%%%%
%  Kopf und Fußzeilen -- links und rechts verschieden 
\newcommand{\kopfseitenummer}{{\bfseries \thepage}}
\newcommand{\kopfkapl}{{\bfseries\leftmark}}
\newcommand{\kopfkapr}{{\bfseries\rightmark}}
\newcommand{\kopfbild}{\includegraphics[width=25mm]{HTL3RLogoRGB}}
\newcommand{\kopfHTL}{Höhere Technische Bundeslehranstalt Wien 3, \\Rennweg 	Abteilung für Informationstechnologie}
\renewcommand{\chaptermark}[1]%
  {\thispagestyle{fancy}\markboth{\thechapter.\ #1}{}}%\thispagestyle{fancy}
\renewcommand{\headrulewidth}{0pt}
%\lhead[\fancyplain{\kopfbild}{\kopfbild}]% li aussen
%      {\fancyplain{\kopfHTL}{\kopfHTL}}% re innen
%\rhead[\kopfHTL]% li innen
%      {\kopfbild}% re aussen

%% mit kapitelautor kann man den Autor festlegen oder auf leer setzen - steht dann in der Fußzeile.
\newcommand{\kapitelautor}{}

%%%
% Alternative: am Rand (Marginale)
%\setlength{\marginparsep}{-5mm}
%\mbox{}\marginpar{\raggedleft\hspace{0pt}Autor: Hans Huber}

%% kopf links: [linke] und {rechte} Seite
\lhead[\kopfbild]{\kopfkapl}
\rhead[\kopfkapr]{\kopfbild}
\chead{}

\lfoot[\kopfseitenummer]{\kapitelautor}
\cfoot[]{}
\rfoot[\kapitelautor]{\kopfseitenummer}

%%
% einfaches "siehe ..." - das Ziel muss man markieren
\newcommand{\kap}[1]{Kapitel~\ref{#1}, Seite~\pageref{#1}}
\newcommand{\siehe}[1]{siehe \kap{#1}}

%%%%%Anfang Titelseite
\pagenumbering{roman}
\title{Diplomarbeit}
\begin{titlepage}
\begin{minipage}[b]{1\columnwidth}
\parbox[b]{50mm}{\includegraphics[width=45mm]{HTL3RLogoRGB}}
\hfill
\parbox[b]{130mm}{\footnotesize \textsc{Höhere Technische Bundeslehranstalt} Wien 3, Rennweg\\
IT \& Mechatronik\\
\\
HTL Rennweg :: Rennweg 89b\\
A-1030 Wien :: Tel +43 1 24215-10 :: Fax DW 18
}\\
\mbox{}
\end{minipage}

\vspace{1cm}


\begin{center}
\textbf{\LARGE{}Diplomarbeit}{\large{}}\\
{\large{}\vspace{15mm}
 }\textbf{\large{}eventuell KURZTITEL}\\
\textbf{\large{}Ausgeschriebener Titel der Diplomarbeit}\\
 \vspace{15mm}
 ausgeführt an der\\
 Höheren Abteilung für Informationstechnologie/Ausbildungsschwerpunkt\\
 der Höheren Technischen Lehranstalt Wien 3 Rennweg\\
 \vspace{1cm}
 im Schuljahr 20??/20??\\
 \vspace{1cm}
 durch\\
 \vspace{0.5cm}
\textbf{\large{}Mitarbeiter Eins (alphabetisch)}\\
\textbf{\large{}Mitarbeiter Zwei}\\
\textbf{\large{}Mitarbeiter Drei}\\
\textbf{\large{}Mitarbeiter Vier}\\

\par\end{center}{\large \par}

\begin{center}
\vspace{20mm}
 \normalsize unter der Anleitung von\\
 \vspace{0.5cm}
 Hauptbetreuer\\
eventuell Nebenbetreuer
\par\end{center}

\begin{center}
\vspace{5mm}
Wien, \today 
\par\end{center}

\end{titlepage}%%%%%%%%%%%%%%%%%%%%% Ende Titelseite %%%%%%%%%%%%%%%%%%%%%%


\addchap*{Kurzfassung}

% Auf Seiten mit einem neuen Kapitel ist keine Kopfzeile -- kann man sich aber wünschen
\thispagestyle{fancy}

Darum geht es.

\blindtext[1]


\addchap*{Abstract}

% mit Kopfzeile
\thispagestyle{fancy}

Thats why.

\blindtext[1]


\addchap*{Ehrenwörtliche Erklärung}

% mit Kopfzeile
\thispagestyle{fancy}

Ich versichere, 
\begin{itemize}
\item dass ich meinen Anteil an dieser Diplomarbeit selbstständig verfasst
habe, 
\item dass ich keine anderen als die angegebenen Quellen und Hilfsmittel
benutzt habe 
\item und mich auch sonst keiner unerlaubten Hilfe bzw. Hilfsmittel bedient
habe.
\end{itemize}
\bigskip{}
Wien, am \today

<eigenhändige Unterschriften aller Teammitglieder>


\addchap*{Präambel}

\thispagestyle{fancy}

Die Inhalte dieser Diplomarbeit entsprechen den Qualitätsnormen für
,,Ingenieurprojekte`` gemäß §\,29 der Verordnung des Bundesministers
für Unterricht und kulturelle Angelegenheiten über die Reife- und
Diplomprüfung in den berufsbildenden höheren Schulen, BGBl. Nr. 847/1992,
in der Fassung der Verordnungen BGBl. Nr. 269/1993, Nr. 467/1996 und
BGBl. II Nr. 123/97.

\vspace{10mm}


Liste der betreuenden Lehrer: 

<{[}Dir|AV|Prof{]}, akad. Grad, Vorname Name Hauptbetreuer> 

<{[}Dir|AV|Prof{]}, akad. Grad, Vorname Name Hauptbetreuer Stellvertreter> 

<{[}Dir|AV|Prof{]}, akad. Grad, Vorname Name Betreuer> ... (in alphabetischer
Reihenfolge des Nachnamens) 

<{[}Dir|AV|Prof{]}, akad. Grad, Vorname Name Betreuer> 

\vspace{10mm}


Liste der Kooperationspartner:%falls vorhanden

%%%%%%%%%%%%%%%%%%%%%%%%%%%%%%%%%%%%%%%%%%%%%%%%%%%%%%%%%%%%%%%%%%%%%%%%%%%%%%%%%%%%%%%%
%Verzeichnisse -- machen wir mit fancy headern
\renewcommand*{\chapterpagestyle}{fancy}
\cleardoublepage{}
\tableofcontents{}
\cleardoublepage{}
\listoftables
\cleardoublepage{}
\listoffigures

\cleardoublepage{}

%hier geht es los mit dem Text - auf einer rechten Seite
\pagenumbering{arabic}
\pagestyle{fancy}
\thispagestyle{fancy} 


\chapter{Beschreibung der Formatierung}

% wer hat diese Kapitel geschrieben oder leer
\renewcommand{\kapitelautor}{Autor: Hans Huber}

\thispagestyle{fancy}


\section{Vorlagen}

In diesen Kapitel gibt es einige Muster für Dinge die oft vorkommen.
Und etwas Blindtext damit man auch volle Seiten sieht.


\subsection{Formatvorlagen}

Alle Formatierungen sollten mit Formatvorlagen vorgenommen werden.
Spätestens bei der Konvertierung in ein Ebook rächen sich diese ,,Sünden``:
Ebooks sind HTML Dokumente mit einer Formatierung mittels CSS.

Auch bei der Umwandlung in interaktive PDFs ist eine konsequente Formatierung
wichtig.


\subsection{Schriften und Absätze}

Hier findet man eine Beschreibung des Layouts -- Details folgen weiter
unten.
\begin{description}
\item [{Schrift:}] dieses \LaTeX{}-Dokument verwendet die Standardschriften.
Die Schriftgröße soll 12\,pt betragen.
\item [{Absatz:}] entweder verwendet man wie in \LaTeX{} einen etwas größeren
Seitenrand oder einen größeren Zeilenabstand. Beides sorgt für bessere
Lesbarkeit. Die erste Zeile eines Absatzes wird etwas eingerückt (nicht
die erste Zeile nach einer Überschrift, nach einem Bild etc.) und
bzw. oder es gibt einen Abstand zwischen den Absätzen. Am Ende und
Anfang einer Seite sollten mindestens zwei Zeilen eines Absatzes sein
(keine Schusterjungen%
\footnote{siehe \url{http://www.typolexikon.de/s/schusterjunge.html}%
} und Hurenkinder%
\footnote{siehe \url{http://www.typolexikon.de/h/hurenkind.html}%
}).
\item [{Blocksatz:}] Alle Texte werden im Blocksatz gesetzt. Die Silbentrennung
ist dann obligatorisch.
\item [{Kapitelüberschriften:}] Überschriften erster Ordnung sollten auf
rechten Seiten beginnen. Über jeder Überschrift sollte ein Abstand
sein. Alle Überschriften müssen mit de nächsten Absatz ,,zusammengehalten``
werden -- keine einsamen Überschriften am Ende einer Seite.
\item [{Inhaltsverzeichnis:}] das Inhaltsverzeichnis sollte möglichst kompakt
sein. Als Gliederung dienen fette Hauptüberschriften und etwas Abstand
über den Zeilen.
\item [{Seitenformat:}] der Ausdruck erfolgt zweiseitig, ein entsprechender
Bundsteg ist zu berücksichtigen%
\footnote{Die Einstellung der Seitenränder ist keinesfalls beliebig. Sie sollte
bewährten Regeln folgen, {[}\ldots{}{]}. Die häufige Zielvorgabe
,,Den Platz auf dem Papier möglichst gut ausnutzen`` ist keine typografische
sondern eine extrem laienhafte Regel. aus \cite{layout}%
}. Nach Rücksprache mit dem Betreuer kann auch eine einseitige Variante
gewählt werden. Bei Bedarf könne auch einzelne Seiten im Querformat
gesetzt werden.
\item [{Kopfzeile:}] die Kopfzeile sollte dieser Vorlage entsprechen. Falls,
nach Rücksprache mit dem Betreuer, der Ausdruck nur in Schwarz-weiß
erfolgt, kann das Logo entfallen.
\item [{Fußzeile:}] hier ist Platz für den Autor des Kapitels und die Seitennummer.
Wie bei technischen Publikationen üblich ist die Einleitung und die
Verzeichnisse mit römischen Seitennummern versehen. Das eigentliche
Dokument wird mit arabischen Ziffern nummeriert. Beide Nummerierungen
sind unabhängig voneinander und beginnen jeweils bei 1.
\item [{Autor:}] Jedes Kapitel muss auch einem Autor haben. Das sieht man
in der Fußzeile oder als Textbox in der Nähe der Überschrift. Alternativ
kann es im Anhang eine Liste geben. Das ist besonders wichtig wenn
es viele Beilagen, z.B. Handbücher ohne direkte Angabe des Autors,
gibt.
\item [{PDF:}] Die PDF Metainformation sollten richtig sein (Autor etc.)
-- siehe Datei/Eigenschaften. Links auf Webseiten, Verweise innerhalb
des Dokuments, das Inhaltsverzeichnis, die Fußnoten usw. sollten ,,klickbar``
sein.
\end{description}

\subsection{Bilder\label{sub:Bilder}}

Das Bild als Gleitobjekt ist genau hier, oder oben auf der Seite,
oder unten, aber immer zentriert mit Nummer und Beschreibung -- wenn
es sinnvoll ist auch mit Querverweis (siehe Abbildung \ref{Bild11}).
Durch Gleitobjekte, d.~h. Bilder oben oder unten auf der Seite statt
,,genau hier``, werden halbleere Seiten durch besonders große Bilder
vermieden.

Wichtig: alle Bilder oder andere Medien z.~B. Screenshots, Audio
oder Video für EBooks und interaktive PDFs sollten mit einen entsprechenden
Quellennachweis versehen sein.

\begin{figure}[tbh]
\begin{centering}
\includegraphics[scale=0.6]{HTL3RLogoRGB}
\par\end{centering}

\protect\caption{Ein Bild}
\label{Bild11}
\end{figure}



\subsection{Tabellen}

In der folgenden Tabelle sieht man: es gibt immer eine Nummer und
eine Beschreibung. Besonders längere Tabellen sollten eventuell als
Gleitobjekt am Ende oder Anfang einer Seite positioniert werden. Geht
die Tabelle über mehrere Seiten so ist die Überschrift zu wiederholen.

\begin{table}[h]
\begin{centering}
\begin{tabular}{|c|c|c|}
\hline 
Überschrift & Wert & noch einer\tabularnewline
\hline 
\hline 
1 & abc & Hallo\tabularnewline
\hline 
2 & def & Latex\tabularnewline
\hline 
\end{tabular}
\par\end{centering}

\protect\caption{So eine tolle Tabelle}
\end{table}



\subsection{Formel}

Etwas Text als Überleitung zu einer Formel:

\selectlanguage{ngerman}%
\[
f(x)=\left\{ \begin{array}{cc}
\log_{8}x & x>0\\
0 & x=0\\
\sum_{i=1}^{5}\alpha_{i}+\sqrt{-\frac{1}{x}} & x<0
\end{array}\right.
\]


\selectlanguage{naustrian}%
Wenn man sehr viele Formeln hat sollte man diese auch nummerieren.
Besonders bei Verweisen ist das sehr sinnvoll.


\subsection{Sourcecode}

Sourcecode sollte in einer Schrift mit fixer Breite sein. Falls man
Verweise braucht sollte man die Listings auch nummerieren. 

% das kann auch ganz oben stehen
% das braucht man nur einmal
\lstset{numbers=left, numberstyle=\tiny, stepnumber=2, numbersep=5pt, showspaces=true, frame=single}
% einmal oder immer was anderes
\lstset{language=C}

% hier könnte man auch aus Dateien lesen
\begin{lstlisting} 
#include <stdio.h>

int main() 
{ 
  printf("Hello world\n"); 
} 
\end{lstlisting} 

Die genaue Formatierung ist freigestellt: Einstellungen wie bunt bzw.
fett, Markierung von Leerzeichen und Zeilennummerierung kann an den
Bedarf der Diplomarbeit angepasst werden. 

Beispiel Java mit anderen Einstellungen -- nur als Beispiel, in der
Diplomarbeit sollte man sich an eine einheitliches Format halten.
Bei längeren Listings muss man eventuell mit Umbrüchen rechnen, oder
man verwendet einen Rahmen der frei angeordnet werden kann (\siehe{sub:Bilder}).

% Einstellungen für die fogenden Listings
% entweder mit \begin{listing} oder in Lyx als Programmlisting
\lstset{numbers=right, numberstyle=\tiny, stepnumber=2, numbersep=5pt, showspaces=false, frame=single}
\lstset{language=Java}

Achtung \LaTeX{}-User: Listing kann keine Umlaute, aber unter \cite{listingtipp}
gibt es eine Lösung.

\inputencoding{latin9}\begin{lstlisting}[caption={Java Beispiel},captionpos=b]
import java.awt.*;  
import java.awt.event.*;
public class AL extends Frame
                 implements WindowListener, ActionListener {
  TextField text = new TextField(20);
  Button b;    
  private int numClicks = 0;
 
  public static void main(String[] args) {
    AL myWindow = new AL("My first window");
    myWindow.setSize(350,100);
    myWindow.setVisible(true);    
  } 
}
\end{lstlisting}
\inputencoding{utf8}


\subsection{Fachbegriffe}

Fachbegriffe in einer Fremdsprache oder Kommandos sollten einheitlich
gekennzeichnet werden. Bei Latex verwendet man dazu ,,logisches Markup``,
bei Word oder Open/Libre-office wird all diesen Wörtern wird eine
Vorlage zugewiesen, das Aussehen wird dann an einer Stelle zentral
festgelegt. 

Als Beispiel soll \emph{Text to Speech}\index{Text to Speech: Umwandlung von Texten in Sprache}
dienen. Solche Wörter sollte natürlich in ein Glossar aufgenommen
werden. 

Oder der Befehl \strong{dir} für die Kommandozeile. Die Angabe von
Dateinamen sollte auch einheitlich sein: entweder \emph{/etc/passwd}
oder \strong{C:\textbackslash{}system32}.


\subsection{Zitieren}

Die Quellenangabe kann in Form eines Vollbelegs in der Fußnote%
\footnote{aus Zitat --- Wikipedia, Die freie Enzyklopädie, \url{http://de.wikipedia.org/w/index.php?title=Zitat},
Abgerufen 2014-09-14%
}(bei technischen Dokumenten eher unüblich) oder als Kurzbeleg am Schluss
der gesamten Arbeit aufgeführt werden. Beim Kurzbeleg sind dabei verschiedene
Formen üblich. Der platzsparendste, aber am wenigsten aussagekräftige
Zitierstil ist die fortlaufende Nummerierung aller zitierten Quellen
{[}123{]}.

Insbesondere in der Informatik üblich ist eine Kombination der ersten
drei Buchstaben des Autorennamens und der letzten beiden Ziffern des
Erscheinungsjahres (z. B. „The04“ für Theisen 2004). Wohl am weitesten
verbreitet ist der vollständige Verfassernamen mit Erscheinungsjahr,
wobei mehrere Quellen desselben Autors innerhalb eines Jahres durch
fortlaufende Buchstaben kenntlich gemacht werden (z. B. „Theisen 2004c“).
Weniger üblich, aber am aussagekräftigsten ist die Quellenangabe unter
Hinzufügung eines Schlagwortes, das den mit der Materie vertrauten
Leser zumeist bereits die zitierte Quelle erkennen lässt, z. B. in
der Form „Theisen (Wissenschaftliches Arbeiten, 2004)“.

Obwohl mehrere Zitierstile bzw. Zitiertechniken zur Verfügung stehen,
werden in einem Dokument üblicherweise nicht mehrere verwendet; ein
ausgewählter Zitierstil wird im gesamten Dokument konsequent beibehalten.
Ein gute Übersicht bietet \cite{wiki:zitat}.


\subsubsection{Quellenverzeichnis}

Unter \LaTeX{} kann das Programm Bib\TeX{} zur Erstellung von Literaturangaben
verwendet werden. 
\begin{itemize}
\item Links auf Wikipedia sollten vermieden werden.
\item Jeder Link sollte mit einem Abfragedatum versehen sein.
\item Das Literaturverzeichnis kommt an das Ende des Dokuments.
\end{itemize}
Viele Details dazu findet man bei \cite{wiki:zitat}.


\subsubsection{Rechtliches zum Zitieren}

Achtung: nicht gekennzeichnete Zitate (Plagiate) führen zu einer negativen
Beurteilung der Diplomarbeit.

Nach \cite{wiki:quelle}:

§ 57 des österreichischen Urheberrechtsgesetzes\cite{ris57} enthält
detaillierte Vorschriften über die Quellenangabe, unter anderem: Werden
Stellen oder Teile von Sprachwerken nach §\,46 vervielfältigt, so
sind sie in der Quellenangabe so genau zu bezeichnen, dass sie in
dem benutzten Werke leicht aufgefunden werden können. In den Erläuterungen
(ErlRV) heißt es: Bei Entlehnungen aus umfangreichen Werken muss also
in der Quellenangabe auch die Seite, der Abschnitt, das Kapitel oder
der Akt, wo sich die entlehnte Stelle befindet, angeführt werden (Dillenz,
Materialien zum österreichischen Urheberrecht, 134, zitiert nach \cite{dittrich},
S. 621)

2002 nahm der österreichische OGH zur Frage der Quellenangabe in der
Entscheidung Riven Rock Stellung: Nach § 57 Abs 4 UrhG bedarf die
Unterlassung einer Quellenangabe der Rechtfertigung durch die im redlichen
Verkehr geltenden Gewohnheiten und Gebräuche. Bei Auslegung dieser
Bestimmung ist eine Abwägung der Interessen des Urhebers mit jenen
des zur freien Werknutzung Berechtigten nach dem Verständnis loyaler,
den Belangen des Urhebers mit Verständnis gegenübertretenden, billig
und gerecht denkenden Benutzern (Vinck aaO § 63 Rz 2) geboten und
danach zu beurteilen, ob dem freien Werknutzer neben der Nennung des
Autors/Verlags auch die Nennung des Namens des Übersetzers von in
einer Rundfunksendung verlesenen Roman-Zitaten zumutbar ist.


\section{Bad Practice}

Was man vermeiden sollte -- diese Dinge führen zur Mehrarbeit und
verursachen zusätzlichen Stress in der hektischen Zeit knapp vor dem
Abgabetermin.


\paragraph{Stil}
\begin{itemize}
\item Extrem lange, geschachtelte Sätze und/oder endlose Textpassagen ohne
Gliederung durch Absätze.\nopagebreak

\begin{itemize}
\item Vielleicht bzw. sinnvollerweise lassen Sie den Text auch von einer
,,außenstehenden Person`` lesen. 
\end{itemize}
\item Aufzählungen im Text statt Listen. Wie man hier sieht dürfen bei Listen
auch mehrere Sätze stehen.
\item \uline{Unterstreichen} ist ein Relikt aus ,,Schreibmaschinen-Zeiten``.
\item Eine Diplomarbeit ist keine Erzählung. Natürlich kann man als ,,ich``
oder ,,wir`` auf ,,unsere Probleme`` eingehen, aber im Allgemeinen
ist ein formaler, beschreibender und technischer Stil einzuhalten.
\item Eine Diplomarbeit ist auch keine Email oder SMS: Schreiben Sie ganze
Sätze ohne kryptische Abkürzungen und Smileys.
\item Ein Mindestmaß an Interpunktion wird vorausgesetzt. Eventuell lassen
Sie den Text durch eine kundige Person Ihres Vertrauens korrigieren.
\item Es gibt viele verschiedene Striche, und alle sehen verschieden aus:
Gedankenstriche, Bindestriche und Minus kommen in einer Diplomarbeit
häufig vor.
\item Weitere Wörter die Ihren Betreuer verzweifeln lassen -- natürlich
nur bei übermäßiger Verwendung

\begin{itemize}
\item Welcher/Welches, Hierbei
\end{itemize}
\end{itemize}

\paragraph{Technik}
\begin{itemize}
\item (viele) händische Formatierungen statt Formatvorlagen.
\item zusätzliche manuelle Seitenumbrüche oder Leerzeilen für ein ,,schöneres``
Layout. Es gibt bei den Absatzformatierungen tolle Möglichkeiten für
Abstände vor und nach einem Absatz bzw. zum Beeinflussen des Textflusses.
\item Arbeiten Sie mit dem Programm statt gegen das Programm:

\begin{itemize}
\item Verweise als fixer Text. Nutzen Sie die Möglichkeiten der Textverarbeitung.
\item Dinge die ,,kompliziert`` einzugeben sind, sind meist falsch --
richtige Lösungen sind in allen Programmen auch ,,leicht`` zu erreichen%
\footnote{Oder Sie verwenden ein für Ihre Zwecke schlecht geeignetes Programm.%
}.
\end{itemize}
\item Kontrollieren Sie beim fertigen PDF die Angaben unter Datei / Eigenschaften
-- dort sollten sinnvolle Dinge stehen. Bei Latex wird dazu das Paket
\texttt{\code{\texttt{hyperref}}} verwendet.
\end{itemize}

\section{Details zu Formatierung}


\subsection{Schriftarten}

Die Word- und Libreoffice-vorlage verwenden etwas andere Schriften
als das Latex Dokument.


\section{Beispiele}


\subsection{Zitieren mit Latex}

Am Beispiel der URL \url{http://de.wikibooks.org/wiki/LaTeX-Kompendium}
und des Buches ,,\LaTeX{}: Einführung``.

Man braucht eine \texttt{.bib} Datei mit den notwendigen Informationen:

\inputencoding{latin9}\begin{lstlisting}[language={[LaTeX]TeX}]
@book{kopka1991latex,
  title={LaTeX: Einf�hrung},
  author={Kopka, Helmut and Rahtz, Sebastian},
  volume={2},
  year={1991},
  publisher={Addison-Wesley} 
}

@online{latexKomp,
  author = {},
  title ={LaTeX-Kompendium - Wikibooks, Sammlung freier Lehr-, 
Sach- und Fachb�cher},
  url = {http://de.wikibooks.org/wiki/LaTeX-Kompendium},
  lastchecked = {2014.09.14},
}
\end{lstlisting}
\inputencoding{utf8}

Diese Einträge werden dann im Text verwendet: \texttt{\textbackslash{}cite\{kopka1991latex\}
}und das Ergebnis sieht dann so \cite{kopka1991latex} und so \cite{latexKomp}
aus. Gleichzeitig erscheinen diese Einträge auch im Literaturverzeichnis
am Ende des Dokuments (Format: \texttt{plaindin}).


\subsection{Formatierungen }

Der Autor kann natürlich auch eingreifen und zum Beispiel\\
Zeilenumbrüche erzwingen und \\
\\
Leerzeilen -- aber das sollte man nicht machen. \newpage{}

Ein Seitenumbruch kostet mich auch nicht mehr als einen müden Lacher,
ist aber noch seltener wirklich sinnvoll.

Schriftgrößen:
\begin{itemize}
\item {\tiny{}tiny} 
\item {\scriptsize{}scriptsize} 
\item {\footnotesize{}footnotesize} 
\item {\small{}small} 
\item normalsize 
\item {\large{}large} 
\item {\Large{}Large} 
\item {\LARGE{}LARGE} 
\item {\huge{}huge} 
\item {\Huge{}Huge} \end{itemize}
\begin{enumerate}
\item Nummerierungen
\item sind 
\item auch 
\item easy
\end{enumerate}
\textbf{bissl was fettes} \textit{italienisches} \uline{für unten
drunter}.


\subsection{Fülltext}

\Blindtext


\chapter{Planung}

%das komplette nächste Kapitel wird in der externen Datei diplomarbeit2.tex gespeichert. Es wird an diese Stelle im Dokument eingebaut.
%Damit ist es möglich, mehrere Personen an diverse Teile der Diplomarbeit arbeiten zu lassen.

\section{Kapitel aus der anderen
Datei}\label{kapitel-aus-der-anderen-datei}

Dieses Kapitel wurde als \emph{diplomarbeit2.md} geschrieben und dann
mit \emph{pandoc} in \TeX~umgewandelt.

\begin{lstlisting}[language=bash]
pandoc --listings -s diplomarbeit2.md -o diplomarbeit2.tex 
\end{lstlisting}

Wie man sieht ist das ganz einfach, sogar Listings sind möglich. Und nun
zu einem Bild.

\begin{figure}[htbp]
\centering
\includegraphics{HTL3RLogo.png}
\caption{Der Text steht unterhalb}
\end{figure}

Achtung: Pandoc skaliert die Bilder nicht! Hier hilft nur eine
vorhergehende Skalierung des Bildes. Oder nachträgliches Editieren --
ganz einfach die passende Breite in der \emph{.tex} Datei ausbessern.

Man kann auch die Breite aber auch durch La\TeX~Befehle angeben -- das
ändert aber die Standardbreite aller folgenden Bilder!

\setkeys{Gin}{width=0.6\textwidth,}

\begin{figure}[htbp]
\centering
\includegraphics{HTL3RLogo.png}
\caption{Das kleinere Bild}
\end{figure}

\setkeys{Gin}{width=2cm}

\begin{figure}[htbp]
\centering
\includegraphics{HTL3RLogo.png}
\caption{Das ganz kleine Bild}
\end{figure}

Auch Listen sind kein Problem, wichtig sind nur Leerzeilen zwischen den
Listenpunkten. Hier sieht man eine einfache Aufzählung.

\begin{itemize}
\item
  wichtig
\item
  auch ganz lange Texte können bei Listen geschrieben werden.

  Sogar mehrere Absätze sind möglich.
\item
  Ende der Liste.
\end{itemize}

Welches Zeichen am Anfang der Liste steht ist dabei leicht einzustellen,
im \emph{pandoc} Manual gibt es nähere Infos:

\begin{enumerate}
\def\labelenumi{\arabic{enumi}.}
\item
  eins
\item
  zwei

  \begin{enumerate}
  \def\labelenumii{\roman{enumii}.}
  \itemsep1pt\parskip0pt\parsep0pt
  \item
    zwei eins -- Mindestens 4 Zeichen eingerückt
  \item
    zwei zwei
  \end{enumerate}
\item
  drei. \emph{Pandoc} zählt richtig, das Zeichen am Anfang der Zeile ist
  nur ein Muster!
\end{enumerate}



\chapter{Umsetzung}

\Blindtext


\chapter{Ergebnisse}

% wer hat diese Kapitel geschrieben oder leer
\renewcommand{\kapitelautor}{Autor: Susi Sorglos}

\blindmathpaper\Blindtext


\chapter{Evaluation}

% wer hat diese Kapitel geschrieben oder leer 
\renewcommand{\kapitelautor}{Autor: Blindtext}

\Blindtext

\Blinddocument\Blindtext\Blinddocument[2]\Blindtext\Blinddocument[5]\Blindtext\Blinddocument[10]\Blindtext\Blindtext

%%%%%%%%%%%%%%%%%%%%%%%%%%%%%%%%%%%%%%%%%%%%%%%%%%%%%%%%%%%%%%%%%%%%%%%%%%%%%%%%%%%%%%%%%%
% wer hat diese Kapitel geschrieben oder leer
\renewcommand{\kapitelautor}{}

\appendix

\chapter{Anhang 1\label{chap:Anhang-1}}

was auch immer

\printindex{}

\bibliographystyle{plaindin}
\bibliography{diplom}

\end{document}
